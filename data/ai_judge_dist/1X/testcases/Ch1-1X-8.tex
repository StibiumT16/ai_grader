\documentclass{article}

\usepackage{CJKutf8}
\usepackage{setspace}


\title{第一章作业-组合}

\begin{document}
\begin{CJK}{UTF8}{gbsn}
\setstretch{1.25}
\date{}


\maketitle

\section*{1.1}

\subsection*{(1)} 

该问题是如何将 \( m \) 个男生和 \( n \) 个女生排成一行,使得任何两个男生不相邻(其中 \( m \leq n + 1 \))。解决方案如下:

首先,女生先排好队,然后在女生队列中的空隙(包括队列的两端)插入男生。由于有 \( n \) 个女生排成一行,这样就会形成 \( n + 1 \) 个空隙。

即从这 \( n + 1 \) 个空隙中选择 \( m \) 个空隙来插入男生,即 \( C(n+1, m) \)。

组合数 \( C(n, k) \) 的计算公式是:
\[
C(n, k) = \frac{n!}{k!(n-k)!}
\]

空隙的方案数可以表示为:
\[
C(n+1, m) = \frac{(n+1)!}{m!(n+1-m)!}
\]

另外,由于男生和女生各自内部的排列也会产生不同的方案,需要乘以男生的排列数 \( m! \) 和女生的排列数 \( n! \)。

所以,最终的方案数是:
\[
m! \times n! \times C(n+1, m)
\]

将组合数的表达式代入,得到:
\[
m! \times n! \times \frac{(n+1)!}{m!(n+1-m)!} = \frac{n! \times (n+1)!}{(n+1-m)!}
\]

\subsection*{(2)} 

将 \( n \) 个女生视为一个整体后,实际上是在排列 \( m+1 \) 个单元(\( m \) 个男生加上 1 个女生整体)。这些单元的排列方式有 \( (m+1)! \) 种。

在每一种排列中,\( n \) 个女生内部也可以互相排列,她们之间的排列方式有 \( n! \) 种。

则排列方案的总数为:

\[
(m + 1)! \times n!
\]

其中,\( (m + 1)! \) 表示 \( m \) 个男生加上一个女生整体的排列数,\( n! \) 表示 \( n \) 个女生内部的排列数。

\subsection*{(3)} 
首先,将男生 $A$ 和女生 $B$ 看作一个整体,这样就有 $m-1$ 个男生和 $n-1$ 个其他女生,加上 $A$ 和 $B$ 这个整体,一共有 $m+n-2$ 个单位。

这 $m+n-2$ 个单位可以以任意顺序排列,所以排列的总数是 $(m+n-2)!$。

接下来,因为男生 $A$ 和女生 $B$ 可以互换位置,所以对于每一种排列,男生 $A$ 和女生 $B$ 都有 $2$ 种不同的排列方式。

因此,男生 $A$ 和女生 $B$ 相邻的排列总数是 $2$ 倍的 $(m+n-2)!$。

最后,我们得到男生 $A$ 和女生 $B$ 相邻的排列方案总数是 
\[
2(m+n-2)!
\]

\section*{1.2}
\subsection*{(1)} 

首先,考虑6个男生的排列方式。由于圆桌没有首尾之分,因此男生的排列方式为:
\[
(6-1)! = 5! = 120
\]

接着,考虑在6个男生形成的空隙中插入5个女生。这里我们有6个空隙,需要从中选择5个来插入女生,且女生之间不能相邻。这是一个排列问题,因此插入女生的方式有:
\[
P(6, 5) = \frac{6!}{(6-5)!} = 6 \times 5 \times 4 \times 3 \times 2 = 720
\]

最后,将男生和女生的排列方式相乘,得到总的排列方案数:
\[
5! \times P(6, 5) = 120 \times 720 = 86400
\]

\subsection*{(2)} 
首先,考虑6个男生的排列方式。由于圆桌没有首尾之分,因此男生的排列方式为:
\[
(6-1)! = 5! = 120
\]

接着,考虑在6个男生形成的空隙中插入5个女生。这里我们有6个空隙,所以有6种插入方法,同时5个女生自身有$5!$种排列方案:
总的排列方案数:
\[
5! \times 5!  \times 6 = 86400
\]
\subsection*{(3)} 
首先将A和两个男生视为整体,另外8人首先入座,有$(8-1)!$种方案,这个整体插入8个空隙,同时在男生中选出2个男生是$A(6,2)$
总的排列方案数:
\[
(8-1)! \times 8  \times A(6,2) =1,209,600
\]
\section*{1.3}
根据公式 $k \cdot k! = (k + 1) \cdot k! - k! = (k + 1)! - k!$,得出
\[
1! + 2! + \cdots + n! = (n + 1)! - 1.
\]

\section*{1.4}
首先,将这两个数分解成质因数的形式:

\[
10^{40} = (2 \times 5)^{40} = 2^{40} \times 5^{40}
\]

\[
20^{30} = (2^2 \times 5)^{30} = 2^{60} \times 5^{30}
\]

接下来,找出这两个数的公共质因数部分。可以看出,它们都包含$2$的幂和$5$的幂。公共部分是$2$的最小幂次和$5$的最小幂次,即$2^{40}$和$5^{30}$。

因此,$10^{40}$和$20^{30}$的公因数形式为:
\[
2^a \times 5^b
\]
其中,$0 \leq a \leq 40$,$0 \leq b \leq 30$。

$a$可以取$41$个值(从$0$到$40$),$b$可以取$31$个值(从$0$到$30$)。所以,公因数的数目是这两个数的组合,即:
\[
\text{公因数的数目} = 41 \times 31 = 1271
\]

所以,$10^{40}$与$20^{30}$的公因数的数目是$1271$。

\section*{1.5}
考虑一个n位数(n大于1),首位有9种选择;剩下的数里如果要有k个0($k<=n-1$),则可能的排列有$9 ^ {n-1-k} \cdot C(n-1, k)$,对结果的总贡献为$9 \cdot k \cdot 9 ^ {n-1-k} \cdot C(n-1, k)$,之后对k从1到n-1求和。

n = 2时,结果为9;n = 3时,结果为180;n = 4时,结果为2700;n = 5时,结果为36000;n = 6时,结果为450000。
1000000有六个0,加和结果为488895。
\section*{1.6}
首先,我们将 \( n \) 个小球中的 \( r \) 个小球分别放入 \( r \) 个盒子中,确保每个盒子至少有一个小球。剩下的 \( n - r \) 个小球可以自由地放入 \( r \) 个盒子中。

将这个问题转化为在 \( n - r \) 个小球之间放置 \( r - 1 \) 个隔板,来将小球分成 \( r \) 份。隔板可以放在小球之间的任意位置,包括小球的两端。这样,我们实际上是在 \( n - r + r - 1 = n - 1 \) 个位置中选择 \( r - 1 \) 个位置放置隔板。

因此,方案数就是从 \( n - 1 \) 个位置中选择 \( r - 1 \) 个位置的组合数,即
\[
C(n - 1, r - 1)
\]
\section*{1.7}

首先,为每个盒子分配 $k$ 个球。由于有 $r$ 个盒子,所以这一步共分配了 $rk$ 个球。
    
接下来,将剩余的 $n - rk$ 个球放入 $r$ 个盒子中,这时盒子可以为空。
    
这个问题可以转化为在 $n - rk$ 个剩余的小球之间放置 $r - 1$ 个隔板,来将小球分成 $r$ 份。隔板可以放在小球之间的任意位置,包括小球的两端。这样,我们实际上是在 $n - rk + r - 1$ 个位置中选择 $r - 1$ 个位置放置隔板。

因此,方案数就是从 $n - rk + r - 1$ 个位置中选择 $r - 1$ 个位置的组合数,即
\[
C(n - rk + r - 1, r - 1)
\]
\section*{1.8}
首先,将5个不同的小球进行全排列,有$5!$种方法。接着,我们需要将3个空盒子插入到这5个球的排列中,使得空盒子不相邻。由于5个球排列后会形成6个空隙(包括两端),我们需要从这6个空隙中选择3个来放置空盒子,即$C(6, 3)$种方法。计算得到
    \[
    C(6, 3) = \frac{6!}{3!(6-3)!} = \frac{6 \times 5 \times 4}{3 \times 2 \times 1} = 20
    \]
    种。
综合以上步骤,总方案数为
\[
5! \times C(6, 3) = 120 \times 20 = 2400
\]

\section*{1.9}
\subsection*{(1)} 
先选定横作标,再选定纵坐标,其方案数为:
$C(10,2)*C(6,2)=675$
\subsection*{(2)} 
按边长分类:

边长为1的正方形:9X5=45

边长为 2的正方形:(9-1)X(5-1)=32

边长为 3 的正方形:(9-2)(5-2)=21

边长为4的正方形:(9-3)X(5-3)=12

边长为 5的正方形:(9-4)X(5-1)=5

故总的以 x-y 平面上A 点作顶点的正方形的数目,按加法原理可得数目为115
\section*{1.10}
\subsection*{(1)} 
0的数量为i个,即选i个相同的,n-i个不同的 因此选取方案数总数为$\sum_{i=0}^{n}C(n, i) = 2 ^ n$。
\subsection*{(2)} 
思路同上,方案数为$\sum_{i=0}^{n}C(2n+1, i) $
计算需要做一些变化
由于$\sum_{i=0}^{n}C(2n+1, n-i) = \sum_{i=0}^{n}C(2n+1, i)$;


所以$2 \cdot \sum_{i=0}^{n}C(2n+1, i) = \sum_{i=0}^{n}C(2n+1, n-i) + \sum_{i=0}^{n}C(2n+1, i) = \sum_{i=0}^{n}C(2n+1, n+i+1) + \sum_{i=0}^{n}C(2n+1, i) = \sum_{i=0}^{2n+1} C(2n+1, i) = 2 ^ {2n+1}$,因此$\sum_{i=0}^{n}C(2n+1, i) = 2 ^ {2n}$


\section*{1.11}
当使用第1台机器的学生为n个时,使用第 2台机器的学生也为n,从m个学生中选出 2n 个使用这两台机器
剩余的学生可以任意使用剩下的机器的组合数为 $C(m,2n)C(2n,n)3(m-2n)$。根据加法原理有:
有$\sum_{k=0}^{\lfloor m/2 \rfloor}C(m, k) \cdot C(m-k, k) \cdot 3^{m-2k}$种方案
\section*{1.12}
首先,总方案数是从$2n$位中选取$n$位填入0的方案数,为 $C(n, 2n)$。

接下来,我们来计算不符合要求的方案数。这些方案是在$2n$位中,存在至少一个奇数位 $2p+1$(其中 $p$ 是从0到$n-1$的整数),在这个位置之前(包括这个位置),1的数目多于0的数目。在这种情况下,我们可以将这个位置之后的所有1和0互换,从而得到一个包含 $n+1$ 个1和 $n-1$ 个0的$2n$位数。这样的方案数是 $C(n+1, 2n)$。

因此,不符合要求的方案数为 $C(n+1, 2n)$。

最后,我们用总的方案数减去不符合要求的方案数,得到满足条件的方案数:
\[
  ans = C(n, 2n) - C(n+1, 2n) = \frac{(2n)!}{n!(2n-n)!} - \frac{(2n)!}{(n+1)!(2n-(n+1))!}
\]
\[
= \frac{(2n)!}{n!n!} - \frac{(2n)!}{(n+1)!(n-1)!} = \frac{(2n)!}{(n+1)!n!}
\]

\end{CJK}
\end{document}
