% Homework template for Inference and Information
% UPDATE: September 26, 2017 by Xiangxiang
\documentclass[a4paper]{article}
\usepackage{ctex}
\ctexset{
proofname = \heiti{证明}
}
\usepackage{amsmath, amssymb, amsthm}
% amsmath: equation*, amssymb: mathbb, amsthm: proof
\usepackage{moreenum}
\usepackage{mathtools}
\usepackage{url}
\usepackage{bm}
\usepackage{enumitem}
\usepackage{graphicx}
\usepackage{subcaption}
\usepackage{booktabs} % toprule
\usepackage[mathcal]{eucal}
\usepackage[thehwcnt = 3]{iidef}

\thecourseinstitute{清华大学}
\thecoursename{组合数学}
\theterm{2024年秋季学期}
\hwname{CH3 基础}
\slname{\heiti{解}}


\begin{document}
\courseheader
\name{*** }

\begin{enumerate}[]
  \setlength{\itemsep}{3\parskip}
  
\item
\begin{solution}
    \begin{equation*}
        G(x) = \sum_{n=0} ^ {\infty} n^3 x^n 
    \end{equation*}
    已知
    \begin{align*}
        \sum_{n=0} ^ {\infty} x^n &= \frac{1}{1-x} \\
        \sum_{n=0} ^ {\infty} nx^n &= x (\frac{1}{1-x})' = \frac{x}{(1-x)^2} \\
        \sum_{n=0} ^ {\infty} n^2 x^n &= \frac{x(1+x)}{(1-x)^3} \\
        G(x)=\sum_{n=0} ^ {\infty} n^3 x^n &= \frac{x(x^2+4x+1)}{(1-x)^4} \\
    \end{align*}
\end{solution}


\item
\begin{solution}
\begin{equation*}
    a_n = \binom{n+3}{3} = \binom{n+2}{3} + \binom{n+2}{2}
\end{equation*}
令
\begin{equation*}
    G_m(x) = \sum_{n=0}^{\infty} \binom{n+m}{m} x^n
\end{equation*}
可得
\begin{align*}
    G_3(x)&=\sum_{n=0}^{\infty} \binom{n+3}{3} x^n  =1+\sum_{n=1}^{\infty} \binom{n+3}{3} x^n\\
          &=x (\sum_{n=1}^{\infty} \binom{n+2}{3} x^{n-1})  + 1+ \sum_{n=1}^{\infty} \binom{n+2}{2} x^n \\
          &=x G_3(x) + \sum_{n=0}^{\infty} \binom{n+2}{2} x^n = xG_3(x)+G_2(x) \\
    G_3(x)&=\frac{G_2(x)}{1-x}
\end{align*}
同理可得
\begin{equation*}
    G_2(x) = \frac{G_1(x)}{1-x} , G_1(x) = \frac{G_0(x)}{1-x}
\end{equation*}
其中 
\begin{equation*}
    G_0(x) = \sum_{n=0}^{\infty} x^n = \frac{1}{1-x}
\end{equation*}
层层带回,可得
\begin{equation*}
    G_3(x) = \frac{1}{(1-x)^4}
\end{equation*}
从而也可得更为一般的结论
\begin{equation*}
    G_m(x) = \frac{1}{(1-x)^{m+1}}
\end{equation*}

\end{solution}


\item 
\begin{solution}
    令
    \begin{equation*}
        b_n=(n+1)^3
    \end{equation*}
    根据3.1结果,则有
    \begin{equation*}
        B(x)=\frac{A(x)-a_0}{x} = \frac{x^2+4x+1}{(1-x)^4}
    \end{equation*}
    令
    \begin{equation*}
        c_n=\sum_{k=1}^{n+1}k^3 = \sum_{k=0}^{n} b_n
    \end{equation*}
    所以得到
    \begin{equation*}
        C(x) = \frac{B(x)}{1-x} = \frac{x^2+4x+1}{(1-x)^5}
    \end{equation*}
    所以母函数为 $\frac{x^2+4x+1}{(1-x)^5}$
    
\end{solution}

\item 
\begin{solution}
    \begin{align*}
        B(x)&=\sum_{n=0}^{\infty} b_n x^n = a_0+\sum_{n=1}^{\infty} (a_n-a_{n-1})x^n \\
        &= \sum_{n=0}^{\infty} a_n x^n - x \sum_{n=1}^{\infty} a_{n-1}x^{n-1} \\
        &= A(x)-xA(x) = \frac{4-3x}{1+x-x^3}
    \end{align*}
\end{solution}


\end{enumerate}
\end{document}
\begin{equation}
\end{equation}

%%% Local Variables:
%%% mode: late\rvx
%%% TeX-master: t
%%% End:
