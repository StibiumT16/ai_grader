% MIT License

% Copyright (c) 2022 Chiyuru

% Permission is hereby granted, free of charge, to any person obtaining a copy of this software and associated documentation files (the "Software"), 
% to deal in the Software without restriction, including without limitation the rights
% to use, copy, modify, merge, publish, distribute, sublicense, and/or sell
% copies of the Software, and to permit persons to whom the Software is
% furnished to do so, subject to the following conditions:

% The above copyright notice and this permission notice shall be included in all copies or substantial portions of the Software.

% THE SOFTWARE IS PROVIDED "AS IS", WITHOUT WARRANTY OF ANY KIND, EXPRESS OR IMPLIED, INCLUDING BUT NOT LIMITED TO THE WARRANTIES OF MERCHANTABILITY,
% FITNESS FOR A PARTICULAR PURPOSE AND NONINFRINGEMENT. IN NO EVENT SHALL THE AUTHORS OR COPYRIGHT HOLDERS BE LIABLE FOR ANY CLAIM, DAMAGES OR OTHER
% LIABILITY, WHETHER IN AN ACTION OF CONTRACT, TORT OR OTHERWISE, ARISING FROM,
% OUT OF OR IN CONNECTION WITH THE SOFTWARE OR THE USE OR OTHER DEALINGS IN THE SOFTWARE.

\documentclass[UTF8]{ctexart}

\usepackage{amsmath}
\usepackage{cases}
\usepackage{cite}
\usepackage{graphicx}
\usepackage[margin=1in]{geometry}
\geometry{a4paper}
\usepackage{fancyhdr}
\pagestyle{fancy}
\fancyhf{}


\title{组合数学第三次作业}
\author{***}
\date{\today}
\pagenumbering{arabic}

\begin{document}

% \fancyhead[L]{驰雨Chiyuru}
\fancyhead[C]{The name of the experiment}
\fancyfoot[C]{\thepage}

\maketitle
% \tableofcontents

\section{第三章习题(基本和进阶)}

\subsection{}

设 $a_{n}= n^{3},(n >= 0)$,求数列 $\{a_{n}\}$ 的母函数,化简至封闭形式.

$\{a_{n}\}$的母函数为$S_{x}=\sum_{n=0}^{+\infty}{n^{3}x^{n}}$。

已知$\sum_{n=0}^{+\infty}{x^{n}}=\frac{1}{1-x}$,此式子对两边求导:$\sum_{n=0}^{+\infty}{nx^{n-1}}=\frac{1}{(1-x)^{2}}$。

两边乘以x,有$\sum_{n=0}^{+\infty}{nx^{n}}=\frac{x}{(1-x)^{2}}$,再两边求导,有$\sum_{n=0}^{+\infty}{n^{2}x^{n-1}}=\frac{x+1}{(1-x)^{3}}$。

两边乘以x,有$\sum_{n=0}^{+\infty}{n^{2}x^{n}}=\frac{x^{2}+x}{(1-x)^{3}}$,再两边求导,有$\sum_{n=0}^{+\infty}{n^{3}x^{n-1}}=\frac{x^{2}+4x+1}{(1-x)^{4}}$。

再两边乘以x,得到$\sum_{n=0}^{+\infty}{n^{3}x^{n}}=\frac{x^{3}+4x^{2}+x}{(1-x)^{4}}$。

即$S_{x}=\frac{x^{3}+4x^{2}+x}{(1-x)^{4}}$。

\subsection{}
设 $a_{n}= \binom{n+3}{3},(n >= 0)$,求数列 $\{a_{n}\}$ 的母函数,化简至封闭形式.

$\{a_{n}\}$的母函数为$S_{x}=\sum_{n=0}^{+\infty}{\frac{(n+3)(n+2)(n+1)}{6}x^{n}}$。

已知$\sum_{n=0}^{+\infty}{x^{n}}=\frac{1}{1-x}$,对此式子两边乘以$x^{3}$,有$\sum_{n=0}^{+\infty}{x^{n+3}}=\frac{x^{3}}{1-x}$。

两边求导,有$\sum_{n=0}^{+\infty}{(n+3)x^{n+2}}=\frac{3x^{2}-2x^{3}}{(1-x)^{2}}$。

再两边求导,有$\sum_{n=0}^{+\infty}{(n+3)(n+2)x^{n+1}}=\frac{2x(x^{2}-3x+3)}{(1-x)^{3}}$。

再两边求导,有$\sum_{n=0}^{+\infty}{(n+3)(n+2)(n+1)x^{n}}=\frac{6}{(1-x)^{4}}$。

对比上式和母函数,可得$S_{x}=\frac{1}{(1-x)^{4}}$。

\subsection{}

设 $a_{n}= \sum_{k=1}^{n+1}{k^{3}},(n >= 0)$,根据习题1求数列 $\{a_{n}\}$ 的母函数,化简至封闭形式.

$\{a_{n}\}$的母函数为$S_{x}=1+(1^{3}+2^{3})x^{1}+(1^{3}+2^{3}+3^{3})x^{2}...$。

对等式两边同时乘以x,得到
$xS_{x}=1^{3}x+(1^{3}+2^{3})x^{2}+(1^{3}+2^{3}+3^{3})x^{3}...$。

上下两式相减,看出$(1-x)S_{x}=1+2^{3}x^{1}+3^{3}x^{2}+4^{3}x^{3}...=\sum_{n=0}^{+\infty}{n^{3}x^{n-1}}=\frac{x^{2}+4x+1}{(1-x)^{4}}$。

即有$S_{x}=\frac{x^{2}+4x+1}{(1-x)^{5}}$。


\subsection{}

数列 $\{a_{n}\}$ 的母函数为$\frac{4-3x}{(1-x)(1+x-x^{3})}$,定义$b_{0}=a_{0},b_{1}=a_{1}-a_{0},b_{n}=a_{n}-a_{n-1}$,求$\{b_{n}\}$的母函数。


设$A_{x}=\frac{4-3x}{(1-x)(1+x-x^{3})}$,

$A_{x}=a_{0}+a_{1}x+a_{2}x^{2}+a_{3}x^{3}...$

$B_{x}=a_{0}+(a_{1}-a_{0})x+(a_{2}-a_{1})x^{2}+(a_{3}-a_{2})x^{3}....$

$B_{x}+xA_{x}=a_{0}+a_{1}x+a_{2}x^{2}+....=A_{x}$

$B_{x}=A_{x}(1-x)=\frac{4-3x}{(1+x-x^{3})}$

\subsection{}

$\{a_{n}\}$的母函数为$S_{x}=\sum_{n=0}^{+\infty}C(n+k,k)x^{n}}$。

已知$\sum_{n=0}^{+\infty}{x^{n}}=\frac{1}{1-x}$,对式子的两边同时取组合数有

\end{document}
