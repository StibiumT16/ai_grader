\documentclass[a4paper]{article}
\usepackage{ctex}
\ctexset{
proofname = \heiti{证明}
}
\usepackage{amsmath, amssymb, amsthm}
% amsmath: equation*, amssymb: mathbb, amsthm: proof
\usepackage{moreenum}
\usepackage{mathtools}
\usepackage{url}
\usepackage{bm}
\usepackage{enumitem}
\usepackage{graphicx}
\usepackage{subcaption}
\usepackage{booktabs} % toprule
\usepackage{geometry}

\usepackage[mathcal]{eucal}
\usepackage[thehwcnt = 3]{iidef}

\geometry{a4paper,scale=0.7}
\thecourseinstitute{清华大学}
\thecoursename{组合数学}
\theterm{2024年秋季学期}
\hwname{作业}
\slname{\heiti{解}}
\begin{document}
\courseheader
\name{***}

\begin{enumerate}
  \setlength{\itemsep}{3\parskip}
  \item 设$a_n=n^3(n\geqslant 0)$,求数列$\left\{a_n\right\}$的母函数,化简至封闭形式。
  \begin{solution}
    设母函数为$G_a(x)$,可由母函数的定义得到:
    $$G_a(x) = \sum\limits_{n=0}^{+\infty }a_nx^n=\sum\limits_{n=0}^{+\infty}n^3x^n$$
    下面将其化简为封闭形式:
    设序列$b_n = 1,c_n = n, d_n = n^2$,其对应母函数分别为$G_b(x),G_c(x), G_d(x)$,则有$c_n=nb_n, d_n=nc_n, a_n=nd_n$。由母函数的性质可知,有如下关系式:$G_c(x)=xG_b^\prime(x), G_d(x)=xG_c^\prime(x),G_a(x)=xG_d^\prime(x)$。又因为$G_b(x)=\frac{1}{1-x}$,则依次求导乘以$x$,即可顺次得到$G_c(x),G_d(x), G_a(x)$,即:
    \begin{equation}
    \begin{aligned}
    G_c(x)&=xG_b^\prime(x)\\
    &=x\frac{1}{(1-x)^2}\\
    &=\frac{x}{(1-x)^2}
    \end{aligned}
    \end{equation}

    \begin{equation}
        \begin{aligned}
        G_d(x)&=xG_c^\prime(x)\\
        &=x\frac{1+x}{(1-x)^3}\\
        &=\frac{x(1+x)}{(1-x)^3}
        \end{aligned}
    \end{equation}

    \begin{equation}
        \begin{aligned}
        G_a(x)&=xG_d^\prime(x)\\
        &=x\frac{x^2+4x+1}{(1-x)^4}\\
        &=\frac{x^3+4x^2+x}{(1-x)^4}
        \end{aligned}
    \end{equation}
  \end{solution}
  \item 设$a_n=\begin{pmatrix}
  n+3 \\ 3
  \end{pmatrix}(n\geqslant 0)$,求数列$\left\{a_n\right\}$的母函数,化简至封闭形式。
  \begin{solution}
  设母函数为$G_a(x)$,可由母函数的定义得到:
  $$G_a(x) = \sum\limits_{n=0}^{+\infty }a_nx^n=\sum\limits_{n=0}^{+\infty}C(n+3,3)x^n$$
  下面将其化简为封闭形式:
  \begin{equation}
  \begin{aligned}
  G_a(x) &= \sum\limits_{n=0}^{+\infty}\frac{(n+3)(n+2)(n+1)}{6}x^n\\
  &=\frac{1}{6}\sum\limits_{n=0}^{+\infty}n^3x^n+\sum\limits_{n=0}^{+\infty}n^2x^n+\frac{11}{6}\sum\limits_{n=0}^{+\infty}nx^n+\sum\limits_{n=0}^{+\infty}x^n\\
  &=\frac{1}{6}\frac{x^3+4x^2+x}{(1-x)^4}+\frac{x(1+x)}{(1-x)^3}+\frac{11}{6}\frac{x}{(1-x)^2}+\frac{1}{1-x}\\
  &=\frac{1}{(1-x)^4}
  \end{aligned}
  \end{equation}


  \end{solution}
  \item 设$a_n=\sum\limits_{k=1}^{n+1}k^3(n\geqslant 0)$,基于习题1求数列$\left\{a_n\right\}$的母函数,化简至封闭形式。
  \begin{solution}
    设数列$b_n = n^3,c_n = (n+1)^3=b_{n+1}$,对应的母函数分别为$G_b(x),G_c(x)$,$a_n$的母函数为$G_a(x)$。则有
    $G_b(x)=\frac{x^3+4x^2+x}{(1-x)^4}$,由母函数的性质可知:
    \begin{equation}
        \begin{aligned}
        G_c(x) &= (G_b(x)-b_0)/x\\
        &=\frac{x^2+4x+1}{(1-x)^4}
        \end{aligned}
    \end{equation}
    又因为有如下关系$a_n = \sum\limits_{k=0}^{n}c_k$,由母函数的性质可知:
    \begin{equation}
        \begin{aligned}
        G_a(x) &= \frac{G_c(x)}{(1-x)}\\
        &=\frac{x^2+4x+1}{(1-x)^5}
        \end{aligned}
    \end{equation}

  \end{solution}
  \item 设数列$\left\{a_n\right\}$的母函数为
  $$A(x)=\frac{4-3x}{(1-x)(1+x-x^3)}$$
  定义
  $$b_0=a_0,b_1=a_1-a_0,\cdots,b_n=a_n-a_{n-1},\cdots$$
  求数列$\left\{b_n\right\}$的母函数,化简至封闭形式。
  \begin{solution}
    设数列$\left\{b_n\right\}$的母函数为$B(x)$,由递推可以得到如下关系:
    $$a_n = \sum\limits_{k=0}^{n}b_k$$
    由母函数的性质,得到如下关系:
    $$B(x)=A(x)(1-x)=\frac{4-3x}{1+x-x^3}$$
  \end{solution}
\end{enumerate}
\end{document}
\begin{equation}
\end{equation}

%%% Local Variables:
%%% mode: late\rvx
%%% TeX-master: t
%%% End:
