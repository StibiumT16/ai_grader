\documentclass{article}
\usepackage{amsmath}
\usepackage{amssymb}
\usepackage{ctex}
\begin{document}

\title{ 第三章基础}
\author{***}
\date{}
\maketitle

3.1\\
设$A(x)=\sum_{n = 0}^{\infty}n^3x^n$,$B(x)=\sum_{n = 0}^{\infty}n^2x^n$,$C(x)=\sum_{n = 0}^{\infty}nx^n$\\
$C(x)=(\sum_{n = 0}^{\infty}x^n)'x=(\frac{1}{1-x})'x=\frac{x}{(1-x)^2}$\\
$B(x)=(\sum_{n = 0}^{\infty}nx^n)'x=(\frac{x}{(1-x)^2})'x=\frac{(1+x)x}{(1-x)^3}$\\
$A(x)=(\sum_{n = 0}^{\infty}{n^2}x^n)'x=(\frac{(1+x)x}{(1-x)^3})'x=\frac{{x^2}+4x+1}{(1-x)^4}$\\
所以母函数为$A(x)=\frac{{x^2}+4x+1}{(1-x)^4}$\\


3.2\\
引理:(数列前缀和)若$a_n=\sum_{k = 0}^{n}b_n$,则$A(x)=\frac{B(x)}{1-x}.$证明参考组合数学讲义第三章定理3.3\\
$a_n=\binom{n+3}{3}=\sum_{k = 0}^{n}\binom{k+2}{2}$,母函数$A(x)$\\
$b_n=\binom{n+2}{2}=\sum_{k = 0}^{n}\binom{k+1}{1}$,母函数$B(x)$\\
$c_n=\binom{n+1}{1}=\sum_{k = 0}^{n}1$,母函数$C(x)$\\
$C(x)=\frac{\sum_{i = 0}^{\infty}x^i}{1-x}=\frac{1}{(1-x)^2}$\\
因为$b_n=\sum_{k = 0}^{n}c_n$,所以$B(x)=\frac{C(x)}{1-x}=\frac{1}{(1-x)^3}$\\
因为$a_n=\sum_{k = 0}^{n}b_n$,所以$A(x)=\frac{B(x)}{1-x}=\frac{1}{(1-x)^4}$\\
所以母函数为$\frac{1}{(1-x)^4}$\\

3.3\\
设$b_n=n^3(n\geqslant0)$\\
$A(x)=a_0+a_1x+a_2x^2+\ldots$\\
$=b_1+(b_1+b_2)x+(b_1+b_2+b_3)x^2+\ldots$\\
$=b_1(1+x+x^2+\ldots)+b_2x(1+x+x^2+\ldots)+b_3x^2(1+x+x^2+\ldots)$\\
$=\frac{b_0+b_1x+b_2x^2+\ldots}{(1-x)x}$\\
$=\frac{x^2+4x+1}{x(1-x)^5}$\\

3.4\\
$B(x)=b_0+b_1x+b_2x^2+\ldots$\\
$=(a_0+a_1x+a_2x^2+\ldots)-x(a_0+a_1x+a_2x^2+\ldots)$\\
$=A(x)(1-x)$\\
$=\frac{4-3x}{1+x-x^3}$\\


\end{document}
