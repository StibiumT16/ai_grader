\documentclass{article}
\usepackage{graphicx,ctex,amsmath} % Required for inserting images
\usepackage{fullpage}

\title{组合数学 HW4(基本)}
\date{November 2024}

\begin{document}

\maketitle

\section*{Problem 4.1}
\subsection*{(1)}
\begin{align*}
    G_n-3G_{n-1}+G_{n-2}&=F_{2n}-3F_{2n-2}+F_{2n-4}\\
    &=F_{2n-1}+F_{2n-2}-3F_{2n-2}+F_{2n-4}\\
    &=F_{2n-1}-F_{2n-2}-F_{2n-2}+F_{2n-4}\\
    &=-F_{2n-2}+F_{2n-3}+F_{2n-4}\\
    &=0.
\end{align*}
因此得证。

\subsection*{(2)}
设母函数为$G(x)$,有\[G(x)-3xG(x)+x^2G(x)=G_0+(G_1-3G_0)x=x.\] 
因此,\[G(x)=\frac{x}{1-3x+x^2}.\]


\section*{Problem 4.2}
递推关系式为$a_n-a_{n-1}+a_{n-2}=0$,我们有\[1=(1-x+x^2)G(x)=a_0+(a_1-a_0)x.\]
因此,$a_0=1,a_1-a_0=0$,即$a_0=a_1=1$.


\section*{Problem 4.3}
特征根为$3,-1$,特征方程为$(x-3)(x+1)=x^2-2x-3$. 由此,递推关系为$a_n-2a_{n-1}-3a_{n-2}=0$.


\section*{Problem 4.4}
\[a_n-2a_{n-1}+a_{n-2}=a_{n-1}-2a_{n-2}+a_{n-3}=5.\]
因此
\[a_n-2a_{n-1}+a_{n-2}-a_{n-1}+2a_{n-2}-a_{n-3}=0,\]
即\[a_n-3a_{n-1}+3a_{n-2}-a_{n-3}=0.\]
特征方程$x^3-3x^2+3x-1$有三重根$x=1$. 因此,$a_n$的形式为$An^2+Bn+C$. 由$a_0=1,a_1=2,a_2=8$得
\[a_n=\frac{5}{2}n^2-\frac{3}{2}n+1.\]


\section*{Problem 4.5}
设满足条件的字符串数目为$a_n$,其中$A$开头的有$b_n$个. 由于$A$开头的可能为$AB$加上任意$n-2$长度字符串或者$A$加上任意满足条件的$n-1$长度字符串,其中$AB$加上满足条件的$n-2$长度字符串的情况被重复计算了,因此我们有
\[b_n=4^{n-2}+a_{n-1}-a_{n-2}.\]
而满足条件的字符串可以为$A$打头的$b_n$个或者$B/C/D$加上任意满足条件的$n-1$长度字符串,因此
\[a_n=3a_{n-1}+b_n.\]
将$b_n$的表达式代入得$a_n-4a_{n-1}+a_{n-2}=4^{n-2}$.
因此有$(a_n-4^n)-4(a_{n-1}-4^{n-1})+(a_{n-2}-4^{n-2})=0$.
对数列$a_n-4^n$,特征多项式为$x^2-4x+1=0$,特征根为$2\pm\sqrt{3}$. 因此$a_n$的形式为$A(2+\sqrt{3})^n+B(2-\sqrt{3})^n+4^n$. 由初始值$a_1=0$, $a_2=1$解得
\[A=-\dfrac{2\sqrt{3}+3}{6},\quad B=\dfrac{2\sqrt{3}-3}{6}.\]
因此,
\[a_n=-\dfrac{2\sqrt{3}+3}{6}(2+\sqrt{3})^n+\dfrac{2\sqrt{3}-3}{6}(2-\sqrt{3})^n+4^n.\]



\section*{Problem 4.6}
不妨设$n$为偶数($n$为奇数只需调换$B,C$两根柱子),要移动编号$n$的盘,我们首先需要移动到$C$柱子有编号$1,2,\dots,n-1$的盘,$A$柱子为空,$B$柱子只有编号$n$的盘的状态,这需要至少$2^{n-1}$次移动. 随后,我们需要移动$n-2$号盘至$B$,我们只能移动使得$1,2,\dots,n-3$号盘都在$A$柱子上再移动$n-2$号盘到$B$,这需要至少$2^{n-3}$次移动. 设共需最少$a_n$次移动,那么移动上述$2^{n-1}+2^{n-3}$次后我们只需考虑$1$至$n-3$号盘子就行,移动它们需要至少$a_{n-3}$次移动,因此有递推关系
\[a_n=a_{n-3}+2^{n-1}+2^{n-3}=a_{n-3}+5\cdot 2^{n-3}.\]
解得$a_n=\frac{5}{7}\cdot 2^n+C$,其中$C$是常数. 带入初始值$a_1=1$, $a_2=2$, $a_3=5$,有
\begin{align*}
    a_n =
    \begin{cases}
    \frac{5\cdot 2^n-5}{7} & n\bmod 3\equiv 0 \\
    \frac{5\cdot 2^n-3}{7} & n\bmod 3\equiv 1 \\
    \frac{5\cdot 2^n-6}{7} & n\bmod 3\equiv 2
    \end{cases}.
\end{align*}


\section*{Problem 4.7}
设方案数为$a_n$. 其中,若最后两个字母不同,有$(k-1)a_{n-1}$种方案,若最后两个字母相同,有$(k-1)a_{n-2}$种方案. 因此有递推关系
\[a_n=(k-1)a_{n-1}+(k-1)a_{n-2}.\]
特征方程为$x^2-(k-1)x-(k-1)=0$,特征根为$\frac{k-1\pm \sqrt{k^2+2k-3}}{2}$. 则$a_n$的形式为
\[a_n=A\cdot \left(\frac{k-1+ \sqrt{k^2+2k-3}}{2}\right)^n+B\cdot  \left(\frac{k-1- \sqrt{k^2+2k-3}}{2}\right)^n.\]
由初值$a_1=k,a_2=k^2$解得
\[A=\dfrac{k(k-1)+k\sqrt{k^2+2k-3}}{2(k-1)\sqrt{k^2+2k-3}},B=\dfrac{-k(k-1)+k\sqrt{k^2+2k-3}}{2(k-1)\sqrt{k^2+2k-3}}.\]
因此,
\begin{align*}
    a_n=&\dfrac{k(k-1)+k\sqrt{k^2+2k-3}}{2(k-1)\sqrt{k^2+2k-3}}\cdot \left(\frac{k-1+ \sqrt{k^2+2k-3}}{2}\right)^n\\
    &+\dfrac{-k(k-1)+k\sqrt{k^2+2k-3}}{2(k-1)\sqrt{k^2+2k-3}}\cdot  \left(\frac{k-1- \sqrt{k^2+2k-3}}{2}\right)^n.
\end{align*}


\section*{Problem 4.8}
\begin{align*}
    \sum_{k=1}^n k^4&=\sum_{k=1}^n k(k+1)(k+2)(k+3)-6\sum_{k=1}^n k^3-11\sum_{k=1}^n k^2-6\sum_{k=1}^n k\\
    &=\frac{n(n+1)(n+2)(n+3)(n+4)}{5}-6\cdot\left(\frac{n(n+1)}{2}\right)^2-11\frac{n(n+1)(2n+1)}{6}-6\cdot\frac{n(n+1)}{2}\\
    &=\frac{n^5}{5}+\frac{n^4}{2}+\frac{n^3}{3}-\frac{n}{30}.
\end{align*}


\section*{Problem 4.9}
\subsection*{(1)}
若取了$n$,方案数为$1$到$n-2$中取$k-1$个符合条件的数的方案数,即$f(n-2,k-1)$. 若没有取$n$,方案数为$1$到$n-1$中取$k$个符合条件的数的方案数,即$f(n-1,k)$. 即
\[f(n,k)=f(n-2,k-1)+f(n-1,k).\]

\subsection*{(2)}
下证:$f(n,k)=\binom{n-k+1}{k}$. 首先,对$(n,k)=(1,1),(2,1),(2,2)$该式满足. 假设该式对任意$k\leq k_0\leq n\leq n_0$成立,有
\begin{align*}
    f(n_0,k_0)&=f(n_0-2,k_0-1)+f(n_0-1,k_0)\\
    &=\binom{n_0-k_0}{k_0-1}+\binom{n_0-k_0}{k_0}\\
    &=\binom{n_0-k_0+1}{k_0}.
\end{align*}
归纳可知对任意正整数$k\leq n$,$f(n,k)=\binom{n-k+1}{k}$成立.

\subsection*{(3)}
只需去除同时选取$1,n$的方案数,即在$3$至$n-2$共$n-4$个数中取$k-2$个数的方案数,为$f(n-4,k-2)$. 因此:
\[g(n,k)=f(n,k)-f(n-4,k-2).\]


\section*{Problem 4.10}
\subsection*{(1)}
设方案数为$a_n$,另设铺设$1\times n$路径加上一个直角边长为1的等腰直角三角形(在$1\times n$路径右边且斜边由左上到右下)的方案数为$b_n$. 考虑铺设加上的一个直角边长为1的等腰直角三角形可以有两种方向,对个方向每种方案都只有唯一的方法将其铺设成$1\times (n+1)$的路径,而不通过先铺设成$1\times n$路径加上一个直角边长为1的等腰直角三角形而直接得到$1\times (n+1)$的路径只能是$1\times n$的路径上加一个正方形,因此有
\[a_{n+1}=2b_n+a_n.\]
另一方面,铺设$1\times n$路径加上一个直角边长为1的等腰直角三角形要么先铺设$1\times n$路径,要么先铺设$1\times (n-1)$路径加上一个直角边长为1的等腰直角三角形再铺设一个斜边长为2的等腰直角三角形,因此有
\[b_n=a_n+b_{n-1}.\]
联立可知
\[3a_n-a_{n-1}=2(a_n+b_{n-1})=2b_{n}=a_{n+1}-a_{n}.\]
即\[a_{n+1}-4a_n+a_{n-1}=0.\]
特征方程为$x^2-4x+1=0$,特征根为$2\pm \sqrt{3}$. 因此,$a_n$的形式为
\[a_n=A(2+\sqrt{3})^n+B(2-\sqrt{3})^n.\]
由初值$a_1=3,a_2=11$解得
\[A=\frac{3+\sqrt{3}}{6},B=\frac{3-\sqrt{3}}{6}.\]
即
\[a_n=\frac{3+\sqrt{3}}{6}\cdot (2+\sqrt{3})^n+\frac{3-\sqrt{3}}{6}\cdot(2-\sqrt{3})^n.\]

\subsection*{(2)}
设所求砖数总和为$c_n$,另设铺设$1\times n$路径加上一个直角边长为1的等腰直角三角形的方案砖数总和为$d_n$. 同(1)中分析可知
\[c_{n+1}=2d_n+c_n+a_{n+1};\]
\[d_n=c_n+d_{n-1}+b_n.\]
联立可知
\[2c_n+(c_n-c_{n-1}-a_n)+(a_{n+1}-a_{n})=2(c_n+d_{n-1}+b_n)=2d_{n}=c_{n+1}-c_{n}-a_{n+1}.\]
即\[c_{n+1}-4c_n+c_{n-1}=2a_{n+1}-2a_n.\]
代入$a_n$的表达式知$c_n$的形式为
\[c_n=(An+B)(2+\sqrt{3})^n+(Cn+D)(2-\sqrt{3})^n.\]
代入初值$c_1=5,c_2=36,a_1=3,a_2=11$得
\[A=\frac{2+\sqrt{3}}{3},B=\frac{\sqrt{3}}{18},C=\frac{2-\sqrt{3}}{3},D=-\frac{\sqrt{3}}{18}.\]
因此
\[c_n=\left(\frac{2+\sqrt{3}}{3}n+\frac{\sqrt{3}}{18}\right)(2+\sqrt{3})^n+\left(\frac{2-\sqrt{3}}{3}n-\frac{\sqrt{3}}{18}\right)(2-\sqrt{3})^n.\]
                                                                                                                                               
\end{document}
