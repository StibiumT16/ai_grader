\documentclass[a4paper]{ctexart}
\usepackage{geometry}		%页面设置
\usepackage{amsmath}		%数学公式
\usepackage{graphicx}		%图像
%\usepackage{subfig}		%子图像
\usepackage{listings}		%表单,用于插入代码
\usepackage{xcolor}		%颜色,用于插入代码
\usepackage{enumerate}	%编号
\usepackage{siunitx}		%带SI单位的数字
\usepackage{multirow}		%表格合并行
\usepackage{listings}
\usepackage{amssymb}
\usepackage{subfigure}

\usepackage{algorithm}
\usepackage{algorithmic}

%\usepackage[framed,numbered,autolinebreaks,useliterate]{mcode}
%\usepackage{textcomp} % 必须加上,否则报错

%\usepackage[section]{placeins}	%强制浮动体与章节对齐

\newcommand*{\dif}{\mathop{}\!\mathrm{d}}	%定义微分算子

%设置用于插入代码的表单格式
\definecolor{mygreen}{rgb}{0,0.6,0}
\definecolor{mygray}{rgb}{0.5,0.5,0.5}
\definecolor{mymauve}{rgb}{0.58,0,0.82}
\lstset{
backgroundcolor=\color{white},   				% choose the background color
language=Matlab,
basicstyle=\footnotesize\ttfamily,        		% size of fonts used for the code
columns=fullflexible,
breaklines=true,                 				% automatic line breaking only at whitespace
captionpos=b,                    				% sets the caption-position to bottom
tabsize=4,
commentstyle=\color{mygreen},    				% comment style
escapeinside={\%*}{*)},          				% if you want to add LaTeX within your code
keywordstyle=\color{blue},       				% keyword style
stringstyle=\color{mymauve}\ttfamily,     		% string literal style
frame=shadowbox,
rulesepcolor=\color{red!20!green!20!blue!20},	% identifierstyle=\color{red},
extendedchars=false,  %解决代码跨页时,章节标题,页眉等汉字不显示的问题
escapebegin=\begin{CJK*}{GBK}{hei},escapeend=\end{CJK*}      % 代码中出现中文必须加上,否则报错
numbers=left,
numberstyle=\tiny,
escapeinside=' ',
xleftmargin=2em,
xrightmargin=2em,
aboveskip=1em
%showstringspaces=flase,%不显示代码字符串中间的空格标记
}

\title{组合数学第四次作业}

\begin{document}

%封面页
\maketitle
\thispagestyle{empty}		%不编页码

%目录页
%\newpage
\tableofcontents
\thispagestyle{empty}		%不编页码

%正文
\newpage
\setcounter{page}{1}

我们选择\textbf{基本}的题目进行作答。
\section{Problem 4.1}
\subsection{Solution}
\subsubsection{(a)}
Proof:
\begin{equation}
    G_n-3G_{n-1}+G_{n-2} = F_{2n} - 3F_{2n-2} + F_{2n-4} = F_{2n-1} - 2F_{2n-2} +F_{2n-4} = -F_{2n-2} + F_{2n-3} + F_{2n-4} = 0
\end{equation}
\subsubsection{(b)}
我们有:
\begin{equation}
    f(x) = G_0 + G_1x + \dots + G_nx^n + \dots
\end{equation}
\begin{equation}
    -3xf(x) = -3G_0(x) -3G_1(x) - \dots - 3G_{n-1}x^n - \dots
\end{equation}
\begin{equation}
    x^2f(x) = G_0x^2 + \dots + G_{n-2}x^n + \dots
\end{equation}
以上三式相加我们可以得到:
\begin{equation}
    (x^2-3x+1)f(x)=G_0+G_1x-3G_0x = x
\end{equation}
从而:
\begin{equation}
    f(x) = \frac{x}{x^2-3x+1}
\end{equation}

\section{Problem 4.2}
\subsection{Solution}
我们有$f(x)=\frac{1}{1-x+x^2}$,即$f(x)-xf(x)+x^2f(x)=1$。考察$x^n$的对应项我们可以得到:
\begin{equation}
    a_n-a_{n-1}+a_{n-2} = 0
\end{equation}
考察初始值,即首先考察常数项即有$a_0=1$,考察一次项我们有$a_1-a_0=0$,从而$a_1=1$。

\section{Problem 4.3}
\subsection{Solution}
注意到该递推关系的两个特征根为3和-1,所以对应的特征方程为$x^2-2x-3=0$,从而对应的递归关系为:
\begin{equation}
    a_n - 2a_{n-1} -3a_{n-2} = 0
\end{equation}

\section{Problem 4.4}
\subsection{Solution}
很明显的该递推式对应的特征方程有三个特征根且均为1,因此对应的通项公式的形式为$a_n = Ax^2 + Bx + C$,首先我们有$a_0=1, a_1=2, a_2 = 2a_1 - a_0 +5 =8$,从而我们可以列出方程求解得到$A= \frac{5}{2},B=-\frac{3}{2}, C=1$,从而该问题的解为:
\begin{equation}
    a_n = \frac{5}{2}n^2 -\frac{3}{2}n+1
\end{equation}


\section{Problem 4.5}
\subsection{Solution}
假设$n$位字符串中不出现AB的种类为$g(n)$,出现至少一次的次数为$f(n)$,显然有:
\begin{equation}
    f(n) + g(n) = 4^n
\end{equation}
分类讨论,若前$n-1$位的字符串中出现子串AB,那么可能的种类为$f(n-1)$,第$n$位随机选取,共$4$种,因此总计$4f(n-1)$种可能的序列;若前$n-1$位的字符串种未出现子串AB,那么只可能出现在第$n-1$位和第$n$位,那么前$n-2$位种未出现子串AB,从而可能的情况数为$g(n-2)$。因此可以得到$f(n)=4f(n-1)+g(n-2)=4f(n-1)+4^{n-2}-f(n-2)$,从而特征方程为:
\begin{equation}
    x^2-4x+1=0
\end{equation}
其两个特征根为$x_1=2+\sqrt{3},x_2=2-\sqrt{3}$,$x_3=4$也是另一个特征根,从而可以写出解的形式:
\begin{equation}
    f(n)=c_1(2+\sqrt{3})^n+c_2(2-\sqrt{3})^n+c_34^n
\end{equation}
考察特解$f(0)=0,f(1)=0,f(2)=1$,可以列出线性方程:
\begin{equation}
    \left[
        \begin{array}{ccc}
            1          & 1           & 1  \\
            2+\sqrt{3} & 2-\sqrt{3}  & 4  \\
            7+\sqrt{3} & 7-4\sqrt{3} & 16
        \end{array}
        \right]
    \left[
        \begin{array}{c}
            c_{1} \\
            c_{2} \\
            c_{3}
        \end{array}
        \right]
    =
    \left[
        \begin{array}{c}
            0 \\
            0 \\
            1
        \end{array}
        \right]
\end{equation}
从而可以解得:
\begin{equation}
    \left[
        \begin{array}{c}
            c_1 \\
            c_2 \\
            c_3
        \end{array}
        \right]
    =
    \left[
        \begin{array}{c}
            -\frac{3+2\sqrt{3}}{6} \\
            -\frac{3-2\sqrt{3}}{6} \\
            1
        \end{array}
        \right]
\end{equation}
从而我们可以得到所有的可能的情况数为:
\begin{equation}
    f(n) = -\frac{3+2\sqrt{3}}{6}(2+\sqrt{3})^n-\frac{3-2\sqrt{3}}{6}(2-\sqrt{3})^n+4^n
\end{equation}

\section{Problem 4.6}
\subsection{Solution}
首先我们需要想到一点,$n$的奇偶性对结论没有区别,因为实际上直接将B和C交换即可,不会影响结论。我们不妨假设$n$为偶数,假设总的可能的情况数量为$f(n)$,那么首先我们需要将$n$号盘移动到B号柱子上,这意味着我们需要先把$1$到$n-1$号盘移动到C号柱子上,由汉诺塔的结论可以知道这样的次数为$2^{n-1}-1$,然后我们就可以将$n$号盘移动到B号柱子上,次数为$1$。接着我们将$n-1$号盘留在C号柱子上,我们现在需要将$n-2$号盘移动到B号柱子上,这意味着我们需要首先将$1$到$n-3$号盘移动到A号柱子上,由汉诺塔的结论,移动的次数为$2^{n-3}-1$,然后将$n-3$号盘移动到B号柱子上,次数为$1$。现在的问题就化归为$n-3$的情况,次数为$f(n-3)$,从而我们得到递推关系:
\begin{equation}
    f(n) = f(n-3)+2^{n-1}+2^{n-3}
\end{equation}
考察初始值,显然由$f(1)=1$,$f(2)=2$,$f(3)=5$,从而我们可以开始通过递推关系求解通项公式。
\par
\begin{itemize}
    \item 当n=3k,k为正整数时:$f(n)=2^{n-1}+2^{n-3}+2^{n-4}+2^{n-6}+\dots+2^5+2^3+f(3)=2^{n-1}+2^{n-2}+\dots+2^3+2^2+2^1+2^0+f(3)-2^2-2^0-(2^{n-1}+2^{n-4}+\dots+2^4+2^1)=\frac{5\times 2^n-5}{7}$
    \item 同理可得n=3k+1,k为正整数的情况:$f(n) = \frac{5\times 2^n-3}{7}$
    \item 同理可的n=3k+2,k为正整数的情况: $f(n)=\frac{5\times 2^n-6}{7}$
\end{itemize}

\section{Problem 4.7}
\subsection{Solution}
假设满足条件的方案数为$f(n)$,若$k=1$,那么$f(1)=f(2)=1,f(n)=0,n\geq 3$,下面考虑$k\geq 2$的情况。假设满足条件且字符串最后两个字母相同的情况数为$g(n)$,满足条件且最后两个字符串不同的条件数为$h(n)$,显然有:
\begin{equation}
    f(n)=g(n)+h(n)
\end{equation}
考察$f(n)$,若字符串倒数第二个和倒数第三个字母不相同,那么最后一个字母就有k种选择,总计为$kh(n-1)$种可能性;若字符串倒数第二个字母和倒数第三个字母相同,那么最后一个字母有$k-1$种选择,总计为$(k-1)g(n-1)$,从而:
\begin{equation}
    f(n) = (k-1)g(n-1)+kh(n-1)=(k-1)f(n-1)+h(n-1)
\end{equation}
再考察$h(n)$,前$n-1$个字母显然是满足没有连续的三个字母,而第$n$个字母有$k-1$种可能性,总计为$(k-1)f(n-1)$种可能性,从而有:
\begin{equation}
    h(n)=(k-1)f(n-1)
\end{equation}
从而我们可以列出递推关系:
\begin{equation}
    f(n)=(k-1)f(n-1)+h(n-1)=(k-1)f(n-1)+(k-1)f(n-2)
\end{equation}
从而可以求出两个特征根是:
\begin{equation}
    x_1=\frac{k-1+\sqrt{k^2+2k-3}}{2}, x_2=\frac{k-1-\sqrt{k^2+2k-3}}{2}
\end{equation}
从而我们写出通解形式:
\begin{equation}
    f(n)=c_1(\frac{k-1+\sqrt{k^2+2k-3}}{2})^n+c_2(\frac{k-1-\sqrt{k^2+2k-3}}{2})^n
\end{equation}
下面确定特解:$f(0)=0, f(1)=k$,从而我们有以下矩阵方程:
\begin{equation}
    \left[
        \begin{array}{cc}
            1   & 1   \\
            x_1 & x_2
        \end{array}
        \right]
    \left[
        \begin{array}{c}
            c_{1} \\
            c_{2}
        \end{array}
        \right]
    =
    \left[
        \begin{array}{c}
            0 \\
            k
        \end{array}
        \right]
\end{equation}
从而可以解得:
\begin{equation}
    \left[
        \begin{array}{c}
            c_1 \\
            c_2
        \end{array}
        \right]
    =
    \left[
        \begin{array}{c}
            \frac{k}{\sqrt{k^2+2k-3}} \\
            -\frac{k}{\sqrt{k^2+2k-3}}
        \end{array}
        \right]
\end{equation}
从而我们可以得到通项公式为:
\begin{equation}
    f(n)= \frac{k}{\sqrt{k^2+2k-3}}((\frac{k-1+\sqrt{k^2+2k-3}}{2})^n - (\frac{k-1-\sqrt{k^2+2k-3}}{2})^n)
\end{equation}

\section{Problem 4.8}
\subsection{Solution}
我们使用逐差法解决该问题。首先设$S_n = \sum_{k=1}^n k^4$,那么我们有$S_n-S_{n-1}=n^4$, 从而$S_{n-1}-S_{n-2}=(n-1)^4$,做差可以得到:
\begin{equation}
    S_{n}-2S_{n-1}+S_{n-2}=4n^3-6n^2+4n
\end{equation}
同理我们继续错位逐差可以继续降次:
\begin{equation}
    S_n-3S_{n-1}+3S_{n-2}-S_{n-3}=12n^2-24n+14
\end{equation}
\begin{equation}
    S_n-4S_{n-1}+6S_{n-2}-4S_{n-3}+S_{n-4}=24n-36
\end{equation}
\begin{equation}
    S_n-5S_{n-1}+10S_{n-2}-10S_{n-3}+5S_{n-4}-S_{n-5}=24
\end{equation}
\begin{equation}
    S_n-6S_{n-1}+15S_{n-2}-20S_{n-3}+15S_{n-4}-6S_{n-5}+S_{n-6}=0
\end{equation}
从而我们可以根据递推式获得其特征方程并求出特征根,我们发现其6个特征根均为1,从而$S_n$具有如下的结构:
\begin{equation}
    S_n= An^5+Bn^4+Cn^3+Dn^2+En+F
\end{equation}
带入起始项可以解得参数,带入即可得到:
\begin{equation}
    S_n=\frac{1}{5}n^5+\frac{1}{2}n^4+\frac{1}{3}n^3-\frac{1}{30}n
\end{equation}

\section{Problem 4.9}
\subsection{Solution}
\subsubsection{(a)}
若第n个未被选中,那么可能的情况数为$f(n-1,k)$;若第n个被选中,那么可能的情况数为$f(n-2,k-1)$,从而递推关系为:
\begin{equation}
    f(n,k) = f(n-1,k) + f(n-2,k-1), n\geq k
\end{equation}
其中初始条件为$f(n,1)=n$
\subsubsection{(b)}
我们用数学归纳法证明:
\begin{equation}
    f(n,k) = \binom{n-k+1}{k}
\end{equation}
对$n+k(\geq 2)$做归纳,首先$n+k=2$时,即$n=k=1$时结论显然成立。假设结论对小于$n+k$的$(n,k)$对成立,那么我们有:
\begin{equation}
    f(n,k) = f(n-1,k) + f(n-2,k-1)=\binom{n-k}{k}+\binom{n-k}{k-1}=\binom{n-k+1}{k}
\end{equation}
根据数学归纳法结论成立!

\subsubsection{(c)}
对于$f(n,k)$考察如下情况:若$(n,1)$这对没有被同时选中,那么就是$g(n,k)$种可能性;若$(n,1)$这对被同时选中了,那么$2$和$n-1$也无法被选中,其可能的情况数为$f(n-4,k-2)$,从而我们可以得到等式$f(n,k)=g(n,k)+f(n-4,k-2)$,整理可得:
\begin{equation}
    g(n,k) = f(n,k) - f(n-4,k-2) = \binom{n-k+1}{k} - \binom{n-k-1}{k-2}
\end{equation}

\section{Problem 4.10}
\subsection{Solution}
\subsubsection{(a)}
假设所有可能的铺砖方案数为$f(n)$,我们思考一下最后一个正方型砖应该怎么铺满。第一种可能是直接使用$1\times 1$的方砖铺满,那么此时所有可能的情况数量为$f(n-1)$;第二种可能是不使用$1\times 1$的方砖,那么我们就需要使用等腰直角三角形的方砖,首先显而易见我们必须使用结构直角边长为1的等腰直角三角形砖去铺右侧,这里有两种铺法且是旋转等价的,我们任取一种即可,那么剩下的一半铺满的方法有两种,其中一种是使用直角边长为1的等腰直角三角形铺满,有$f(n-1)$种,另一种是使用斜边长为 2 的等腰直角三角形砖,那么我们需要考虑第$n-1$块砖,同理的可以进行分析结论为$f(n-2)$,并继续下去直到第一块砖,并考虑两种铺法的等价性,因此第二种情况的种类总计为$2(f(n-1)+f(n-2)+\dots+f(1))$。从而我们现在能够得到关系式:
\begin{equation}
    f(n) = f(n-1) + 2(f(n-1)+f(n-2)+\dots+f(1))
\end{equation}
我们考察$f(n+1)$并重复利用上述等式:
\begin{equation}
    f(n+1) = 3f(n) + 2(f(n-1)+f(n-2)+\dots+f(1))=3f(n)+f(n)-f(n-1)=4f(n)-f(n-1)
\end{equation}
从而其两个特征根为:
\begin{equation}
    x_1 = 2+\sqrt{3},x_2=2-\sqrt{3}
\end{equation}
从而通项公式为:
\begin{equation}
    f(n) = c_1(2+\sqrt{3})^n+c_2(2-\sqrt{3})^n
\end{equation}
考察初始值:$f(1)=3,f(2)=11$,我们有:
\begin{equation}
    \left[
        \begin{array}{cc}
            x_1   & x_2   \\
            x_1^2 & x_2^2
        \end{array}
        \right]
    \left[
        \begin{array}{c}
            c_{1} \\
            c_{2}
        \end{array}
        \right]
    =
    \left[
        \begin{array}{c}
            3 \\
            11
        \end{array}
        \right]
\end{equation}
从而可以解得:
\begin{equation}
    \left[
        \begin{array}{c}
            c_1 \\
            c_2
        \end{array}
        \right]
    =
    \left[
        \begin{array}{c}
            \frac{3+\sqrt{3}}{3} \\
            \frac{3-\sqrt{3}}{3}
        \end{array}
        \right]
\end{equation}
从而我们可以得到通项公式为:
\begin{equation}
    f(n) = \frac{3+\sqrt{3}}{3}(2+\sqrt{3})^n+\frac{3-\sqrt{3}}{3}(2-\sqrt{3})^n
\end{equation}
\subsubsection{(b)}
我们考察最后一块砖,设总的可能情况的总砖数量为$g(n)$,分两种情况讨论:第一种情况是最后一块砖使用$1\times 1$的方砖铺满,那么此时使用砖块的总数量为$g(n-1)+f(n-1)$;第二种情况是最后一块砖不使用$1\times 1$的方砖,那么右半侧的砖头只能使用直角边长为1的等腰直角三角形砖铺,这有两种铺法且是旋转对称的,我们任取一种考察,最后一块砖的另一部分可以使用直角边长为1的等腰直角三角形砖铺,那么所使用的砖块为$g(n-1)+2f(n-1)$,如果使用斜边长为2的等腰直角三角形砖,那么同上面(a)的情况一样我们可以有如下递推关系:
\begin{equation}
    g(n) = g(n-1) + f(n-1) + 2(g(n-1)+2f(n-1)) + 2(g(n-2) + 3f(n-2)) + \dots + 2(g(1)+nf(1))
\end{equation}
将$n$改为$n+1$可得:
\begin{equation}
    g(n+1) = g(n) + f(n) + 2(g(n) + 2f(n)) + 2(g(n-1) + 3f(n-1)) + \dots + 2(g(1)+(n+1)f(1))
\end{equation}
两式做差我们可以得到:
\begin{equation}
    g(n+1) - 4g(n)+g(n-1) = 6f(n)-2f(n-1)
\end{equation}
注意到$g(n+1)-4g(n)+g(n-1)$的特征根也为$x_1,x_2$,而$f(n)$也为$Ax_1^n+Bx_2^n$的形式,因此我们可以知道该递推式的特征根为2重$x_1$和2重$x_2$。从而$g(n)$有如下形式:
\begin{equation}
    g(n) = (An+B)(2+\sqrt{3})^n + (Cn+D)(2-\sqrt{3})^n
\end{equation}
然后我们可以根据初值求解,其中$g(0)=0,g(1)=5,g(2)=36,g(3)=199$,我们可以得到最终通解为:
\begin{equation}
    f(n) = (\frac{2+\sqrt{3}}{3}n+\frac{\sqrt{3}}{18})(2+\sqrt{3})^n + (\frac{2-\sqrt{3}}{3}n-\frac{\sqrt{3}}{18})(2-\sqrt{3})^n
\end{equation}
\end{document}

