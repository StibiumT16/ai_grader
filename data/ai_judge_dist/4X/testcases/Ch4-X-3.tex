%
% This is a borrowed LaTeX template file for lecture notes for CS267,
% Applications of Parallel Computing, UCBerkeley EECS Department.
% Now being used for CMU's 10725 Fall 2012 Optimization course
% taught by Geoff Gordon and Ryan Tibshirani.  When preparing
% LaTeX notes for this class, please use this template.
%
% To familiarize yourself with this template, the body contains
% some examples of its use.  Look them over.  Then you can
% run LaTeX on this file.  After you have LaTeXed this file then
% you can look over the result either by printing it out with
% dvips or using xdvi. "pdflatex template.tex" should also work.
%

\documentclass[UTF8,oneside]{article}

\usepackage[UTF8,scheme=plain]{ctex}
% \usepackage[AutoFakeBold,AutoFakeSlant]{xeCJK}  % 载入 xeCJK以支持中文,支持伪粗体,伪斜体
\usepackage[margin=1in]{geometry}
\usepackage{amsmath,amsthm,amssymb}
\usepackage{graphicx}
\usepackage{mathtools}
\usepackage{float, bm}

\setlength{\oddsidemargin}{0.25 in}
\setlength{\evensidemargin}{-0.25 in}
\setlength{\topmargin}{-0.6 in}
\setlength{\textwidth}{6.5 in}
\setlength{\textheight}{8.5 in}
\setlength{\headsep}{0.75 in}
\setlength{\parindent}{0 in}
\setlength{\parskip}{0.1 in}

%
% ADD PACKAGES here:
%

\usepackage{amsmath,amsfonts,graphicx}
\usepackage{etoolbox}
\AtBeginEnvironment{proof}{\normalsize}

%
% The following commands set up the lecnum (lecture number)
% counter and make various numbering schemes work relative
% to the lecture number.
%

%
% The following macro is used to generate the header.
%
\newcommand{\lecture}[4]{
   \pagestyle{myheadings}
   \thispagestyle{plain}
   \newpage
   \setcounter{page}{1}
   \noindent
   \begin{center}
   \framebox{
      \vbox{\vspace{2mm}
    \hbox to 6.28in { {\bf Combination Math
	\hfill 2024 Fall} }
       \vspace{4mm}
       \hbox to 6.28in { {\Large \hfill HW #1  \hfill} }
       \vspace{2mm}
       \hbox to 6.28in { {\it Student: #2 \hfill Time: #3} }
      \vspace{2mm}}
   }
   \end{center}
   \markboth{Lecture #1: #2}{Lecture #1: #2}

}
%
% Convention for citations is authors' initials followed by the year.
% For example, to cite a paper by Leighton and Maggs you would type
% \cite{LM89}, and to cite a paper by Strassen you would type \cite{S69}.
% (To avoid bibliography problems, for now we redefine the \cite command.)
% Also commands that create a suitable format for the reference list.
\renewcommand{\cite}[1]{[#1]}
\def\beginrefs{\begin{list}%
        {[\arabic{equation}]}{\usecounter{equation}
         \setlength{\leftmargin}{2.0truecm}\setlength{\labelsep}{0.4truecm}%
         \setlength{\labelwidth}{1.6truecm}}}
\def\endrefs{\end{list}}
\def\bibentry#1{\item[\hbox{[#1]}]}


\begin{document}
%FILL IN THE RIGHT INFO.
%\lecture{**LECTURE-NUMBER**}{**DATE**}{**LECTURER**}{**SCRIBE**}
\lecture{4}{}{Oct. 31}
%\footnotetext{These notes are partially based on those of Nigel Mansell.}

% **** YOUR NOTES GO HERE:

% Some general latex examples and examples making use of the
% macros follow.
%**** IN GENERAL, BE BRIEF. LONG SCRIBE NOTES, NO MATTER HOW WELL WRITTEN,
%**** ARE NEVER READ BY ANYBODY.

\fontsize{12pt}{24pt}

\section{4.1}

\subsection{(1)}

\begin{align*}
    G_n - 3G_{n-1}+G_{n-2} &= F_{2n} - 3F_{2n-2} + F_{2n-4}\\
    &= F_{2n-1} + F_{2n-2} - 3F_{2n-2} + F_{2n-4} \\
    &= 2F_{2n-2} + F_{2n-3} - 3F_{2n-2} + F_{2n-4} \\
    &= F_{2n-3} + F_{2n-4} - F_{2n-2} \\
    &= 0
\end{align*}

\subsection{(2)}

设母函数为$G(x)$, 根据递推关系有:
\[
    G(x) -3xG(x) + x^2G(x) = G_0 + (G_1 - 3G_0)x
\]
其中初始值$G_0 = F_0 = 0, G_1 = F_2 = 1$, 于是:
\[
    G(x) = \frac{x}{1-3x + x^2}
\]


\section{4.2}

由题意可得递推式为$a_n - a_{n-1} + a_{n-2} = 0$, 利用长除法可得$a_0=1, a_1=1$.

\section{4.3}

由于可知特征根为$3, -1$, 根据初始值$a_0, a_1$, 构造母函数$G(x) = \frac{1}{1 - 2x - 3x^2}$, 对应的递推关系为$a_n - 2a_{n-1} - 3a_{n-2} = 0$, 可以验证对任意$c,d$, 该递推关系成立.

\section{4.4}

注意到递推关系为$a_n - 2a_{n-1} + a_{n-2} = 5\cdot 1^n$, 而通解$a_n - 2a_{n-1} + a_{n-2} = 0$为$1$并且是二重根, 于是可设$a_n = An^2 + Bn+c$, 代入初始值$a_0 =1, a_1=2, a_2 = 8$, 解得:
\[
    a_n = \frac{5}{2}n^2 -\frac{3}{2}n + 1
\]

\section{4.5}

假设总的数量为$a_n$, 其中以字母$A$开头数量为$b_n$, 可得递推关系:
\begin{align*}
    a_n = 3a_{n-1} + b_n \\
    b_n = b_{n-1} + 4^{n-2} + 2a_{n-2}
\end{align*}
可得$b_n = a_n - 3a_{n-1}$, 代入可得:
\[
    a_n - 3a_{n-1} = a_{n-1} - 3a_{n-2} + 4^{n-2} + 2a_{n-2}
\]
化简得:
\[
    a_n - 4a_{n-1} +a_{n-2} = 4^{n-2}
\]
其中初始值$a_1=0,a_2=1$. 可得特征方程为$x^2 - 4x + 1=0$, 根为$2\pm \sqrt{3}$, 可设通解为$a_n = A(2+\sqrt{3})^n + B(2-\sqrt{3})^n$. 对于其特解, 可设$a_n = c4^n$, 于是代入初始值后求解得:
\[
    a_n = - \frac{2\sqrt{3}+3}{6}(2+\sqrt{3})^n + \frac{2\sqrt{3}-3}{6}(2-\sqrt{3})^n + 4^n
\]

\section{4.6}

设原始汉诺塔问题的步骤数为$b_n$, 可得递推关系$b_n = 2b_{n-1} + 1$, 初始值$b_1=1$. 可解得$b_n = 2^n - 1$. 假设所求变种问题步骤数为$a_n$, 不失一般性, 考虑$n$为偶数: 可以首先用$b_{n-1}$步将$n-1$个圆盘移动到$C$柱, 在用一次将第$n$个圆盘移动到$B$柱. 接着用$b_{n-3}$次将$C$柱上前$n-3$个移动到$A$柱子, 用1次把第$n-2$个移到$B$柱子上, 最后问题转化为剩下$n-3$个圆盘所需的方案数, 于是:
\[
    a_n = b_{n-1} + 1 + b_{n-3} + 1 + a_{n-3} = a_{n-3} + 5\cdot 2^{n-3}
\]
根据初始值$a_1 = 1, a_2 = 2, a_3 = 5$, 可得:
\[
    a_{n} + a_{n-1} + a_{n-2} = 5\cdot 2^{n-2} - 2
\]
对于其通解$a_{n} + a_{n-1} + a_{n-2}=0$, 由于有一对共轭复根$\frac{-1\pm \sqrt{3}i}{2}$, 可设$a_n = A\cos \frac{2\pi}{3}n + B\sin \frac{2\pi}{3}n$. 关于其特解, 可设$a_n = k\cdot 2^n + b$, 于是解的形式为:
\[
    a_n = A\cos \frac{2\pi}{3}n + B\sin \frac{2\pi}{3}n + k\cdot 2^n + b
\]
待定常数后可得:
\[
    a_n = -\frac{1}{21}\cos \frac{2\pi}{3}n + \frac{\sqrt{3}}{7}\sin \frac{2\pi}{3}n + \frac{5}{7} \cdot 2^n - \frac{2}{3}
\]

\section{4.7}

当$k=1$时, 当$n$任意大时, 无解。当$k>1$时, 假设长度为$n$时有$a_n$种方案, 若倒数第二个字母与最后一个字母不同, 则有$(k-1)a_{n-1}$种方案; 若相同, 则倒数第三个字母一定与这两个字母不同, 有$(k-1)a_{n-2}$种方案, 故递推关系为:
\[
    a_n = (k-1)a_{n-1} + (k-1)a_{n-2}
\]
初始值$a_1 = k, a_2 = k^2$. 可得特征方程为$x^2 - (k-1)x - (k-1) = 0$, 有两个实根$\frac{k-1 \pm \sqrt{(k-1)(k+3)}}{2}$. 设:
\[
    a_n = A\cdot \left ( \frac{k-1 + \sqrt{(k-1)(k+3)}}{2} \right)^n + B\cdot \left ( \frac{k-1 - \sqrt{(k-1)(k+3)}}{2} \right)^n
\]
代入初始值, 解得:
\[
    A = \frac{k}{2}\left (\frac{1}{k-1} + \frac{1}{\sqrt{(k-1)(k+3)}} \right ),\quad  B = \frac{k}{2}\left ( \frac{1}{k-1} - \frac{1}{\sqrt{(k-1)(k+3)}} \right )
\]

\section{4.8}

假设$f$是一个光滑的函数, 且满足$f(t, n) = \sum_{k=1}^n k^t$, 于是$f(t, n) = f(t, n-1) + n^t$, 进而:
\begin{align*}
    f'(t, n) &= f'(t, n-1) + t\cdot n^{t-1} \\
    &= f'(t, n-2) + t\cdot (n-1)^{t-1} + t\cdot n^{t-1} \\
    &= \cdots \\
    &= f'(t, 0) + t\cdot \sum_{k=1}^n k^{t-1} \\
    &= tf(t-1, n) + f'(t, 0)
\end{align*}
于是:
\[
    f(t, n) = \int_{0}^n tf(t-1, x) + f'(t, 0) dx
\]
于是在求出$f(t-1, n)$后, 仅需求解不定积分结果后乘上$t$在待定常数项$f'(t, 0)$使得$f(t, 1)=1$即可. 已知$f(1,n)= \frac{n(n+1)}{2} = \frac{1}{2}n^2 + \frac{1}{2}n$, 可以依次算得:
\begin{align*}
    f(2, n) &= \frac{1}{3}n^3 + \frac{1}{2}n^2 + \frac{1}{6}n \\
    f(3, n) &= \frac{1}{4}n^3 + \frac{1}{2}n^3 + \frac{1}{4}n^2\\
    f(4, n) &= \frac{1}{5}n^5 + \frac{1}{2}n^4 + \frac{1}{3}n^3 - \frac{1}{30}n^2
\end{align*}

\section{4.9}

\subsection{(1)}

假设这$k$个不同的数中的第一个是$i$, 则剩下$k-1$个数需要在$[i+2, n]$这$n-i-1$个数中选择. 又因为所选结果必须两两不相邻, 所以$i\in[1, n-2k+2]$, 于是可得递推关系:
\[
    f(n, k) - \sum_{i=1}^{n-2k+2} f(n-i-1, k-1) = 0
\]

\subsection{(2)}

猜想$f(n, k)=\binom{n-k+1}{k}$, 当$n=k=1$时符合, 假设$f(j, k)=\binom{j-k+1}{k}$, 则:
\begin{align*}
    f(j+1, k) &= \sum_{i=1}^{j-2k+3} f(j-i, k-1)\\
    &= \sum_{i=0}^{j-2k+2} f(j-i-1, k-1)\\
    &= \left (\sum_{i=1}^{j-2k+2} f(j-i-1, k-1) \right ) + f(j-1, k-1) \\
    &= f(j, k) + f(j-1, k-1) \\
    &= \binom{j-k+1}{k} + \binom{j-k+1}{k-1}\\
    &= \binom{j-k+2}{k}
\end{align*}
得证.

\subsection{(3)}

相当于我们要去掉同时选了$1,n$的方案, 剩下的$k-2$个数在$[3, n-2]$中选, 这样的组合有$f(n-2, k-2)$个, 所以$g(n,k) = f(n, k) - f(n-4, k-2)$.

\section{4.10}

\subsection{(1)}

设$a_n$为有$n$个正方形组成的路径的方案数, $b_n$为由$n-1$个正方形与一个直角边长为1的等腰直角三角形的方案数. 考虑直角边长为1的等腰直角三角形铺设的方向性, 可知$a_1=3, b_1=1$. 

对于$a_n$,若最后一块砖是方砖, 则方案数为$a_{n-1}$; 若最后一块砖为直角边长为1的等腰直角三角形, 考虑两种摆放方式, 方案数为$2b_n$. 最后一块砖不可能是斜边长为2的砖头. 对于$b_n$, 若最后一块砖为斜边长为2的等腰直角三角形, 则方案数为$b_{n-1}$. 最后一块不可能为方砖.

于是可得递推关系:
\begin{align*}
    a_n &= a_{n-1} + 2b_n \\
    b_n &= a_{n-1} + b_{n-1}
\end{align*}
消去$b_n$, 得$a_n - 4a_{n-1} + a_{n-2} = 0$. 初始值$a_0 = 1, a_1=3, a_2 = 11$. 根据特征方程$x^2-4x+1=0$可知根为$2\pm\sqrt{3}$, 于是设解的形式为$A(2+\sqrt{3})^n + B(2-\sqrt{3})^n$, 代入初始值求得:
\[
    a_n = \frac{3+\sqrt{3}}{6}(2+\sqrt{3})^n + \frac{3-\sqrt{3}}{6}(2-\sqrt{3})^n
\]

\subsection{(2)}

设$c_n$为总的砖数, $d_n$为由$n-1$个正方形与一个小等腰直角三角形组成的方案的总砖数. 

对于$c_n$:
\begin{itemize}
    \item 最后一块砖为方砖, 方案数有$a_{n-1}$种, 前面合计用了$c_{n-1}$块砖
    \item 最后一块为小等腰直角三角形, 方案数为$2b_n$, 前面合计用了$2d_n$块砖
    \item 最后一块砖不可能是大等腰直角三角形砖
\end{itemize}
于是:
\[
    c_n = c_{n-1} + a_{n-1} + 2d_n + 2b_n
\]

对于$d_n$:
\begin{itemize}
    \item 最后一块为小等腰直角三角形砖, 方案数为$a_{n-1}$, 前面合计用了$c_{n-1}$块砖
    \item 最后一块为大等腰直角三星砖, 方案数为$b_{n-1}$, 前面合计用了$d_{n-1}$块砖
    \item 最后一块不可能是方砖
\end{itemize}
于是:
\[
    d_n = c_{n-1} + a_{n-1} + d_{n-1} + b_{n-1}
\]

整理以上递推式, 消去$d_n$,可得:
\[
    c_n - 4c_{n-1} + c_{n-2} = 2(a_n - a_{n-1})
\]
根据特征方程$x^2-4x+1=0$, 其根为$2\pm \sqrt{3}$. 又根据$a_n$的形式, 可设$c_n$解的形式为$(An + B)(2+\sqrt{3})^n + (Cn + D)(2-\sqrt{3})^n$, 待定常数项后得:

\[
    A = \frac{2+\sqrt{3}}{3}, B = \frac{\sqrt{3}}{18}, C = \frac{2-\sqrt{3}}{3}, D = -\frac{\sqrt{3}}{18}
\]

\end{document}





