\documentclass{article}
\usepackage{ctex}
\usepackage{graphicx} % Required for inserting images
\usepackage[a4paper]{geometry}
\usepackage{amsmath}
\usepackage{amssymb}
\DeclareMathOperator*{\lt}{\ensuremath <}
\DeclareMathOperator*{\gt}{\ensuremath >}

\title{组合数学\quad 第四章作业(基础)}
\author{}

\begin{document}

\maketitle

\section*{4.1.}
\paragraph{(1)}
\begin{align}
    G_n&=F_{2n}=F_{2n-1}+F_{2n-2}=2F_{2n-2}+F_{2n-3}=3F_{2n-3}+2F_{2n-4}, \\
    G_{n-1}&=F_{2n-2}=F_{2n-3}+F_{2n-4}, \\
    G_{n-2}&=F_{2n-4}.
\end{align}
故 $G_n-3G_{n-1}+G_{n-2}=0$。

\paragraph{(2)}
% 对应的特征方程为 $x^2-3x+1=0$,解得 $x=\frac{3\pm\sqrt{5}}{2}$,即
% \begin{equation}
%     G_n=c_1\left(\frac{3+\sqrt 5}{2}\right)^n+c_2\left(\frac{3-\sqrt 5}{2}\right)^n.
% \end{equation}
% 另一方面, $G_0=F_0=0$,$G_1=F_2=1$,故
% \begin{align}
%     c_1+c_2&=0, \\
%     c_1\left(\frac{3+\sqrt 5}{2}\right)+c_2\left(\frac{3-\sqrt 5}{2}\right)&=2.
% \end{align}
% 解得 $c_1=\frac{2}{5}\sqrt 5,c_2=-\frac {2}{5}\sqrt 5$。故通项公式为
记 $H(x)=G_0+G_1x+G_2x^2+\cdots$,则
\begin{align}
    H(x)-3xH(x)+x^2H(x)&=G_0+G_1x-3xG_0x+\sum_{k\ge 2}(G_k-3G_{k-1}+G_{k-2})x^2 \\
    &=G_0+G_1x-3xG_0x \\
    &=x.
\end{align}
故 $\{G_n\}$ 的母函数为 $\frac{x}{1-3x+x^2}$。

\section*{4.2.}
记 $G(x)=\frac{1}{1-x+x^2}$,则
\begin{align}
    1&=(1-x+x^2)G(x) \\
    &=a_0 + a_1x-a_0x + \sum_{k\ge 2}(a_k-a_{k-1}+a_{k-2})x^k.
\end{align}
对比系数得
\begin{align}
    a_0 &= 1, \\
    a_1-a_0 &= 0, \\
    a_k-a_{k-1}+a_{k-2} &= 0\quad (k\ge 2).
\end{align}
故 $a_0=1,a_1=1$,递推式为 $a_k-a_{k-1}+a_{k-2}=0$.

\section*{4.3.}
从通项形式可以猜测,递推关系形如 $a_n+ua_{n-1}+va_{n-2}=0$。于是
\begin{align}
    0&=a_n+ua_{n-1}+va_{n-2} \\
    &=3^{n-2}c(9+3u+v)+(-1)^{n-2}d(1-u+v).
\end{align}
比对系数可得
\begin{align}
    9+3u+v&=0, \\
    1-u+v&=0.
\end{align}
解得 $u=-2,v=-3$。故所求递推关系为 $a_n-2a_{n-1}-3a_{n-2}=0$.

\section*{4.4.}
首先求 $a_n-2a_{n-1}+a_{n-2}=0$ 的通解。对应的特征方程为 $x^2-2x+1=0$,有二重根 1。故通解为 $a_n=cn+d$。

原递推式的特解形如 $a_n=cn^2$,带入得
\begin{align}
    5&=a_n-2a_{n-1}+a_{n-2} \\
    &=c(n^2-2n^2+4n-2+n^2-4n+4) \\
    &=2c.
\end{align}
故 $c=\frac 52$。因此,原递推式的通项形如 $a_n=\frac 52n^2+cn+d$。带入初值,
\begin{align}
    1&=d, \\
    2&=\frac 52+c+d.
\end{align}
故 $c=-\frac 32,d=1$,即
\begin{equation}
    a_n=\frac 52n^2-\frac 32n+1.
\end{equation}

\section*{4.5.}
记满足条件的字符串数量为 $a_n$,其中以 B 结尾的数量为 $b_n$。考虑满足条件字符串的后两位:
\begin{itemize}
    \item 若为 AB,则前 $n-2$ 位可任取,共 $4^{n-2}$ 种方案;
    \item 若不以 B 结尾,则 AB 必须在前 $n-1$ 位中出现,共 $3a_{n-1}$ 种方案;
    \item 若为 CB 或 DB,则 AB 必须在前 $n-2$ 位中出现,共 $2a_{n-2}$ 种方案;
    \item 若为 BB,则转化为以 B 结尾的 $n-1$ 位串,共 $b_{n-1}$ 种方案。
\end{itemize}
于是可以得到递推式
\begin{align}
    a_n&=4^{n-2}+3a_{n-1}+2a_{n-2}+b_{n-1}, \\
    b_n&=4^{n-2}+2a_{n-2}+b_{n-1},
\end{align}
故 $b_n=a_n-3a_{n-1}$,带入得
\begin{equation}
    a_n=4^{n-2}+3a_{n-1}+2a_{n-2}+(a_{n-1}-3a_{n-2})=4^{n-2}+4a_{n-1}-a_{n-2}.
\end{equation}
对应特征方程为 $x^2-4x+1=0$,解得 $x=\frac{4\pm\sqrt{16-4}}{2}=2\pm\sqrt 3$。

首先求递推关系的特解,形如 $a_n=c\cdot 4^n$,代入得
\begin{equation}
    4^{n-2}=c(4^n-4\cdot 4^{n-1}+4^{n-2})=c\cdot 4^{n-2},
\end{equation}
故 $c=1$,特解为 $a_n=4^n$,因此原递推关系的解形如
\begin{equation}
    a_n=4^n+c_1(2+\sqrt 3)^n+c_2(2-\sqrt 3)^n.
\end{equation}
带入初值 $a_0=0,a_1=0$,可得
\begin{align}
    0&=1+c_1+c_2, \\
    0&=4 + c_1(2+\sqrt 3)+ c_2(2-\sqrt 3).
\end{align}
解得
\begin{equation}
    c_1=-\frac{3+2\sqrt 3}{6},\quad c_2=\frac{-3+2\sqrt 3}{6},
\end{equation}
故满足条件的字符串个数为
\begin{equation}
    a_n=4^n-\frac{3+2\sqrt 3}{6}(2+\sqrt 3)^n+\frac{-3+2\sqrt 3}{6}(2-\sqrt 3)^n.
\end{equation}

\section*{4.6.}
记所需次数为 $a_n$,原汉诺塔问题所需次数为 $b_n$。

当 $n$ 为偶数时,$n$ 号圆盘需要移动到 B 柱上,此前需将 $1\sim (n-1)$ 均移动至 C 柱上,共需 $b_{n-1}+1$ 步。此时 $n$, $n-1$ 号圆盘位置均已正确。要将 $n-2$ 号圆盘移动到 B 柱,首先需要将 $1\sim (n-3)$ 均移动至 A 柱,共需 $b_{n-3}+1$ 步。此时 $n$, $n-1$, $n-2$ 位置均已正确,且 $1\sim (n-3)$ 位于 A 柱,故还需 $a_{n-3}$ 步。于是可以得到递推式
\begin{equation}
    a_n=(b_{n-1}+1)+(b_{n-3}+1)+a_{n-3}.
\end{equation}
当 $n$ 为奇数时,同理也可得到同样的递推式。又由于 $b_n=2^n-1$,可得
\begin{equation}
    a_n=2^{n-1}+2^{n-3}+a_{n-3}.
\end{equation}
对应特征方程为 $x^3-1=0$,解为 $x=1,e^{\pm\frac 23\pi}$。首先求特解,特解形如 $a_n=c\cdot 2^n$,带入得
\begin{equation}
    2^{n-1}+2^{n-3}=c\cdot(2^n-2^{n-3})=7c\cdot 2^{n-3}.
\end{equation}
故 $c=\frac 57$。故解形如
\begin{equation}
    a_n=\frac 57\cdot 2^n+c_1+c_2\cos(\frac 23n\pi)+c_3\sin(\frac 23n\pi).
\end{equation}
带入初值有
\begin{align}
    0&=\frac 57+c_1+c_2, \\
    1&=\frac{10}7+c_1-\frac{1}{2}c_2+\frac{\sqrt 3}2c_3, \\
    2&=\frac{20}7+c_1-\frac 12c_2-\frac{\sqrt 3}2c_3.
\end{align}
解得$c_1=-\frac{2}{3},c_2=-\frac{1}{21},c_3=\frac{\sqrt 3}7$。故最小移动次数为
\begin{equation}
    a_n=\frac 57\cdot 2^n-\frac 23-\frac 1{21}\cos(\frac 23n\pi)+\frac{\sqrt 3}7\sin(\frac 23n\pi).
\end{equation}

\section*{4.7.}
记方案数为 $a_n$。考虑最后出现的字母:
\begin{itemize}
    \item 若出现一次,相当于长为 $n-1$ 的串,最后添加一个不同的字母,共 $(k-1)a_{n-1}$ 种可能;
    \item 若出现两次,同理,共 $(k-1)a_{n-2}$ 种可能。
\end{itemize}
故递推关系为 $a_n=(k-1)(a_{n-1}+a_{n-2})$。对应的特征方程为 $x^2-(k-1)x-(k-1)=0$,解得 $x=\frac{k-1\pm\sqrt{k^2+2k-3}}{2}$,故通项公式形如
\begin{equation}
    a_n=c_1\left(\frac{k-1+\sqrt{k^2+2k-3}}{2}\right)^n+c_2\left(\frac{k-1-\sqrt{k^2+2k-3}}{2}\right)^n.
\end{equation}
初值为 $a_1=k,a_2=k^2$,倒推得
\begin{equation}
    a_0=\frac{a_2-(k-1)a_1}{k-1}=\frac{k^2-k(k-1)}{k-1}=\frac{k}{k-1}.
\end{equation}
带入初值得
\begin{align}
    \frac k{k-1}&=c_1+c_2, \\
    k&=c_1\left(\frac{k-1+\sqrt{k^2+2k-3}}{2}\right)+c_2\left(\frac{k-1-\sqrt{k^2+2k-3}}{2}\right).
\end{align}
解得
\begin{align}
    c_1&=\frac{k}{2\sqrt{(k-1)(k+3)}}+\frac{k}{2(k-1)}, \\
    c_2&=-\frac{k}{2\sqrt{(k-1)(k+3)}}+\frac{k}{2(k-1)}.
\end{align}
故总方案数为
\begin{align}
    a_n&=\left(\frac{k}{2\sqrt{(k-1)(k+3)}}+\frac{k}{2(k-1)}\right)\left(\frac{k-1+\sqrt{k^2+2k-3}}{2}\right)^n \\
    &\quad +\left(-\frac{k}{2\sqrt{(k-1)(k+3)}}+\frac{k}{2(k-1)}\right)\left(\frac{k-1-\sqrt{k^2+2k-3}}{2}\right)^n \\
    &=\frac{k}{(k-1)\sqrt{(k-1)(k+3)}}\left(\frac{k-1+\sqrt{k^2+2k-3}}{2}\right)^{n+1} \\
    &\quad-\frac{k}{(k-1)\sqrt{(k-1)(k+3)}}\left(\frac{k-1-\sqrt{k^2+2k-3}}{2}\right)^{n+1}.
\end{align}

\section*{4.8.}
记 $n^{\underline{m}}=\prod_{k=0}^{m-1}(n-k)$,则有
\begin{equation}
    (n+1)^{\underline{m+1}}-n^{\underline{m+1}}=n^{\underline{m}}((n+1)-(n-m))=(m+1)n^{\underline{m}},
\end{equation}
即
\begin{equation}
    \sum_{k=1}^nk^{\underline{m}}=\frac 1{m+1}\sum_{k=1}^n\left((k+1)^{\underline{m+1}}-k^{\underline{m+1}}\right)=\frac{(n+1)^{\underline{{m+1}}}}{m+1}.
\end{equation}

设
\begin{align}
    n^4&=an^{\underline{4}}+bn^{\underline{3}}+cn^{\underline{2}}+dn^{\underline{1}}+e \\
    &=an(n-1)(n-2)(n-3)+bn(n-1)(n-2)+cn(n-1)+dn+e \\
    &=a(n^4-6n^3+11n^2-6n)+b(n^3-3n^2+2n)+c(n^2-n)+dn+e \\
    &=an^4+(-6a+b)n^3+(11a-3b+c)n^2+(-6a+2b-c+d)n+e.
\end{align}
比较系数可得
\begin{align}
    1&=a, \\
    0=-6a+b, \\
    0=11a-3b+c, \\
    0&=-6a+2b-c+d, \\
    0&=e.
\end{align}
解得 $a=1,b=6,c=7,d=1,e=0$。于是
\begin{align}
    \sum_{k=1}^nk^4&=\sum_{k=1}^n\left(k^{\underline{4}}+6k^{\underline{3}}+7k^{\underline{2}}+1k^{\underline{1}} \right) \\
    &=\frac 15(n+1)^{\underline{5}}+\frac 32(n+1)^{\underline{4}}+\frac 73(n+1)^{\underline{3}}+\frac 12(n+1)^{\underline{2}} \\
    &=\frac{n(n+1)}{30}\left(6(n-1)(n-2)(n-3)+45(n-1)(n-2)+70(n-1)+15\right) \\
    &=\frac{n(n+1)}{30}(6n^3-36n^2+66n-36+45n^2-135n+90+70n-70+15) \\
    &=\frac{n(n+1)}{30}(6n^3+9n^2+n-1) \\
    &=\frac{n(n+1)(2n+1)(3n^2+3n-1)}{30}.
\end{align}

\section*{4.9.}
\paragraph{(1)} 考虑是否选择 $n$,若选择,剩下的数需从 $1\sim(n-2)$ 种选择,共 $f(n-2,k-1)$ 种方案。若不选择,共 $f(n-1,k)$ 种方案。故 $f(n,k)=f(n-2,k-1)+f(n-1,k)$。

\paragraph{(2)}
考虑初值,$f(n,0)=1$,对 $k\ge 1 $ 有 $f(0,k)=0$,$f(1,1)=1$,对 $k\ge 2$ 有 $f(1,k)=0$。我们将使用数学归纳法证明 $f(n,k)=\binom{n-k+1}{k}$($n\ge k$)。

$n=1$ 时,$f(1,0)=1=\binom{1-0+1}{0},f(1,1)=1=\binom{1-1+1}{1}$。$n=2$ 时,$f(2,0)=1=\binom{2-0+1}{0},f(2,1)=2=\binom{2-1+1}{2},f(2,2)=0=\binom{2-2+1}{2}$。

下面假设 $n\lt n_0$ 时成立,证明 $n=n_0$ 时也成立。若 $n\gt k\gt 0$,此时有
\begin{align}
    f(n,k)&=f(n-2,k-1)+f(n-1,k) \\
    &=\binom{n-2-(k-1)+1}{k-1}+\binom{n-1-k+1}{k} \\
    &=\binom{n-k}{k-1}+\binom{n-k}{k} \\
    &=\binom{n-k+1}{k}.
\end{align}
若 $n=k\ge 3$,则 $f(n,k)=0=\binom{1}{k}=\binom{n-k+1}{k}$。
若 $k=0$,此时有 $f(n,k)=1=\binom{n-k+1}{k}$。

故所求通项公式为 $f(n,k)=\binom{n-k+1}{k}$($n\ge k$)。

\paragraph{(3)}
将 $f(n,k)$ 分为两类:
\begin{itemize}
    \item 若 $1,n$ 均被选取,则不包含在 $g(n,k)$ 中。剩余的 $k-2$ 个数须在 $3\sim (n-2)$ 中选取,因此共 $f(n-4,k-2)$ 种;
    \item 若 $1,n$ 不都被选取,这部分完全满足 $g(n,k)$ 的条件。
\end{itemize}
于是 $f(n,k)=f(n-4,k-2)+g(n,k)$,即 $g(n,k)=f(n,k)-f(n-4,k-2)$。

\section*{4.10.}
\paragraph{(1)}
记方案数为 $f_n$,同时令 $g_n$ 表示铺满 $1\times n$ 的道路,并且右侧突出一块的方案数。这里突出一块是指,最后 $1\times 1$ 区域中,因为放入了一个斜边长为 2 的等腰直角三角形,在整个道路外面突出了一块边长为 1 的等腰直角三角形区域。由对称性,$g_n$ 只包括右上突出一块的情况。

对于 $f_n$,考虑最后 $1\times 1$ 区域是否存在斜边长为 2 的等腰直角三角形:
\begin{itemize}
    \item 不存在,最后 $1\times 1$ 区域共三种可能,故共有 $3f_{n-1}$ 种可能;
    \item 存在,共 $2g_{n-1}$ 种可能。
\end{itemize}
对于 $g_n$,同样考虑 $1\times n$ 道路最后的 $1\times 1$ 区域,是否存在另一个斜边长为 2 的等腰直角三角形:
\begin{itemize}
    \item 不存在,则放入了一个边长为 1 的等腰直角三角形,共 $f_{n-1}$ 种可能;
    \item 存在,共 $g_{n-1}$ 种可能。
\end{itemize}
于是有
\begin{align}
    f_n&=3f_{n-1}+2g_{n-1}, \\
    g_n&=f_{n-1}+g_{n-1}.
\end{align}
由于 $f_n-2g_n=(3f_{n-1}+2g_{n-1})-2(f_{n-1}+g_{n-1})=f_{n-1}$,带入得
\begin{equation}
    f_n=3f_{n-1}+2g_{n-1}=3f_{n-1}+(f_{n-1}-f_{n-2})=4f_{n-1}-f_{n-2}.
\end{equation}
对于特征方程为 $x^2-4x+1=0$,解得 $x=\frac{4\pm\sqrt{16-4}}{2}=2\pm\sqrt 3$,故通项公式形如
\begin{equation}
    a_n=c_1(2+\sqrt 3)^n+c_2(2-\sqrt 3)^n.
\end{equation}
初值 $f_0=1,f_1=3$。代入得
\begin{align}
    1&=c_1+c_2, \\
    3&=c_1(2+\sqrt 3)+c_2(2-\sqrt 3).
\end{align}
解得 $c_1=\frac 12+\frac{\sqrt 3}{6},c_2=\frac 12-\frac{\sqrt 3}{6}$。因此方案数为
\begin{equation}
    f_n=\left(\frac 12+\frac{\sqrt 3}{6}\right)(2+\sqrt 3)^n+\left(\frac 12-\frac{\sqrt 3}{6}\right)(2-\sqrt 3)^n.
\end{equation}

\paragraph{(2)} 按照前一问的方式定义 $a_n,b_n$,只是将方案数改为了砖块数。同样地,按照前一问的分类,可以得到递推关系:
\begin{align}
    a_n&=(3a_{n-1}+1+2+2)+(2b_{n-1}+2)=3a_{n-1}+2b_{n-1}+7, \\
    b_n&=(a_{n-1}+2)+(b_{n-1}+1)=a_{n-1}+b_{n-1}+3.
\end{align}
同理有
\begin{equation}
    a_n-2b_n=(3a_{n-1}+2b_{n-1}+7)-2(a_{n-1}+b_{n-1}+3)=a_{n-1}+1.
\end{equation}
代入得
\begin{equation}
    a_n=3a_{n-1}+(a_{n-1}-a_{n-2}-1)+7=4a_{n-1}-a_{n-2}+7.
\end{equation}
特解形如 $a_n=c$,带入得
\begin{equation}
    c=4c-c+7,
\end{equation}
即 $c=-\frac 72$,故通项公式形如
\begin{equation}
    a_n=-\frac 72+c_1(2+\sqrt 3)^n+c_2(2-\sqrt 3)^n.
\end{equation}
初值为 $a_0=0,a_1=5$,代入得
\begin{align}
    0&=-\frac 72+c_1+c_2, \\
    5&=-\frac 72+c_1(2+\sqrt 3)+c_2(2-\sqrt 3).
\end{align}
解得 $c_1=\frac{7+\sqrt 3}{4},c_2=\frac{7-\sqrt 3}{4}$,即砖块数总和为
\begin{equation}
    a_n=-\frac 72+\frac{7+\sqrt 3}{4}(2+\sqrt 3)^n+\frac{7-\sqrt 3}{4}(2-\sqrt 3)^n.
\end{equation}

\end{document}
