% Homework template for Inference and Information
% UPDATE: September 26, 2017 by Xiangxiang
\documentclass[a4paper]{article}
\usepackage{ctex}
\ctexset{
proofname = \heiti{证明}
}
\usepackage{amsmath, amssymb, amsthm}
% amsmath: equation*, amssymb: mathbb, amsthm: proof
\usepackage{moreenum}
\usepackage{mathtools}
\usepackage{url}
\usepackage{bm}
\usepackage{enumitem}
\usepackage{graphicx}
\usepackage{subcaption}
\usepackage{booktabs} % toprule
\usepackage[mathcal]{eucal}
\usepackage[thehwcnt = 1]{iidef}

\usepackage{ctex}
\setCJKfamilyfont{myfont}{simkai.ttf}
\newcommand{\MyFont}{\CJKfamily{myfont}}

\thecourseinstitute{清华大学计算机系}
\thecoursename{组合数学}
\theterm{2024年秋季学期}

\slname{\heiti{解}}
\begin{document}
\courseheader

\begin{enumerate}

4.1

\begin{solution}

(1)

\begin{align}
    G_n &= F_{2n}\\
    G_n-3G_{n-1}+G_{n-2} &= F_{2n}-3F_{2n-2}+F_{2n-4}\\
    &= F_{2n-1}+F_{2n-2}-3F_{2n-3}-3F_{2n-4}+F_{2n-4}\\
    &= 2F_{2n-2}-2F_{2n-3}-2F_{2n-4}=0
\end{align}

(2)

\begin{align}
    G(0) &= 1, G(1)= 2, G(2)=5\\
    G(x) &= \Sigma_{n=0}G_n x^n\\
    (1-3x+x^2)G(x) &= \Sigma_{n=0}G_nx^n - 3\Sigma_{n=0}G_nx^{n+1}+\Sigma_{n=0}G_n x^{n+2}\\
    (1-3x+x^2)G(x) &= G_0+G_1x+\Sigma_{n=2}G_nx^n-3G_0x-3\Sigma_{n=2}G_{n-1}x^n+\Sigma_{n=2}G_{n-2}x^n\\
    (1-3x+x^2)G(x)&=1+2x-3x+\Sigma_{n=2}(G_n-3G_{n-1}+G_{n-2})x^n\\
    (1-3x+x^2)G(x)&=1-x\\
    G(x)&=\frac{1-x}{1-3x+x^2}
\end{align}

\end{solution}

4.2

\begin{solution}

\begin{align}
    G(x) &= \frac{1}{1-x+x^2}\\
    (1-x+x^2)G(x) &= 1\\
    G(x)-xG(x)+x^2G(x) &= 1\\
    a_n-a_{n-1}+a_{n-2} &= 0\\
    a_0=1\\
    a_1=1
\end{align}

\end{solution}

4.3

\begin{solution}

\begin{align}
    a_n-2a_{n-1}-3a_{n-2} &= c3^n+d(-1)^n-2c3^{n-1}-2d(-1)^{n-1}-3c3^{n-2}-3d(-1)^{n-2}\\
    &= c3^n+d(-1)^n-2c3^{n-1}+2d(-1)^{n}-c3^{n-1}-3d(-1)^{n}\\
    &=c3^n-2c3^{n-1}-c3^{n-1}+d(-1)^n+2d(-1)^{n}-3d(-1)^{n}\\
    &=0
\end{align}

\end{solution}

4.4

\begin{solution}

\begin{align}
    a_n-2a_{n-1}+a_{n-2}&=5\\
    a_{n+1}-2a_{n}+a_{n-1} - a_n+2a_{n-1}-a_{n-2}&=0\\
    a_{n+1}-3a_n+3a_{n-1}-a_{n-2} &= 0\\
    x^3-3x^2+3x-1&=0\\
    (x-1)^3&=0\\
    x_1=x_2=x_3&=1
\end{align}

\begin{align}
    a_n &= (A+Bn+Cn^2)1^n\\
    as\ a_0=1, a_1=2,a_2&=8\\
    so\ A&1=\\
    and\ A+B+C&=2\\
    and\ A+2B+4C&=8\\
    so\ A=1,B=-1.5,C&=2.5\\
    a_n &=1-1.5n+2.5n^2
\end{align}

\end{solution}

4.5

\begin{solution}

\begin{align}
    a_n &= 4a_{n-1}+4^{n-2}-a_{n-2}\\
    a_n - 4a_{n-1} + a_{n-2} &= 4^{n-2}\\
    a &= 4^n\\
    x^2-4x+1&=0\\
    x_1 &= 2+\sqrt{3}\\
    x_2 &= 2-\sqrt{3}\\
    a_n &= A(2+\sqrt{3})^n +B (2-\sqrt{3})^n+4^n\\
    a_0 &= A+B+1=0\\
    a_1 &= A(2+\sqrt{3}) +B (2-\sqrt{3})+4=0\\
    A&=\frac{-2\sqrt{3}-3}{6}\\
    B&=\frac{2\sqrt{3}-3}{6}\\
    a_n&=\frac{-2\sqrt{3}-3}{6}(2+\sqrt{3})^n +\frac{2\sqrt{3}-3}{6} (2-\sqrt{3})^n+4^n
\end{align}

\end{solution}

4.6

\begin{solution}

设最小移动次数为f(n),同时标准汉诺塔问题的操作次数是
\[
h(n) = 2^n-1
\]

那么显然,f(1)=1, f(2)=2, f(3) = 5。

同时对于有n个圆盘的情况:

如果n是偶数,那么先将n-1个圆盘借助B柱全都转移到C柱上,花费h(n-1)次,然后将最后一个圆盘放在B柱上,然后借助B柱,将n-3个圆盘从C柱上转移回A柱上,需要h(n-3)次,然后将n-2号盘子从C柱移动到B柱,接着将A柱上的n-3个圆盘通过f(n-3)步操作,将偶数编号放在B上,将奇数编号放在C上;

同理,如果n是奇数,也是完全一样的

所以得到递推关系为
\begin{align}
    f(n) = f(n-3)+h(n-1)&+h(n-3)+2 = f(n-3) + 5*2^{n-3} \\
    x &= c2^n\\
    c2^n &= c2^{n-3}+ 5*2^{n-3} = 8c2^{n-3}\\
    c &= \frac{5}{7}\\
    x &= \frac{5}{7} 2^n\\
    x^3 &= 1 \\
    x_1 = 1, x_2 &= e^{\frac{2\pi}{3}}, x_3 = e^{-\frac{2\pi}{3}}\\
    f(n) = \frac{5}{7} 2^n + A +&Be^{\frac{2n\pi}{3}} + Ce^{-\frac{2n\pi}{3}}\\
    f(n) = -\frac{2}{3} + \frac{5}{7}2^n&-\frac{1}{21}cos\frac{2n\pi}{3} +\frac{\sqrt{3}}{7}sin\frac{2n\pi}{3}
\end{align}


\end{solution}


4.7

\begin{solution}

\begin{align}
    a_n = (k-1)a_{n-1}+(k-1)a_{n-2}
\end{align}

\begin{align}
    x^2-(k-1)x-(k-1)&=0\\
    x_1 &= \frac{k-1+\sqrt{(k+3)(k-1)}}{2}\\
    x_2&= \frac{k-1-\sqrt{(k+3)(k-1)}}{2}\\
    a_n &= Ax_1^n+Bx_2^n\\
    so\ A+B&=0\\
    and\ A\frac{k-1+\sqrt{(k+3)(k-1)}}{2}&+B\frac{k-1-\sqrt{(k+3)(k-1)}}{2}=k\\
    so\ A &= \frac{k}{\sqrt{(k+3)(k-1)}}\\
    B &= -\frac{k}{\sqrt{(k+3)(k-1)}}\\
    a_n = \frac{k}{\sqrt{(k+3)(k-1)}}&\frac{k-1+\sqrt{(k+3)(k-1)}}{2}^n \\
    -\frac{k}{\sqrt{(k+3)(k-1)}}&\frac{k-1-\sqrt{(k+3)(k-1)}}{2}^n
\end{align}

\end{solution}
4.8

\begin{solution}

数学归纳法证明:

\begin{align}
    \sum_{k=1}^n k^4 = \left( \frac{n(n+1)}{2} \right)^2
\end{align}

当 \(n=1\) 时:

\begin{align}
    \sum_{k=1}^1 k^4= 1
\end{align}

假设对于某个 \(n\) 成立,即:
\begin{align}
   \sum_{k=1}^n k^4 = \left( \frac{n(n+1)}{2} \right)^2
\end{align}

需要证明对于 \(n+1\) 也成立:
\begin{align}
    \sum_{k=1}^{n+1} k^4 &= \sum_{k=1}^n k^4 + (n+1)^4\\
    &= \left( \frac{n(n+1)}{2} \right)^2 + (n+1)^4\\
    &= \frac{n^2(n+1)^2}{4} + (n+1)^4 = \frac{n^2(n+1)^2 + 4(n+1)^4}{4}
\end{align}

 即证:
\begin{align}
     n^2(n+1)^2 + 4(n+1)^4 &= (n+1)^2(5n^2 + 8n + 4)\\
     &= (n+1)^2(n^2 + 4n^2 + 8n + 4) \\
     &=(n+1)^2 \left( n^2 + 4(n+1)^2 \right)
\end{align}
拆开发现确实如此,证毕

\end{solution}

4.9

\begin{solution}
(1)

\begin{align}
    f(n,k) = f(n-1, k) + f(n-2, k-1)
\end{align}

(2)

数学归纳法证明
\[
f(n,k) = C_{n-k+1}^k
\]

首先,
\[
f(1,1)=1,f(2,1)=2,f(3,1) = 1,f(3,2)=1, f(4,1)=4,f(4,2)=3,f(5,1)=5,f(5,2)=6,f(5,3)=1;
\]

其次,如果对$n < N_0, k<K_0$的时候成立,那么对于$n=N_0, k=K_0$时,
\[
f(n-1, k) + f(n-2, k-1) = 
C_{n-k}^k + C_{n-k}^{k-1} = C_{n-k+1}^k
\]

证毕

(3)

 如果选了n,那么不能选1和n-1,剩下k-1个数就需要在2~n-2中选择,一共有f(n-3,k-1)中选择方式。
 
 如果没选n,那么剩下n-1个数里随便选k个数,一共有f(n-1,k)中选择方式。
 
 所以
 
$$
g(n,k) = f(n-1,k)+ f(n-3,k-1) = C_{n-k}^k + C_{n-k-1}^{k-1}
$$

\end{solution}


4.10

\begin{solution}

(1)


\begin{align}
    a_n &= 3a_{n-1} + 2\Sigma_{i=0}^{n-2}a_i\\
    a_{n+1} &= 3a_{n} + 2\Sigma_{i=0}^{n-1}a_i\\
    a_{n+1} &-4a_{n}+a_{n-1} = 0\\
    x^2 &- 4x+1=0\\
    x_1=2+\sqrt{3}&, x_2=2-\sqrt{3}\\
    a_n &= Ax_1^n+Bx_2^n\\
    a_0=1 &,a_1=3\\
    a_n &= \frac{3+\sqrt{3}}{6}(2+\sqrt{3})^n + \frac{3-\sqrt{3}}{6}(2-\sqrt{3})^n 
\end{align}


(2)

设砖头总和为$b_n$

如果最后一个矩形块是$1\times 1$的,那么可能是直接$1 \times 1$矩形实现的,可能是两块直角边为1的三角形合起来的,这种情况下方案的砖块数之和为$3b_{n-1}+5a_{n-1}$。如果最后一个矩形块是$1\times k$的,那么说明是通过斜边为2的直角三角形块拼接而成,同时两侧空隙部分填补的是直角边为1的三角形块。考虑到翻转情况,这种方案下的砖块数之和为$2\Sigma_{i=2}^n[b_{n-i}+(i+1)a_{n-i}]$

\begin{align}
    b_n = 3b_{n-1}+5a_{n-1}+2\Sigma_{i=2}^n[b_{n-i}+(i+1)a_{n-i}]\\
    b_{n-1} = 3b_{n-2}+5a_{n-2}+2\sigma_{i=2}^{n-1}[b_{n-i}+(i+1)a_{n-i}]\\
    b_n - 4b_{n-1}+b_{n-2} = 5a_{n-1}+a_{n-2}+2\Sigma_{i=0}^{n-3}a_i\\
    b_n - 4b_{n-1}+b_{n-2} = \frac{2\sqrt{3}}{3} (2+\sqrt{3})^n - \frac{2\sqrt{3}}{3} (2-\sqrt{3})^n\\
    x_1 = 2+\sqrt{3}, x_2 = 2- \sqrt{3}\\
    b_n = (An+B)x_1^n+(Cn+D)x_2^n\\
    B+D=0\\
    (2+\sqrt{3})(A+B)+(2-\sqrt{3})(C+D)=5\\
    (2+\sqrt{3})^2(2A+B)+(2-\sqrt{3})^2(2C+D)=36\\
    (2+\sqrt{3})^3(3A+B)+(2-\sqrt{3})^3(3C+D)=199\\
    A =\frac{2+\sqrt{3}}{3}, B = \frac{\sqrt{3}}{18}, C=\frac{2-\sqrt{3}}{3}, D =- \frac{\sqrt{3}}{18}\\
    b_n = (\frac{2+\sqrt{3}}{3}n+\frac{\sqrt{3}}{18})(2+\sqrt{3})^n+(\frac{2-\sqrt{3}}{3}n+- \frac{\sqrt{3}}{18})(2- \sqrt{3})^n
\end{align}

\end{solution}


\end{enumerate}
\end{document}

%%% Local Variables:
%%% mode: late\rvx
%%% TeX-master: t
%%% End:
