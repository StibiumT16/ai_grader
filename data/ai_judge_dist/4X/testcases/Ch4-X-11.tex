\documentclass{article}
\usepackage{graphicx} % Required for inserting images
\usepackage{amsmath}
\usepackage{ctex} % 支持中文
\setlength{\parindent}{0pt}  % 取消全局缩进
\title{组合数学: 第四章作业}

\begin{document}

\maketitle

\section*{4.1}

\textbf{解答:}

    \section*{(1) 证明:\( G_n - 3G_{n-1} + G_{n-2} = 0 \quad (n \geq 2) \)}
    
    
    由 \( G_n = F_{2n} \)有:
    \[
    G_{n-1} = F_{2(n-1)} = F_{2n-2}, \quad G_{n-2} = F_{2(n-2)} = F_{2n-4}.
    \]
    
    代入原式得:
    
    \[
    G_n - 3G_{n-1} + G_{n-2} = F_{2n} - 3F_{2n-2} + F_{2n-4}.
    \]
    
    又 \( F_n = F_{n-1} + F_{n-2} \),可得:
    
    \[
    F_{2n} = F_{2n-1} + F_{2n-2}, \quad F_{2n-2} = F_{2n-3} + F_{2n-4}.
    \]
    
    再次代入后得到:
    
    \[
    F_{2n} - 3F_{2n-2} + F_{2n-4} = F_{2n-3} + F_{2n-4} - F_{2n-2} = 0,
    \]
    
    因此原式 \( G_n - 3G_{n-1} + G_{n-2} = 0 \) 对于 \( n \geq 2 \) 成立。

    \section*{(2) 求数列 \( \{G_n\} \) 的母函数。}
    
    设 $H(x) = G_0 + G_1 x + G_2 x^2 + G_3 x^3 + \dots$,则
    \begin{align}
    H(x) &= G_0 + G_1 x + G_2 x^2 + G_3 x^3 + G_4 x^4 + \dots \tag{1} \\
    3xH(x) &= 3G_0 x + 3G_1 x^2 + 3G_2 x^3 + 3G_3 x^4 + \dots \tag{2} \\
    x^2 H(x) &= G_0 x^2 + G_1 x^3 + G_2 x^4 + \dots \tag{3}
    \end{align}
    
    \noindent
    由 (1) - (2) + (3),得:
    \[
    (1 - 3x + x^2) H(x) = G_0 + G_1 x - 3G_0 x
    \]
    
    \noindent
    又已知 $G_n = F_{2n}$,则 $G_0 = F_0 = 0$,$G_1 = F_2 = 1$
    
    \noindent
    所以,
    \[
    H(x) = \frac{x}{1 - 3x + x^2} = \frac{x}{\left(\frac{3 + \sqrt{5}}{2} - x\right)\left(\frac{3 - \sqrt{5}}{2} - x\right)}
    \]
    
    设
    \[
    H(x) = \frac{A}{\frac{3 + \sqrt{5}}{2} - x} + \frac{B}{\frac{3 - \sqrt{5}}{2} - x}
    \]
    
    则有:
    \[
    \begin{cases}
    A + B = 1 \\
    \frac{3 - \sqrt{5}}{2} A + \frac{3 + \sqrt{5}}{2} B = 0
    \end{cases}
    \Rightarrow 
    \begin{cases}
    A = \frac{5 + 3\sqrt{5}}{10} \\
    B = \frac{5 - 3\sqrt{5}}{10}
    \end{cases}
    \]
    即:
    \[
    H(x) = -\frac{\sqrt{5}}{5} \sum_{n=0}^{\infty} \left( \frac{3 - \sqrt{5}}{2} x \right)^n + \frac{\sqrt{5}}{5} \sum_{n=0}^{\infty} \left( \frac{3 + \sqrt{5}}{2} x \right)^n
    \]
    
    \[
    \boxed{ = \frac{\sqrt{5}}{5} \sum_{n=0}^{\infty} \left[ \left( \frac{3 + \sqrt{5}}{2} \right)^n - \left( \frac{3 - \sqrt{5}}{2} \right)^n \right] x^n}
    \]

\section*{4.2}

\textbf{解答:}

由题意有:

\[
(1 - x + x^2)A(x) = 1.
\]

将 $A(x)$ 展开,得到:

\[
A(x) = \sum_{n=0}^{\infty} a_n x^n.
\]

因此代入后得到,

\[
(1 - x + x^2) \sum_{n=0}^{\infty} a_n x^n = 1.
\]

展开后有:

\[
\sum_{n=0}^{\infty} a_n x^n - \sum_{n=0}^{\infty} a_n x^{n+1} + \sum_{n=0}^{\infty} a_n x^{n+2} = 1.
\]

将求和打开:

\[
a_0 + (a_1 - a_0) x + (a_2 - a_1 + a_0) x^2 + (a_3 - a_2 + a_1) x^3 + \cdots = 1.
\]

为了使上式成立,必须有每一阶的系数都为零,因此,得到以下递推关系:

\[
a_n - a_{n-1} + a_{n-2} = 0, \quad \text{对于} \, n \geq 2.
\]

\[a_1 = a_0
\]

即数列 $\{a_n\}$ 满足以下二阶齐次线性常系数递推关系:

\[
a_n = a_{n-1} - a_{n-2} \quad (n \geq 2).
\]

母函数 $A(x)$ 的常数项即为 $a_0$, 可以直接得到:

\[
a_0 = A(0) = \frac{1}{1 - 0 + 0^2} = 1,
\]

\[
a_1 = \left. \frac{d}{dx} A(x) \right|_{x=0} = \left. \frac{d}{dx} \frac{1}{1 - x + x^2} \right|_{x=0}.
\]

又$A(x)$ 的导数:

\[
A'(x) = - \frac{-1 + 2x}{(1 - x + x^2)^2}.
\]

代入 $x = 0$:

\[
a_1 = \frac{-(-1 + 0)}{(1 - 0 + 0^2)^2} = \frac{1}{1^2} = 1.
\]

因此:

\[
a_0 = a_1 = 1.
\]

最终得到,

数列 $\{a_n\}$ 满足的二阶齐次线性常系数递推关系为:

\[
\boxed{a_n = a_{n-1} - a_{n-2} \quad (n \geq 2)}
\]

且初始条件为:

\[
\boxed{a_0 = 1, \quad a_1 = 1}
\]

\section*{4.3}
\textbf{解答:}


\[
 3 \text{ 和 } -1 \text{ 为特征根}
\]

特征方程为
\[
x^2 - 2x - 3 = 0
\]

最终得到, 满足题意的一个线性常系数递推关系为:

\[
\boxed{a_n - 2a_{n-1} - 3a_{n-2} = 0}
\]

\section*{4.4}
\textbf{解答:}

\[
a_n - 2a_{n-1} + a_{n-2} = 5
\]

齐次特征方程是:

\[
r^2 - 2r + 1 = 0 \quad \Rightarrow \quad (r - 1)^2 = 0,
\]

所以,齐次方程的解为:

\[
a_n^{\text{齐}} = A + Bn.
\]


因为\[5 = 1^n * 5\], 且1是二重根, 设特解的形式为:

\[
a_n^{\text{特}} = k n^2.
\]

将特解代入递推关系 \(a_n - 2a_{n-1} + a_{n-2} = 5\) 中:

- \(a_n^{\text{特}} = k n^2\),

- \(a_{n-1}^{\text{特}} = k (n-1)^2 = k (n^2 - 2n + 1)\),

- \(a_{n-2}^{\text{特}} = k (n-2)^2 = k (n^2 - 4n + 4)\).

代入递推关系:

\[
a_n^{\text{特}} - 2a_{n-1}^{\text{特}} + a_{n-2}^{\text{特}} = k n^2 - 2k (n^2 - 2n + 1) + k (n^2 - 4n + 4).
\]

展开各项并合并同类项:

\[
k n^2 - 2k (n^2 - 2n + 1) + k (n^2 - 4n + 4) = k n^2 - 2k n^2 + 4k n - 2k + k n^2 - 4k n + 4k.
\]

整理后:

\[
= (k n^2 - 2k n^2 + k n^2) + (4k n - 4k n) + (-2k + 4k).
\]

简化:

\[
= 0n^2 + 0n + 2k.
\]

要求这个表达式等于 5,因此有:

\[
2k = 5 \quad \Rightarrow \quad k = \frac{5}{2}.
\]

因此,通解为齐次解和特解的和:

\[
a_n = a_n^{\text{齐}} + a_n^{\text{特}} = A + Bn + \frac{5}{2}n^2.
\]

- 当 \(n = 0\) 时:

\[
a_0 = A + B \times 0 + \frac{5}{2} \times 0^2 = 1 \quad \Rightarrow \quad A = 1.
\]

- 当 \(n = 1\) 时:

\[
a_1 = A + B \times 1 + \frac{5}{2} \times 1^2 = 2 \quad \Rightarrow \quad 1 + B + \frac{5}{2} = 2.
\]

解得:

\[
B + \frac{7}{2} = 2 \quad \Rightarrow \quad B = 2 - \frac{7}{2} = -\frac{3}{2}.
\]

最终得到:
\[
\boxed{
a_n = 1 - \frac{3}{2}n + \frac{5}{2}n^2}
\]

\section*{4.5}
\textbf{解答:}

设 \( b_n \) 为不包含 "AB" 子串的字符串数目:

- 假设第一位4种可能都取, 有 \(4b_{n-1} \)种情况

- 但是考虑到其中会出现第一位为A, 第二位为B的违法情况, 有\(b_{n-2} \)种

因此,递推关系为:
\[
b_n = 4b_{n-1} - b_{n-2}
\]

对于 \( a_n \),可以用总数减去不包含 "AB" 的情况:
\[
a_n = 4^n - b_n
\]

计算得到以下初始条件:
\[
b_0 = 1, \quad b_1 = 4, \quad b_2 = 15, \quad b_3 = 56
\]

构造其特征方程:
\[
q^2 - 4q + 1 = 0
\]

解这个特征方程,得到两个特征根:
\[
q = 2 \pm \sqrt{3}
\]

递推关系的通解可以表示为:
\[
b_n = A (2 + \sqrt{3})^n + B (2 - \sqrt{3})^n
\]


使用初始条件 \( b_0 = 1 \) 和 \( b_1 = 4 \) 来确定常数 \( A \) 和 \( B \):

1. 当 \( n = 0 \) 时:
   \[
   b_0 = A + B = 1
   \]

2. 当 \( n = 1 \) 时:
   \[
   b_1 = A(2 + \sqrt{3}) + B(2 - \sqrt{3}) = 4
   \]

解方程组:
\[
\begin{cases}
A + B = 1 \\
A(2 + \sqrt{3}) + B(2 - \sqrt{3}) = 4
\end{cases}
\]
得到:
\[
A = \frac{3 + 2\sqrt{3}}{6}, \quad B = \frac{3 - 2\sqrt{3}}{6}
\]


因此,\( b_n \) 的通项公式为:
\[
b_n = \frac{3 + 2\sqrt{3}}{6} (2 + \sqrt{3})^n + \frac{3 - 2\sqrt{3}}{6} (2 - \sqrt{3})^n
\]

根据关系 \( a_n = 4^n - b_n \),最终得到包含至少一个 "AB" 子串的 \( n \) 位字符串数目:
\[
\boxed{a_n = 4^n - \left(\frac{3 + 2\sqrt{3}}{6} (2 + \sqrt{3})^n + \frac{3 - 2\sqrt{3}}{6} (2 - \sqrt{3})^n\right)}
\]

\section*{4.6}
\textbf{解答:}

假设圆盘个数 \(n\) 为奇数:

\begin{enumerate}
    \item 先将 \(n-1\) 个盘从\(A\)通过 \(C\) 移动到 \(B\);
    \item 将第 \(n\) 个盘从\(A\)移动到 \(C\);
    \item 再将 \(n-3\) 个盘从从\(B\)通过 \(C\) 移动到 \(A\);
    \item 最后将第 \(n-2\) 个盘移动到 \(C\)。
    \item 循环往复
\end{enumerate}

若 \(n\) 为偶数:

\begin{enumerate}
    \item 先将 \(n-1\) 个盘从\(A\)通过 \(B\) 移动到 \(C\);
    \item 将第 \(n\) 个盘从\(A\)移动到 \(B\);
    \item 再将 \(n-3\) 个盘从\(C\)通过 \(B\) 移动到 \(A\);
    \item 最后将第 \(n-2\) 个盘移动到 \(B\)。
    \item 循环往复
\end{enumerate}

对于 \(n\) 为奇数或者偶数的情形,上述步骤均成立,只不过 \(B\) 和 \(C\) 的角色对调。

因此,有递推关系:
\[
K(n) = H(n-1) + 1 + H(n-3) + 1 + K(n-3)
\]
其中:
\[
K(1) = 1, \quad K(2) = 2, \quad K(3) = 5, \quad H(k) \text{ 为汉诺塔步数序列。}
\]

递推关系可化简为:
\[
K(n) - K(n-3) = 2^{n-1} + 2^{n-3}
\]
\[
2[K(n-1) - K(n-4)] = 2^{n-1} + 2^{n-3}
\]

得到特征方程:
\[
(x-2)(x^3-1) = 0
\]
化简为:
\[
(x-2)(x-1)(x^2 + x + 1) = 0
\]

解得:
\[
x_1 = 1, \quad x_2 = 2, \quad x_{3,4} = -\frac{1}{2} \pm \frac{\sqrt{3}}{2}i
\]

所以通解为:
\[
K(n) = A \cdot 2^n + B + C \cos \frac{2}{3} \pi n + D \sin \frac{2}{3} \pi n
\]

使用初始条件 \(K(1) = 1\), \(K(2) = 2\), \(K(3) = 5\),以及通过递推公式计算的 \(K(4) = 11\),建立方程组:
\[
\begin{cases}
K(1) = 2A + B - \frac{1}{2}C + \frac{\sqrt{3}}{2}D = 1, \\
K(2) = 4A + B - \frac{1}{2}C - \frac{\sqrt{3}}{2}D = 2, \\
K(3) = 8A + B + C = 5, \\
K(4) = 16A + B - \frac{1}{2}C + \frac{\sqrt{3}}{2}D = 11.
\end{cases}
\]

解这个方程组,得到:
\[
A = \frac{5}{7}, \quad B = -\frac{2}{3}, \quad C = -\frac{1}{21}, \quad D = \frac{\sqrt{3}}{7}.
\]

因此,通解的具体表达式为:
\[
\boxed{ K(n) = \frac{5}{7} \cdot 2^n - \frac{2}{3} - \frac{1}{21} \cos \left( \frac{2\pi}{3} n \right) + \frac{\sqrt{3}}{7} \sin \left( \frac{2\pi}{3} n \right)}
\]

\section*{4.7}
\textbf{解答:}

设:

\begin{itemize}
    \item \( a_n \):长度为 \( n \) 的、最后两个字母 \textbf{不相同} 的字符串数量。
    \item \( b_n \):长度为 \( n \) 的、最后两个字母 \textbf{相同但前一个字母与前前一个字母不同} 的字符串数量(即最后两个字母相同,但没有出现 3 连同字母)。
\end{itemize}

总的方案数 \( S(n) \) 为:
\[
S(n) = a_n + b_n
\]

1. 对于 \( a_n \):
   \begin{itemize}
       \item 来源 1:长度为 \( n-1 \) 的字符串,最后两个字母不相同,然后添加一个与前一个字母 \textbf{不相同} 的字母。
       \[
       \text{方案数为: } a_{n-1} \times (k - 1)
       \]
       \item 来源 2:长度为 \( n-1 \) 的字符串,最后两个字母相同,然后添加一个与前一个字母 \textbf{不相同} 的字母(防止出现 3 连同字母)。
       \[
       \text{方案数为:  }  b_{n-1} \times (k - 1)
       \]
       \item 总计:
       \[
       a_n = (a_{n-1} + b_{n-1}) \times (k - 1) = S(n-1) \times (k - 1)
       \]
   \end{itemize}

2. 对于 \( b_n \):
   \begin{itemize}
       \item 来源:长度为 \( n-1 \) 的字符串,最后两个字母不相同,然后添加一个与前一个字母 \textbf{相同} 的字母(形成最后两个字母相同,但不至于 3 连同)。
       \[
       \text{方案数为: } a_{n-1} \times 1
       \]
       \item 总计:
       \[
       b_n = a_{n-1}
       \]
   \end{itemize}

初始条件:

\begin{itemize}
    \item 当 \( n = 1 \) 时:
    \[
    a_1 = k \quad \text{(长度为 1,无连续相同字母的可能)}, \quad b_1 = 0 \quad \text{(无法有连续相同字母)}.
    \]
    \item 当 \( n = 2 \) 时:
    \[
    a_2 = k \times (k - 1)
    \]
    \[
    b_2 =  k
    \]
\end{itemize}

根据上述定义,设总方案数 为\( S(n) \) 等于:
\[
S(n) = a_n + b_n
\]

又,
\[
       b_n = a_{n-1}
       \]
       
于是:
\[
S(n) = a_n + a_{n-1}
\]

将 \(a_n\) 的表达式代入:
\[
\begin{aligned}
S(n) &= a_n + a_{n-1} \\
&= \left[(a_{n-1} + a_{n-2}) \times (k-1)\right] + a_{n-1} \\
&= a_{n-1} \times (k-1) + a_{n-2} \times (k-1) + a_{n-1} \\
&= a_{n-1} \times (k-1+1) + a_{n-2} \times (k-1) \\
&= a_{n-1} \times k + a_{n-2} \times (k-1).
\end{aligned}
\]

由于 \(a_{n-1} = S(n-1) - a_{n-2}\),所以:
\[
\begin{aligned}
S(n) &= \left[S(n-1) - a_{n-2}\right] \times k + a_{n-2} \times (k-1) \\
&= S(n-1) \times k - a_{n-2} \times k + a_{n-2} \times (k-1).
\end{aligned}
\]

展开并简化:
\[
\begin{aligned}
S(n) &= S(n-1) \times k - a_{n-2} \\
&= S(n-1) \times (k-1) + S(n-1) - a_{n-2} \\
&= S(n-1) \times (k-1) + a_{n-1}
\end{aligned}
\]

又\(a_{n-1} = (k-1) \times (a_{n-2} + b_{n-2})\),所以:
\[
\begin{aligned}
S(n) &= S(n-1) \times (k-1) + S(n-2) \times (k-1)\\
\end{aligned}
\]

于是得到递推关系:
\[
S(n) = (k-1) \left[S(n-1) + S(n-2)\right].
\]

因此我们有特征方程:
\[
r^2 - (k-1)r - (k-1) = 0
\]
其中:

1. 特征根 \(r_1\) 和 \(r_2\):
\[
r_1 = \frac{(k-1) + \Delta}{2}, \quad r_2 = \frac{(k-1) - \Delta}{2}
\]
\[
\Delta = \sqrt{(k-1)(k+3)}
\]

因此有,
\[
S(n) = C_1 r_1^n + C_2 r_2^n
\]

2. 联立初始条件得到的方程组求解系数 \(C_1\) 和 \(C_2\):

当 \(n = 1\) 时:
\[
S(1) = C_1 r_1 + C_2 r_2 = k
\]

当 \(n = 2\) 时:
\[
S(2) = C_1 r_1^2 + C_2 r_2^2 = k^2
\]
\[
C_1 = \frac{k(k-1 + \Delta)}{2(k-1)\Delta}, \quad C_2 = \frac{-k(k-1 - \Delta)}{2(k-1)\Delta}
\]

代入至表达式中, 最终得到:
\[
\boxed{S(n) = \frac{k(k-1+\Delta)}{2(k-1)\Delta} \left(\frac{(k-1) + \Delta}{2}\right)^n + \frac{-k(k-1-\Delta)}{2(k-1)\Delta} \left(\frac{(k-1) - \Delta}{2}\right)^n 
\boxed{\Delta = \sqrt{(k-1)(k+3)}}
}
\]

\section*{4.8}
\textbf{解答:}


首先,$S_n$ 的一次差分为:
\[
\Delta S_n = S_n - S_{n-1} = n^4
\]

由于 $n^4$ 是四次多项式,根据差分理论,其第五阶差分为零。因此,$S_n$ 满足六阶线性齐次差分方程:
\[
S_n - 6S_{n-1} + 15S_{n-2} - 20S_{n-3} + 15S_{n-4} - 6S_{n-5} + S_{n-6} = 0
\]


对应的特征方程为:
\[
r^6 - 6r^5 + 15r^4 - 20r^3 + 15r^2 - 6r + 1 = 0
\]
特征根为 $r = 1$,重数为 6。

为了方便求解, 将$S_n$ 的通解形式设为:
\[
S_n = A_0 +  A_1\binom{n}{1} +  A_2 \binom{n}{2} +  A_3 \binom{n}{3} +  A_4 \binom{n}{4} +  A_5 \binom{n}{5}
\]


有初始条件:
\begin{align*}
S_0 &= 0 \\
S_1 &= 1 \\
S_2 &= 17 \\
S_3 &= 98 \\
S_4 &= 354 \\
S_5 &= 979
\end{align*}

将这些值代入通解,建立方程组, 解得:
\[
\begin{cases}
A_0 = 0, \\
A_1 = 1, \\
A_2 = 15, \\
A_3 = 50, \\
A_4 = 60, \\
A_5 = 24.
\end{cases}
\]

最终求得, 四次方和的表达式为:
\[
\boxed{\sum_{k=1}^n k^4 = \binom{n}{1} + 15 \binom{n}{2} + 50 \binom{n}{3} + 60 \binom{n}{4} + 24 \binom{n}{5}}
\]
\section*{4.9}
\textbf{解答:}
\section*{(1) $f(n, k)$ 满足的一个线性常系数递推关系}
考虑第 $n$ 个位置是否被选取,分为以下两种情况:

\begin{enumerate}
    \item \textbf{第 $n$ 个数未被选取}:此时,需要从前 $n-1$ 个数中选取 $k$ 个不相邻的数,方案数为 $f(n-1, k)$;
    \item \textbf{第 $n$ 个数被选取}:由于相邻数不能被同时选取,因此第 $n-1$ 个数不能被选取。此时,需要从前 $n-2$ 个数中选取 $k-1$ 个不相邻的数,方案数为 $f(n-2, k-1)$。
\end{enumerate}

因此,$f(n, k)$ 满足递推关系:
\[
\boxed{f(n, k) = f(n-1, k) + f(n-2, k-1)}
\]

\section*{(2) $f(n, k)$ 的通项公式}

通过观察递推关系并分析小规模情况,猜测通项公式为:
\[
f(n, k) = \binom{n-k+1}{k}
\]

\subsection*{数学归纳法证明}

\paragraph{1. 基础情况:}
\begin{itemize}
    \item 当 $k = 0$ 时,$f(n, 0) = \binom{n - 0 + 1}{0} = \binom{n+1}{0} = 1$,显然成立;
    \item 当 $n = k$ 时,若 $n = k = 1$:
    \[
    f(1, 1) = \binom{1 - 1 + 1}{1} = \binom{1}{1} = 1
    \]
    同样成立。
\end{itemize}

\paragraph{2. 归纳假设:}
假设对于小于 $n$ 的所有整数,公式成立,即:
\[
f(n-1, k) = \binom{n-k}{k}, \quad f(n-2, k-1) = \binom{n-k-1}{k-1}
\]

\paragraph{3. 归纳推导:}
根据递推关系:
\[
f(n, k) = f(n-1, k) + f(n-2, k-1)
\]
代入归纳假设:
\[
f(n, k) = \binom{n-k}{k} + \binom{n-k-1}{k-1}
\]
利用组合数的性质(帕斯卡恒等式):
\[
\binom{n-k}{k} + \binom{n-k-1}{k-1} = \binom{n-k+1}{k}
\]
因此:
\[
f(n, k) = \binom{n-k+1}{k}
\]

\paragraph{结论:}
\[
\boxed{f(n, k) = \binom{n-k+1}{k}}
\]

\section*{(3) 利用$f(n, k)$ 求$g(n, k)$}
\subsection*{分析}

\begin{enumerate}
    \item \textbf{当 1 和 $n$ 同时被选取时}:
    
    \begin{itemize}
        \item 1 和 $n$ 被选中后,数 2 和 $n-1$ 不能被选取;
        \item 剩余可选的数是从 3 到 $n-2$ 的 $n-4$ 个数中选择 $k-2$ 个不相邻的数;
        \item 方案数为 $f(n-4, k-2)$。
    \end{itemize}
    
    \item \textbf{总方案数}:
    
    \[
    g(n, k) = f(n, k) - f(n-4, k-2)
    \]
\end{enumerate}

---

\subsection*{递推关系和公式}

由 $f(n, k) = \binom{n-k+1}{k}$,可以写出:
\[
g(n, k) = \binom{n-k+1}{k} - \binom{(n-4)-(k-2)+1}{k-2}
\]

化简后最终得到:
\[
\boxed{g(n, k) = \binom{n-k+1}{k} - \binom{n-k-1}{k-2}}
\]



\section*{4.10}
\textbf{解答:}
\section*{(1) 所有可能的铺砖方案数}

设 $T(n)$ 表示长度为 $n$ 的路径的所有铺砖方案数。

\begin{enumerate}
    \item **当铺设长度为 1 的砖块(类型 A 或类型 B)时:**
    \begin{itemize}
        \item 每次可以选择铺设类型 A 或类型 B 的砖块;
        \item 剩余的长度为 $n-1$ 的路径可以有 $T(n-1)$ 种铺法;
        \item 方案数贡献:$2 \times T(n-1)$。
    \end{itemize}
    \item **当铺设长度为 2 的砖块(类型 C)时:**
    \begin{itemize}
        \item 只能铺设一种类型的砖块(类型 C);
        \item 剩余的长度为 $n-2$ 的路径可以有 $T(n-2)$ 种铺法;
        \item 方案数贡献:$1 \times T(n-2)$。
    \end{itemize}
\end{enumerate}

\textbf{递推关系:}
\[
T(n) = 2 \times T(n-1) + T(n-2)
\]

\subsection*{初始条件}
\[
T(0) = 1, \quad T(1) = 2
\]

特征方程:
\[
\quad r^2 - 2r - 1 = 0
\]

特征根:
\[
r = 1 + \sqrt{2}, \quad r = 1 - \sqrt{2}.
\]

通解:
\[
T(n) = A (1 + \sqrt{2})^n + B (1 - \sqrt{2})^n
\]

利用初始条件:
\begin{align*}
T(0) &= A + B = 1, \\
T(1) &= A (1 + \sqrt{2}) + B (1 - \sqrt{2}) = 2.
\end{align*}

解得:
\[
A = \frac{1 + \sqrt{2}}{2 \sqrt{2}}, \quad B = \frac{1 - \sqrt{2}}{-2 \sqrt{2}}.
\]

通项公式:
\[
\boxed{T(n) = \frac{(1 + \sqrt{2})^{n+1} + (1 - \sqrt{2})^{n+1}}{2 \sqrt{2}}}
\]

\section*{(2) 所有方案中使用的砖数总和}

\begin{itemize}
    \item 情况 1(类型 A):每个方案增加 1 块砖;
    \item 情况 2(两个类型 B):每个方案增加 2 块砖;
    \item 情况 3(类型 C + 两个类型 B):每个方案增加 3 块砖。
\end{itemize}

---

\subsection*{递推关系}

从长度为 $n-1$ 的方案延伸:
\begin{itemize}
    \item 类型 A 增加的总砖数:$S(n-1) + T(n-1) \times 1$;
    \item 两个类型 B 增加的总砖数:$S(n-1) + T(n-1) \times 2$。
\end{itemize}

从长度为 $n-2$ 的方案延伸:
\begin{itemize}
    \item 类型 C + 两个类型 B 增加的总砖数:$S(n-2) + T(n-2) \times 3$。
\end{itemize}

因此,总的砖数为:
\[
S(n) = [S(n-1) + T(n-1) \times 1] + [S(n-1) + T(n-1) \times 2] + [S(n-2) + T(n-2) \times 3]
\]

得到递推关系:
\[
\boxed{S(n) = 2S(n-1) + S(n-2) + 3T(n-1) + 3T(n-2)}
\]



\end{document}



