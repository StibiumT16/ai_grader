\documentclass[a4paper,12pt]{article}
\usepackage{CTEX}
\usepackage{geometry}
\usepackage{mathptmx}
\usepackage{amsmath}
\geometry{left=2.0cm, right=2.0cm, top=3.0cm, bottom=3.0cm}
\linespread{1.5}

\begin{document}
	
	\begin{center}
		{\large \textbf{组合数学第四章作业}}\\
	\end{center}
	
	\noindent
	\textbf{4.1}\\
	(1)证明:\\
	即证\[F_{2n}-3F_{2n-2}+F_{2n-4}=0 \tag{1.1} \]
	$F_{2n}=F_{2n-1}+F_{2n-2}=2F_{2n-2}+F_{2n-3}=3F_{2n-3}+2F_{2n-4}$\\
	$3F_{2n-2}=3F_{2n-3}+3F_{2n-4}$\\
	所以式(1.1)左边:$3F_{2n-3}+2F_{2n-4}-3F_{2n-3}-3F_{2n-4}+F_{2n-4}=0$\\
	原式得证\\
	(2)\\
	$G_0=F_0=0,G_1=F_2=1$。
	设$G_{n}$的母函数为$G(x)$,则\\
	$G(x)=G_{0}+G_{1}x+G_{2}x^{2}+G_{3}x^{3}+\dots $\\
	$3xG(x)=3G_{0}x+3G_{1}x^{2}+3G_{2}x^{3}+3G_{3}x^{4}+\dots $\\
	根据(1)中证明的序列关系,并将上面两式相减,得\\
	$(1-3x)G(x)=x+x^{2}G(x)$\\
	因此,序列$G_{n}$的母函数为\[
	G(x)=\frac{x}{1-3x-x^2}
	\]
	
	\noindent
	\textbf{4.2}\\
	由母函数可知递推关系为\[
		a_{n}-a_{n-1}+a_{n-2}=0
	\]
	$A(0)=1$\\
	$A'(x)=\frac{1-2x}{(1-x+x^2)^2},A'(0)=1$\\
	可得$a_{0}=1,a_{1}=1$
	\\
	
	\noindent
	\textbf{4.3}\\
	设满足的递推关系为$a_{n+2}+pa_{n+1}+qa_{n}=0$\\
	将题目给的$a_{n}$代入上面的递推关系式,得到\[
	3^{3n+2}c+(-1)^{n+2}d+(3^{n+1}c+(-1)^{n+1}d)p+(3^{n}c+(-1)^{n}d)q \tag{3.1}
	\]
	要使不含c和d,即令式(3.1)中c和d的系数为0,从而得到方程组:\\
	$$
	\begin{cases}
		9+3p+q=0\\
		1-p+q=0
	\end{cases}
	$$
	解得$p=-2,q=-3$,则所求递推关系为$a_{n+2}-2a_{n+1}-3a_{n}=0$
	\\

	\noindent
	\textbf{4.4}\\
	由递推关系和初值,可求出$a_2=8$,同时可得以下关系式\\
	$$
	\begin{cases}
		a_{n}-2a_{n-1}+a_{n-2}=5\\
		a_{n-1}-2a_{n-2}+a_{n-3}=5
	\end{cases}
	$$
	两式相减得:$a_{n}-3a_{n-1}+3a_{n-2}-a_{n-3}=0$
	此递推关系对应的特征多项式为:$C(x)=x^3-3x^2+3x-1=(x-1)^3$\\
	有1个3重根$\alpha=1$,$a_{n}$的通项表达式为:\[
		a_{n}=A_{1}+(n+1)A_{2}+\frac{(n+2)(n+1)}{2}A_{3} \tag{4.1}
	\]
	将$a_0=1,a_1=2,a_3=8$代入式(4.1)得
	$$
	\begin{cases}
		A_{1}+A_{2}+A_{3}=1\\
		A_{1}+2A_{2}+3A_{3}=2\\
		A_{1}+3A_{2}+6A_{3}=8
	\end{cases}
	$$
	解得$A_{1}=5,A_{2}=-9,A_{3}=5$,则\[
		a_n=\frac{5n^2-3n+2}{2}
	\] 
	\\
	
	\noindent
	\textbf{4.5}\\
	转换为A,B,C,D四个字母先可重复地组成n-2位字符串,有$4^{n-2}$种,再将子串AB插入到n-1个位置中,可得满足要求的字符串数目为:$(n-1)4^{n-2}$。
	\\
	
	\noindent
	\textbf{4.6}\\
	尝试移动计数得到结果:\\
	盘子数量 \quad 变种最小移动次数 \quad 经典移动次数\\
	1	\qquad\qquad\qquad\qquad 1	\qquad\qquad\qquad\qquad 1 \\
	2	\qquad\qquad\qquad\qquad 2	\qquad\qquad\qquad\qquad 3 \\
	3	\qquad\qquad\qquad\qquad 5	\qquad\qquad\qquad\qquad 7 \\
	4	\qquad\qquad\qquad\qquad 11	\qquad\qquad\qquad\quad 15 \\
	根据归纳法,可以发现变种问题的移动次数比经典的移动次数少$2^{n-2}$次\\
	故该变种汉诺塔问题的最小移动次数为$2^{n}-1-2^{n-2}=3\cdot2^{n-2}-1,n\geq2$
	\\
	
	\noindent
	\textbf{4.7}\\
	设$b_{n}$为最后两位字母不同的n位字符串数目,$c_{n}$为最后两位字母相同的n位字符串数目。\\
	$b_{n}$的最后一位有$(k-1)$种选择,因此$b_{n}=(k-1)a_{n-1}$\\
	$c_{n}$的最后一位有1种选择,其前$(n-1)$位必须是$b_{n-1}$,因此$c_{n}=b_{n-1}$\\
	可得,$a_{n}=b_{n}+c_{n}=(k-1)a_{n-1}+(k-1)a_{n-2}$。由题易知$a_{1}=k,a_{2}=k^2$\\
	递推关系的特征方程:$x^2-(k-1)x-(k-1)=0$\\
	k不为1时有2个不同的根,$\alpha_1$和$\alpha_2$。\\
	因此通项表达式为\[
		a_{n}=A_{1}\alpha_1^{n}+A_{2}\alpha_2^{n}
	\]
	代入$a_{1},a_{2}$,可得
	$$
	\begin{cases}
		A_{1}\alpha_1+A_{2}\alpha_2=k\\
		A_{1}\alpha_1^2+A_{2}\alpha_2^2=k^2
	\end{cases}
	$$
	求解上述方程组即可得到$a_{n}$关于k的通项公式。
	\\
	
	\noindent
	\textbf{4.8}\\
	令\[
		S_{n}=\sum_{k=1}^{n}{k^{4}}
	\]
	设\[
		S_{n}=A+Bn+\frac{Cn(n-1)}{2!}+\frac{Dn(n-1)(n-2)}{3!}+\frac{En(n-1)(n-2)(n-3)}{4!}
	\]
	由$S_{0}=0$,得$A=0$;\\
	由$S_{1}=1$,得$B=1$;\\
	由$S_{2}=17$,得$C=15$;\\
	由$S_{3}=0$,得$D=50$;\\
	由$S_{4}=0$,得$E=60$;\\
	可得\[
		S_{n}=n+\frac{15n(n-1)}{2}+\frac{25n(n-1)(n-2)}{3}+\frac{5n(n-1)(n-2)(n-3)}{2}
	\]
	化简得\[
		S_{n}=\frac{15n^4-40n^3+60n^2-29n}{6}
	\]
	\\
	
	\noindent
	\textbf{4.9}\\
	(1)\\
	易知$n\geq2k-1$\\
	①选取的k个数不包含第n个数,方案数相当于从前n-1个选k个,方案数为$f(n-1,k)$;\\
	②选取的k个数包含第n个数,则第n-1个数不能选,另外需要从前n-2个选k-1个,方案数为$f(n-2,k-1)$。\\
	则递推关系为:$f(n,k)=f(n-1,k)+f(n-2,k-1)$。\\
	(2)\\
	①n=2时\\
	$f(2,0)=f(0,-1)+f(1,0)=0+1=1$\\
	$f(2,1)=f(0,0)+f(1,1)=1+1=2$\\
	②n=3时\\
	$f(3,0)=f(1,-1)+f(2,0)=0+1=1$\\
	$f(3,1)=f(1,0)+f(2,1)=1+2=3$\\
	$f(3,2)=f(1,1)+f(2,2)=1+0=1$\\
	③n=4时\\
	$f(4,0)=f(2,-1)+f(3,0)=0+1=1$\\
	$f(4,1)=f(2,0)+f(3,1)=1+3=3$\\
	$f(4,2)=f(2,1)+f(3,2)=2+1=3$\\
	$f(4,3)=f(2,2)+f(3,3)=0+0=0$\\
	由数学归纳法,通项表示为:\[
		f(n,k)=\binom{n-k+1}{k}
	\]
	(3)\\
	①选取的k个数不包含第n个数,方案数相当于从前n-1个选k个,方案数为$f(n-1,k)$;\\
	②选取的k个数包含第n个数,则第n-1个数和第1个数不能选,另外需要从剩下的n-3个选k-1个,方案数为$f(n-3,k-1)$。\\
	则\[
		g(n,k)=f(n-1,k)+f(n-3,k-1)=\binom{n-1-k+1}{k}+\binom{n-3-(k-1)+1}{k-1}=\frac{n}{n-k}\binom{n-k}{k}
	\]
	\\
	
	\noindent
	\textbf{4.10}\\
	在此不考虑三角形砖的放置方向,那么小三角形砖必然成对出现,大三角形砖必然和两个小三角形砖同时出现。这样使问题转换回长度分别为1和2的矩形砖铺路,其中长度为1的砖有2种。\\
	(1)\\
	设铺长度为n的路,方案数为$a_{n}$\\
	①最后1块是长度为1的砖,最后一块有2种选择,那么方案数为$2a_{n-1}$\\
	②最后1块是长度为2的砖,那么方案数为$a_{n-2}$\\
	可以得到递推关系:$a_{n}=2a_{n-1}+a_{n-2}$,\qquad 有$a_1=2,a_2=5$\\
	对应的特征方程为:$x^2-2x-1=0$,有两个不同的根,$\alpha_1=1+\sqrt{2},\alpha_2=1-\sqrt{2}$\\
	可以得到通项表达式为;$a_n=A(1+\sqrt{2})^n+B(1-\sqrt{2})^n$,代入$a_1$和$a_2$可得\\
	$$
	\begin{cases}
		A(1+\sqrt{2})+B(1-\sqrt{2})=2\\
		A(3+2\sqrt{2})+B(3-2\sqrt{2})=5
	\end{cases}
	$$
	求得$A=\frac{\sqrt{2}-2}{4},B=-\frac{10+9\sqrt{2}}{4}$\\
	则方案数有\[
		a_n=\frac{\sqrt{2}-2}{4}(1+\sqrt{2})^n-\frac{10+9\sqrt{2}}{4}(1-\sqrt{2})^n
	\]
	(2)\\
	设$b_n$为所有方案的总砖数。\\
	①最后1块是长度为1的砖,那么最后1块是正方形砖和2个小三角形砖组合的方案数各为$a_{n-1}$,则总砖数为$b_{n-1}+3a_{n-1}$\\
	②最后1块是长度为2的砖,方案数为$a_{n-2}$,则总砖数为$b_{n-2}+3a_{n-2}$\\
	可得递推关系:$b_{n}=b_{n-1}+b_{n-2}+3a_{n-1}+3a_{n-2}$,其中$b_1=3,b_2=15$\\
	$b_n$满足的特征多项式为:$x^2-x-1=0$\\
	有两个不同的根:$\beta_1=\frac{1+\sqrt{5}}{2},\beta_2=\frac{1-\sqrt{5}}{2}$\\
	通过特解法求$b_n$的通项,即为总砖数。
	\\
	
	\noindent
	\textbf{4.11——进阶}\\
	(1)\\
	①对于放球问题$F(X)=\frac{1}{(1-x)(1-x^2)\dots(1-x^m)}$\\
	②对于无序整数拆分$G(x)=(1+x+x^2+\dots)(1+x^2+x^4+\dots)\dots(1+x^m+x^2m+\dots)=\frac{1}{(1-x)(1-x^2)\dots(1-x^m)}$
	因此$f(n,m)=g(n,m)$\\
	(2)\\
	可以把盒子中的球穿成串(Ferrers图像),这样最多可能有n层,每层最多可能是m个球,将所有盒子每层的球数相加,得到的总数就是n。这每层的球数就相当于整数拆分中的组成数。\\
	
	\noindent
	\textbf{4.12——进阶}\\
	设非齐次递推关系特解为$\alpha=c \cdot sin\frac{n\pi}{2}$\\
	代入原递推关系式得$c \cdot sin\frac{n\pi}{2}=-3c \cdot cos\frac{n\pi}{2}+2c \cdot sin\frac{n\pi}{2}+3sin\frac{n\pi}{2}$\\
	解得$c=\frac{3sin\frac{n\pi}{2}}{3cos\frac{n\pi}{2}-sin\frac{n\pi}{2}}$\\
	满足齐次递推关系的通解$b_{n}-3b_{n-1}+2b_{n-2}=0$\\
	特征方程为$x^2-3x+2=0$,$x_1=1,x_2=2$\\
	则$a_n=k_1 + k_2 2^n + c$,将$a_0,a_1$代入,得\\
	$$
	\begin{cases}
		k_1 + k_2=5\\
		k_1 + 2k_2 -3=3
	\end{cases}
	$$
	解得$k_1=4,k_2=1$,则\[
		a_n=4+2^n+\frac{3sin\frac{n\pi}{2}}{3cos\frac{n\pi}{2}-sin\frac{n\pi}{2}}=2^n+\frac{12cos\frac{n\pi}{2}-sin\frac{n\pi}{2}}{3cos\frac{n\pi}{2}-sin\frac{n\pi}{2}}
	\]
	
	
\end{document}