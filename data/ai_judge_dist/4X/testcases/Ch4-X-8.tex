\documentclass{article}
\usepackage{graphicx} % Required for inserting images
\usepackage[utf8]{ctex}
\usepackage{titlesec}  % For customizing section titles
\usepackage{amsmath,amssymb}
\usepackage{fontspec} % Allows font customization in XeLaTeX

\usepackage[a4paper, margin=1in]{geometry} % Adjust margins here (1 inch in this case)

\title{组合数学HW4}

\begin{document}

\maketitle
\section*{4.1}

(1) 证明:

\[
F_n=F_{n-1}+F_{n-2}
\]

\[
\Rightarrow
G_n=F_{2n}=F_{2n-1}+F_{2n-2}=2F_{2n-2}+F_{2n-3}=3F_{2n-3}+2F_{2n-4}
\]

\[
\Rightarrow
\left\{
\begin{aligned}
G_{n-1}&=F_{2n-2}=F_{2n-3}+F_{2n-4}, \\
G_{n-2}&=F_{2n-4}
\end{aligned}
\right.
\]

\[
\Rightarrow
G_n=3F_{2n-3}+2F_{2n-4}=3(F_{2n-3}+F_{2n-4})-F_{2n-4}=3G_{n-1}-G_{n-2}
\]

\[
\Rightarrow
G_n-3G_{n-1}+G_{n-2}=0
\]


(2) 

根据(1):
\[
\sum_{n=2}^{\infty}(G_n-3G_{n-1}+G_{n-2})x^n=0
\]

又:
\[
\left\{
\begin{aligned}
G_0&=0 \\
G_1&=1 \\
\end{aligned}
\right.
\]

\[
\Rightarrow
\sum_{n=2}^\infty G_nx^n = G(x)-G_0-G_1x = G(x)-x
\]

\[
\Rightarrow
G(x)-x=3(G(x)-x)x-G(x)x^2
\]

\[
\Rightarrow
G(x)=\dfrac{x}{1-3x+x^2}
\]

\subsection*{4.2}

\[
(1-x+x^2)G(x) = 1
\]

\[
\Rightarrow
\sum_{n=0}^\infty{a_nx^n} - \sum_{n=0}^\infty{a_{n}x^{n+1}} + \sum_{n=0}^\infty{a_{n}x^{n+2}} = 1
\]

\[
\Rightarrow
\sum_{n=0}^\infty{a_nx^n} - \sum_{n=1}^\infty{a_{n-1}x^{n}} + \sum_{n=2}^\infty{a_{n-2}x^{n}} = 1
\]

\[
\Rightarrow
a_0+(a_1-a_0)x+\sum_{n=2}^\infty{(a_n-a_{n-1}+a_{n-2})x^n}=1
\]

\[
\Rightarrow
\left\{
\begin{aligned}
&a_0=1 \\
&a_1-a_0=0 \\
&a_n-a_{n-1}+a_{n-2}=0
\end{aligned}
\right.
\]

\[
\Rightarrow
\left\{
\begin{aligned}
&\text{递推式:}a_n=a_{n-1}-a_{n-2} \\
&\text{初值:}a_0 = a_1 = 1
\end{aligned}
\right.
\]

\subsection*{4.3}

特征方程的根为:$3,-1$,据此可得不含$c,d$的线性常系数递推关系:$a_n-2a_{n-1}-3a_{n-2}=0$,$a_0=c+d,a_1=3c-d$

\subsection*{4.4}

\[
\left\{
\begin{aligned}
&a_n-2a_{n-1}+a_{n-2}=5 \\
&a_{n+1}-2a_n+a_{n-1}=5
\end{aligned}
\right.
\]

\[
\Rightarrow
\left\{
a_{n+1}-3a_n+3a_{n-1}-a_{n-2}=0
\right.
\]

特征方程为

\[
x^3-3x^2+3x-1=0
\]

有 3 重根 $\alpha=1$,故通解为 $a_n=(A+Bn+Cn^2)1^n=A+Bn+Cn^2$

由
\[
\left\{
\begin{aligned}
&a_0=A=1 \\
&a_1=A+B+C=2 \\
&a_2=A+2B+4C=8
\end{aligned}
\right.
\]

解得 $A=1,B=-\dfrac{3}{2},C=\dfrac{5}{2}$,故 $a_n=1-\dfrac{3}{2}n+\dfrac{5}{2}n^2$

\subsection*{4.5}

设 $a_n$ 为满足要求的 $n$ 位字符串数目。

考虑最后一次出现 $AB$ 的位置:设 $AB$ 的最后一个字母 $B$ 在第 $k$ 位,则 $k$ 从 2 到 $n$ 遍历。

当 $k=n$ 时,前 $n-2$ 位可以是任意字母,共 $4^{n-2}$ 种。

当 $k<n$ 时,第 $k+1$ 到第 $n$ 位不能出现 $AB$,设这样的字符串数为 $b_{n-k}$。

则有:$a_n=4^{n-2}+\sum_{k=2}^{n-1}4^{k-2}b_{n-k}$

对于不含 $AB$ 的字符串,考虑每个位置:若前一个字母不是 $A$,则当前位置可以是任意字母;若前一个字母是 $A$,则当前位置不能是 $B$。

设 $b_n$ 为长度为 $n$ 的不含 $AB$ 的字符串数,$c_n$ 为长度为 $n$ 且末尾是 $A$ 的不含 $AB$ 的字符串数。则:

$b_n=4b_{n-1}-c_{n-1}$(总数减去末尾是 $A$ 的情况下只能接 3 个字母)

$c_n=b_{n-1}$(去掉末尾的 $A$ 后一定是不含 $AB$ 的字符串)

初值:$b_1=4,c_1=1$

解得:$b_n=3*4^{n-1}+1$

代回原式:$a_n=4^{n-2}+\sum_{k=2}^{n-1}4^{k-2}(3*4^{n-k-1}+1)$

$=4^{n-2}+3\sum_{k=2}^{n-1}4^{n-3}+\sum_{k=2}^{n-1}4^{k-2}$

$=4^{n-2}+3(n-2)4^{n-3}+\dfrac{4^{n-2}-1}{3}$

$=4^{n-2}(1+\dfrac{3(n-2)}{4}+\dfrac{1}{3})-\dfrac{1}{3}$

$=\dfrac{4^{n-2}(3n+1)-1}{3}$



\subsection*{4.6}

设 $f(n)$ 为将 $n$ 个圆盘按要求移动所需的最小步数。

在移动最大的圆盘之前,需要将上面 $n-1$ 个圆盘移到某根柱子上(不能是目标柱)。移动完最大圆盘后,再将这 $n-1$ 个圆盘按要求分配到 $B,C$ 两根柱子上。

记 $g(n)$ 为将 $n$ 个圆盘按大小顺序移到某根柱子上所需的最小步数,显然 $g(n)=2^n-1$。

则有递推关系:
\[
f(n) = g(n-1) + 1 + f(n-1) = 2^{n-1} + f(n-1)
\]

初值 $f(1)=1$

解得:$f(n)=2^n-1+2^{n-1}-1+2^{n-2}-1+...+2^1-1+1-1=3*2^{n-1}-1$

因此,将 $n$ 个圆盘按要求移动所需的最小步数为 $3*2^{n-1}-1$。


\subsection*{4.7}

设 $a_n$ 为长度为 $n$ 的满足条件的字符串数,$b_n$ 为长度为 $n$ 且末尾两个字母相同的满足条件的字符串数。

对于长度为 $n$ 的字符串:
\begin{itemize}
    \item 若末尾两个字母不同,则可以接任意字母
    \item 若末尾两个字母相同,则不能再接相同的字母
\end{itemize}

因此有递推关系:
\[
\begin{cases}
a_n = (a_{n-1}-b_{n-1})*k + b_{n-1}*(k-1) = ka_{n-1}-b_{n-1} \\
b_n = (a_{n-1}-b_{n-1})*1 = a_{n-1}-b_{n-1}
\end{cases}
\]

初值:$a_1=k,b_1=0,a_2=k^2,b_2=k$

解得:$b_n=a_{n-1}-b_{n-1}$,代入第一个式子:
\[
a_n = ka_{n-1}-(a_{n-1}-b_{n-1}) = (k-1)a_{n-1}+b_{n-1}
\]

又 $b_n=a_{n-1}-b_{n-1}$,消去 $b_{n-1}$:
\[
a_n = (k-1)a_{n-1}+a_{n-2}-b_{n-2} = (k-1)a_{n-1}+a_{n-2}-(a_{n-2}-b_{n-2})
\]

即 $a_n = (k-1)a_{n-1}+b_{n-1}$

特征方程为 $x^2-(k-1)x-1=0$

解得:$x=\dfrac{k-1\pm\sqrt{(k-1)^2+4}}{2}$

\[
\Rightarrow
a_n = \dfrac{k(\dfrac{k-1+\sqrt{(k-1)^2+4}}{2})^n-k(\dfrac{k-1-\sqrt{(k-1)^2+4}}{2})^n}{\sqrt{(k-1)^2+4}}
\]


\subsection*{4.8}

设 $S_n=\sum_{k=1}^n{k^4}$,$S_{n-1}=\sum_{k=1}^{n-1}{k^4}$

则 $S_n-S_{n-1}=n^4$

设 $S_n=an^5+bn^4+cn^3+dn^2+en+f$

代入上式得:$a(n^5-(n-1)^5)+b(n^4-(n-1)^4)+c(n^3-(n-1)^3)+d(n^2-(n-1)^2)+e(n-(n-1))=n^4$

展开得:$a(5n^4-10n^3+10n^2-5n+1)+b(4n^3-6n^2+4n-1)+c(3n^2-3n+1)+d(2n-1)+e=n^4$

比较系数得:
\[
\begin{cases}
5a=1 \\
-10a+4b=0 \\
10a-6b+3c=0 \\
-5a+4b-3c+2d=0 \\
a-b+c-d+e=0
\end{cases}
\]

解得:$a=\dfrac{1}{5},b=\dfrac{1}{2},c=\dfrac{1}{3},d=\dfrac{1}{4},e=0$

故 $S_n=\dfrac{n^5}{5}+\dfrac{n^4}{2}+\dfrac{n^3}{3}-\dfrac{n^2}{4}$

\subsection*{4.9}

(1) 考虑 $f(n,k)$ 的递推关系:

若不选择 $n$,则问题转化为从 $1$ 到 $n-1$ 中选取 $k$ 个不相邻的数,方案数为 $f(n-1,k)$;

若选择 $n$,则不能选择 $n-1$,问题转化为从 $1$ 到 $n-2$ 中选取 $k-1$ 个不相邻的数,方案数为 $f(n-2,k-1)$。

因此有递推关系:$f(n,k)=f(n-1,k)+f(n-2,k-1)$

(2) 猜想:$f(n,k)=\binom{n-k+1}{k}$

当 $k=1$ 时,$f(n,1)=n=\binom{n}{1}$,成立。

当 $k=0$ 时,$f(n,0)=1=\binom{n+1}{0}$,成立。

假设对于所有小于 $n$ 的正整数和小于等于 $k$ 的非负整数,结论成立。

考虑 $f(n,k)$:

$f(n,k)=f(n-1,k)+f(n-2,k-1)$

$=\binom{n-k}{k}+\binom{n-k-1}{k-1}$

$=\binom{n-k+1}{k}$

归纳得证。

(3) 考虑 $g(n,k)$:

若不选择 $n$,则问题转化为从 $1$ 到 $n-1$ 中选取 $k$ 个不相邻的数且首尾不能同时选择,方案数为 $f(n-1,k)$;

若选择 $n$,则不能选择 $1$ 和 $n-1$,问题转化为从 $2$ 到 $n-2$ 中选取 $k-1$ 个不相邻的数,方案数为 $f(n-3,k-1)$;

因此:$g(n,k)=f(n-1,k)+f(n-3,k-1)$

$=\binom{n-k}{k}+\binom{n-k-1}{k-1}$

\subsection*{4.10}

设 $f(n)$ 为铺设 $1 \times n$ 的路径的方案数。

考虑最左边的铺设方式:
\begin{itemize}
\item 若使用 $1 \times 1$ 的方砖,则剩余部分为 $f(n-1)$;
\item 若使用两个直角边长为 $1$ 的等腰直角三角形砖配对,剩余部分为 $f(n-1)$;
\item 若使用两个斜边长为 $2$ 的等腰直角三角形砖配对,占据长度为 $2$,剩余部分为 $f(n-2)$;
\item 若使用一个斜边长为 $2$ 的等腰直角三角形砖和两个直角边长为 $1$ 的等腰直角三角形砖组合,占据长度为 $2$,剩余部分为 $f(n-2)$。
\end{itemize}

因此有递推关系:$f(n)=2f(n-1)+2f(n-2)$

初始值:$f(0)=1$,$f(1)=2$

解得特征方程:$x^2-2x-2=0$

特征根:$x_1=1+\sqrt{3}$,$x_2=1-\sqrt{3}$

通解:$f(n)=c_1(1+\sqrt{3})^n+c_2(1-\sqrt{3})^n$

代入初始值解得:$c_1=\frac{1}{2}+\frac{\sqrt{3}}{6}$,$c_2=\frac{1}{2}-\frac{\sqrt{3}}{6}$

故答案为 $f(n)=(\frac{1}{2}+\frac{\sqrt{3}}{6})(1+\sqrt{3})^n+(\frac{1}{2}-\frac{\sqrt{3}}{6})(1-\sqrt{3})^n$

(2) 设 $g(n)$ 为所有方案中使用的砖数总和。

考虑最左边的铺设方式:
\begin{itemize}
\item 使用 $1 \times 1$ 的方砖时,贡献为 $1 \cdot f(n-1)$;
\item 使用两个直角边长为 $1$ 的等腰直角三角形砖配对时,贡献为 $2 \cdot f(n-1)$;
\item 使用两个斜边长为 $2$ 的等腰直角三角形砖配对时,贡献为 $2 \cdot f(n-2)$;
\item 使用一个斜边长为 $2$ 的等腰直角三角形砖和两个直角边长为 $1$ 的等腰直角三角形砖组合时,贡献为 $3 \cdot f(n-2)$。
\end{itemize}

因此:$g(n)=3f(n-1)+5f(n-2)+g(n-1)+g(n-2)$

初始值:$g(0)=0$,$g(1)=3$

通解为 $g(n)=n[(2+\sqrt{3})(1+\sqrt{3})^n+(2-\sqrt{3})(1-\sqrt{3})^n]$

\end{document}

