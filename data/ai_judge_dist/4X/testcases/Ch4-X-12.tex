% Homework template for Inference and Information
% UPDATE: September 26, 2017 by Xiangxiang
\documentclass[a4paper]{article}
\usepackage{ctex}
\ctexset{
proofname = \heiti{证明}
}
\usepackage{amsmath, amssymb, amsthm}
% amsmath: equation*, amssymb: mathbb, amsthm: proof
\usepackage{moreenum}
\usepackage{mathtools}
\usepackage{url}
\usepackage{bm}
\usepackage{enumitem}
\usepackage{graphicx}
\usepackage{subcaption}
\usepackage{booktabs} % toprule
\usepackage[mathcal]{eucal}
\usepackage[thehwcnt = 4]{iidef}

\thecourseinstitute{清华大学}
\thecoursename{组合数学}
\theterm{2024年秋季学期}
\hwname{CH4 基础}
\slname{\heiti{解}}


\begin{document}
\courseheader
\name{}

\begin{enumerate}[]
  \setlength{\itemsep}{3\parskip}
  
\item
\begin{solution}
    1) \begin{align*}
        G_n-3G_{n-1}+G_{n-2}&=F_{2n}-3F_{2n-2}+F_{2n-4}= F_{2n-1} -2(F_{2n-3}+F_{2n-4}) +F_{2n-4} \\
        &= F_{2n-1} -2F_{2n-3} -F_{2n-4} = F_{2n-2} - F_{2n-3} - F_{2n-4} = 0 \\
    \end{align*}

    2) 特征方程为 $C(m) = m^2-3m+1 =0$,解得$m_1=\frac{3+\sqrt{5}}{2},m2=\frac{3-\sqrt{5}}{2}$,所以
    \begin{equation*}
        G(x) = \frac{A}{1-m_1x} + \frac{B}{1-m_2x}
    \end{equation*}
    结合$G_0=F_0=0,G_1=F_2=1$,可解得$A=\frac{\sqrt{5}}{5},B=-\frac{\sqrt{5}}{5}$,所以
    \begin{equation*}
        G(x)=\frac{\frac{\sqrt{5}}{5}}{1-\frac{3+\sqrt{5}}{2}x} -  \frac{\frac{\sqrt{5}}{5}}{1-\frac{3-\sqrt{5}}{2}x}
    \end{equation*}
\end{solution}


\item
\begin{solution}
\begin{equation*}
    G(x)=\frac{1}{1-x+x^2}
\end{equation*}
所以可得特征多项式为 $C(m)=m^2-m+1$,所以递推关系为 $a_n-a_{n-1}+a_{n-2}=0$。根据长除法,可得$a_0=1,a_1=1$
\end{solution}


\item 
\begin{solution}
    可以知道其特征方程的解为3和-1,于是构造特征方程
    \begin{equation*}
        C(m)=(m-3)(m+1)= m^2-2m-3
    \end{equation*}
    所以可得线性常系数递推关系为 $a_n-2a_{n-1}-3a_{n-2}=0$
\end{solution}

\item 
\begin{solution}
    差分可得
    $a_n-3a_{n-1}+3a_{n-2}-a_{n-3}=0$。
    特征方程$C(m)=m^3-3m^2+3m-1=(m-1)^3=0$,所以$a_n=An^2+Bn+C$,带入$a_0=1,a_2=2,a_3=8$,可得$A=\frac{5}{2},B=-\frac{3}{2},C=1$,所以
    \begin{equation*}
        a_n=\frac{5}{2}n^2-\frac{3}{2}n+1
    \end{equation*}
\end{solution}

\item 
\begin{solution}
    令$f_n$表示长度为n的不包含AB的串个数,$g_n$表示长度为n不包含AB,且不以A结尾的串个数。考虑最后一位如果不是B,那么前面n位只要不出现AB即可。如果最后一位是B,那么要求倒数第二位不是A,也就是$g_{n-1}$,所以得到
    \begin{equation*}
        f_n= 3 f_{n-1} + g_{n-1}
    \end{equation*}
    另外对于$g_n$,如果最后一位不是B,则前面n-1位不出现AB即可;如果最后一位是B,那么倒数第二位不能是A,所以得到
    \begin{equation*}
        g_n = 2 f_{n-1} + g_{n-1}
    \end{equation*}
    联立方程可以消去g,得到
    \begin{equation*}
        f_n -4 f_{n-1} +f_{n-2} = 0
    \end{equation*}
    $f_1=4,f_2=15,f_0=1$
    特征方程 $C(m) = m^2 -4m+1 =0$ 解得$m_1 = 2+\sqrt{3},m_2 = 2-\sqrt{3}$,
    所以$f_n = A (2+\sqrt{3})^n + B(2-\sqrt{3})^n$,带入$f_0,f_1$可得
    \begin{equation*}
        f_n = \frac{\sqrt{3}}{6} ((2+\sqrt{3})^{n+1} - (2-\sqrt{3})^{n+1}) 
    \end{equation*}
    所以最后答案为
    \begin{equation*}
        a_n = 4^n - f_n = 4^n - \frac{\sqrt{3}}{6} ((2+\sqrt{3})^{n+1} - (2-\sqrt{3})^{n+1}) 
    \end{equation*}
\end{solution}

\item 
\begin{solution}
    令$a_n$表示新规则下的最小移动次数,$b_n$表示将n个盘移动到另一根柱子的最小移动次数,也就是原始汉诺塔的定义,所以$b_n=2^n-1$。考虑要想完成新规则下的移动,可以分为以下几个步骤,假设n为偶数:
    \begin{enumerate}
        \item 将n-1个盘先移动到C柱上,代价为$b_{n-1}$
        \item 将A柱上剩下的第n个盘移动到B柱上,代价为1
        \item 将C柱上的n-3个盘移动到A柱上,代价为 $b_{n-3}$
        \item 将C柱上第n-2个盘移动到B柱上,代价为1
    \end{enumerate}
    这时问题变成一个子问题,C柱上有第n-1个盘,B柱上有n和n-2号盘,移动A柱上剩下的n-3个盘需要$a_{n-3}$的代价即可完成。同时n为奇数时只需要将B,C柱调换即可,结果是一样的,所以得到递推式
    \begin{equation*}
        a_n = a_{n-3} + 2^{n-1} + 2^{n-3} = a_{n-3} + 5 \cdot 2^{n-3}
    \end{equation*}
    $a_1=1,a_2=2,a_3=5 $,可得$a_0=0$,特征方程 $C(m)=m^3-1=0$,解得$m_1=1,m_2=\frac{-1-i\sqrt{3}}{2},m_3=\frac{-1+i\sqrt{3}}{2}$
    特解的形式为$D\cdot 2^n $,带入递推式解得$D=\frac{5}{7}$,所以解的形式为$a_n = A+Bcos \frac{2}{3} n \pi + C sin \frac{2}{3} n \pi + \frac{5}{7} 2^n$,带入$a_0,a_1,a_2$解得
    \begin{equation*}
        a_n = -\frac{2}{3} -\frac{1}{21} cos \frac{2}{3} n \pi + \frac{\sqrt{3}}{7} sin \frac{2}{3} n \pi + \frac{5}{7} 2^n
    \end{equation*}
\end{solution}

\item 
\begin{solution}
    令$a_n$表示长度为n且满足条件的字符串中末尾两个字符不同的个数,$b_n$表示长度为n且满足条件的字符串中末尾两个字符相同的个数,那么有递推关系
    \begin{align*}
        a_n &= (a_{n-1}+b_{n-1}) * (k-1) \\
        b_n &= a_{n-1} \\
    \end{align*}
    消去b可得, 
    \begin{equation*}
        a_n = (a_{n-1} + a_{n-2}) * (k-1)
    \end{equation*}
    所以特征方程为
    \begin{equation*}
        C(m) = m^2-(k-1)m-(k-1) = 0
    \end{equation*}
    解为$m_1 = \frac{k-1+\sqrt{(k-1)^2+4(k-1)}}{2}, m_2 = \frac{k-1-\sqrt{(k-1)^2+4(k-1)}}{2}$
    代入$a_1=k,a_2=k(k-1)$,得$a_0=0$,所以得到通项公式为
    \begin{equation*}
        a_n = \frac{k}{\sqrt{k^2+2k-3}} [ (\frac{k-1+\sqrt{k^2+2k-3}}{2}) ^ n - (\frac{k-1-\sqrt{k^2+2k-3}}{2}) ^n ]
    \end{equation*}
    注意到原问题答案$f_n=a_n+b_n$,可得$f_n = a_{n+1}/(k-1)$,所以答案为
    \begin{equation*}
        f_n = \frac{k}{(k-1)\sqrt{k^2+2k-3}} [ (\frac{k-1+\sqrt{k^2+2k-3}}{2}) ^ {n+1} - (\frac{k-1-\sqrt{k^2+2k-3}}{2}) ^{n+1} ]
    \end{equation*}
    
\end{solution}

\item 
\begin{solution}
    令$S_n= \sum _{k=1} ^ {n} k^4$ ,可以得到其递推关系式为
    \begin{equation*}
        S_{n} - 6S_{n-1} + 15S_{n-2} - 20S_{n-3} + 15S_{n-4} - 6S_{n-5} + S_{n-6} = 0
    \end{equation*}
    其特征方程为$C(m) = (m-1)^6 = 0$
    因为$S_0=0$,不妨设
    \begin{equation*}
        S_n = A_1 \binom{n}{1} + A_2 \binom{n}{2} + A_3 \binom{n}{3} + A_4 \binom{n}{4} + A_5 \binom{n}{5}
    \end{equation*}
    带入$S_1 = 1$,得 $A_1 = 1$; 带入$S_2 = 17$,得 $A_2 = 15$; 带入$S_3 = 98$,得 $A_3 = 50$; 带入$S_4 = 354$,得 $A_4 = 60$; 
    带入$S_5 = 979$,得 $A_5 = 24$,所以可得
    \begin{equation*}
        S_n = \binom{n}{1} + 15 \binom{n}{2} + 50 \binom{n}{3} + 60 \binom{n}{4} + 24 \binom{n}{5}
    \end{equation*}
\end{solution}

\item 
\begin{solution}
    1) 根据第n个数选不选,可以进行分类。如果选第n个数,那么就需要从n-2个数中选k-1个数,如果不选那么需要从n-1个数中选k个数,所以得到递推关系为
    \begin{equation*}
        f(n,k)=f(n-1,k)+f(n-2,k-1)
    \end{equation*}
    对于初始值,f(i,0)=1,f(i,1)=i

    2) $f(n,k) = \binom{n-k+1}{k}$,下面用数学归纳法证明:
    首先对于边界条件,当$k=0,f(n,0)=1$成立;当$k=1,f(n,1)=n$成立。

    假设当i<n,j<k时,有$f(i,j) = \binom{i-j+1}{j}$成立,则根据递推关系可得
    \begin{align*}
        f(n,k) &= f(n-1,k)+f(n-2,k-1) = \binom{n-k}{k} + \binom{n-k}{k-1} \\
        &= \binom{n-k+1}{k} \\
    \end{align*}
    所以原式得证。

    3) 考虑1,n这两个元素会不会出现在最终的选择,第一种情况两个都不选,那么结果为$f(n-2,k)$。另一种情况是选择1或者n,这两个选择对答案的贡献是一样的,假设选了1那么n和2都不能选,那么这种情况的总贡献为$2f(n-3,k-1)$,所以可得
    \begin{align*}
        g(n,k) &= f(n-2,k) + 2f(n-3,k-1) = f(n-1,k) + f(n-3,k-1) \\
        &= \binom{n-k}{k} + \binom{n-k-1}{k-1}  \\
    \end{align*}
\end{solution}

\item 
\begin{solution}
    1) 首先我们认为对于一块砖有三种不同的铺法,分别是用1*1的方砖,根据对角线的方向,用三角形的砖有两种方法。如果不考虑斜边为2的大三角形,那么每块砖其实是独立的。令$f_n$表示铺设1*n的地方案数,将大三角形考虑进去,可以发现大三角形可以连续占据至少2块地砖,且根据方向不同,存在两种不同的放置方法。那么对于第n格,考虑它前面一次放置的是大三角形还是其他的地砖分类讨论,可以得到递推关系:
    \begin{equation*}
        f_n = 2 \sum_{i=0} ^ {n-2} f_i + 3f_{n-1}
    \end{equation*}
    第一项是考虑大三角形地砖延续到哪,最远是从头开始都用大三角形来填。第二项则是用三种方法填第n块砖,不用大三角形。为了求解f,令$F_n=\sum_{i=1}^n f_n$,得到递推式
    \begin{equation*}
        F_n -4F{n-1} + F_{n-2} = 0
    \end{equation*}
    特征方程$C(m)=m^2-4m+1=0$,解得$m_1=2+\sqrt{3},m_2=2-\sqrt{3}$
    根据$F_1=4,F_2=15,F_0=1$,得到$F_n$表达式为
    \begin{equation*}
        F_n = \frac{1}{2\sqrt{3}} [ (2+\sqrt{3})^{n+1} - (2-\sqrt{3})^{n+1}]
    \end{equation*}
    所以 
    \begin{equation*}
        f_n = F_n-F_{n-1}=\frac{1}{2\sqrt{3}} [ (\sqrt{3}+1)(2+\sqrt{3})^{n} + (\sqrt{3}-1)(2-\sqrt{3})^{n}]
    \end{equation*}

    2) 需要考虑每种砖对于答案的贡献
    \begin{enumerate}
        \item 1*1的方砖:假设第i格用方砖,那么它出现在$f_{i-1} * f_{n-i}$种方案中,对答案的总贡献为$S_1 =\sum_{i=1}^{n} f_{i-1} * f_{n-i}$
        \item 小三角形:如果在第i格放置了一个小三角形,考虑到直角的位置,总共有四种小三角形,但是其共同的特点是放置完小三角形,必然有一边是填满的,另一边是一个缺口,可以放置小三角形或者大三角形。考虑一定会连续放置若干个大三角形后放置一个小三角形来填满,那么这个小三角形(可以在左边放大三角形)对答案的贡献为$\sum_{j=0}^{i-1} f_j*f_{n-i}$。对于可以在右边放大三角形的方案为$\sum_{j=i}^{n} f_{n-j}*f_{i-1}$,所以小三角形总的贡献为
        \begin{equation*}
            S_2 = 2 \sum_{i=1} ^ n (\sum_{j=0}^{i-1} f_j*f_{n-i} + \sum_{j=i}^{n} f_{n-j}*f_{i-1})
        \end{equation*}

        \item 大三角形:对于第i格和i-1格放置大三角形,其两边都没有填满,所以可以向两边扩展,其贡献为$\sum_{j=0} ^ {i-2} \sum_{k=i}^n f_j * f_{n-k}$,所以总贡献为
        \begin{equation*}
            S_3 = \sum_{i=1} ^ n \sum_{j=0} ^ {i-2} \sum_{k=i}^n f_j * f_{n-k}
        \end{equation*}
    \end{enumerate}
    所以总砖数为
    \begin{equation*}
        S = \sum_{i=1}^{n} ( f_{i-1} * f_{n-i} + 2(\sum_{j=0}^{i-1} f_j*f_{n-i} + \sum_{j=i}^{n} f_{n-j}*f_{i-1}) + \sum_{j=0} ^ {i-2} \sum_{k=i}^n f_j * f_{n-k})
    \end{equation*}
    其中
    \begin{equation*}
        f_n =\frac{1}{2\sqrt{3}} [ (\sqrt{3}+1)(2+\sqrt{3})^{n} + (\sqrt{3}-1)(2-\sqrt{3})^{n}]
    \end{equation*}
    
\end{solution}

\end{enumerate}
\end{document}
\begin{equation}
\end{equation}

%%% Local Variables:
%%% mode: late\rvx
%%% TeX-master: t
%%% End:
