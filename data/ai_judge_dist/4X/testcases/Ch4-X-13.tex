\documentclass[12pt]{article}
\usepackage{amsmath} % For mathematical symbols
\usepackage{amssymb} % For more math symbols
\usepackage{geometry} % For page layout
\usepackage{ctex} % 加载 ctex 宏包以支持中文
\geometry{a4paper, margin=1in}

\title{第四章习题(基本)}

\begin{document}

\maketitle

\section*{4.1}
设 \( G_n = F_{2n} \) (\( n \geq 0 \)),其中 \( F_n \) 是第 \( n \) 个 Fibonacci 数:
\begin{itemize}
    \item (1) 证明:\( G_n - 3G_{n-1} + G_{n-2} = 0 \) (\( n = 2, 3, 4, \dots \));
    \item (2) 求数列 \(\{G_n\}\) 的母函数。
\end{itemize}

\subsection*{解答}
\begin{itemize}
    \item (1) 代入\( G_n = F_{2n} \) (\( n \geq 0 \))
    \begin{align}
     G_n - 3G_{n-1} + G_{n-2} &= F_{2n} - 3F_{2n-2} + F_{2n-4}\\
     &=(F_{2n-1} + F_{2n-2}) - 3F_{2n-2} + F_{2n-4}\\
     &=(F_{2n-2} + F_{2n-3}) + F_{2n-2} - 3F_{2n-2} + F_{2n-4}\\
     &=F_{2n-3} - F_{2n-2} + F_{2n-4}\\
     &=0 \quad (n = 2, 3, 4, \dots) 
    \end{align}
    因此得证\( G_n - 3G_{n-1} + G_{n-2} = 0 \) \quad (\( n = 2, 3, 4, \dots \))。
    \item (2)由递推式\( G_n - 3G_{n-1} + G_{n-2} = 0\),令母函数\(A(x) = \sum_{n=0}^{\infty} G_nx^n\)
    \[[x^n]A(x)-[x^n](3xA(x))+[x^n](x^2A(x))=0\]
    \[A(x)-3xA(x)+x^2A(x)=G_0+(G_1-3G_0)x\]
    \[G_0 = F_0 = 0, G_1 = F_2 = 1\]
    \[A(x)-3xA(x)+x^2A(x)=x\]
    \[A(x) = \frac{x}{x^2-3x+1}\]
\end{itemize}

\section*{4.2}
已知数列 \(\{a_n\}\) 的母函数为:
\[
\frac{1}{1 - x + x^2}。
\]
求 \(\{a_n\}\) 满足的二阶齐次线性常系数递推式,并求 \(a_0, a_1\)。
\subsection*{解答}
\[G(x) = \frac{1}{1 - x + x^2} = \sum_{n=0}^{\infty} a_nx^n\]
\[(1 - x + x^2)G(x)=1\]
\[(1 - x + x^2) \sum_{n=0}^\infty a_n x^n = 1\]
\[
\left( \sum_{n=0}^\infty a_n x^n \right)
- \left( \sum_{n=0}^\infty a_n x^{n+1} \right)
+ \left( \sum_{n=0}^\infty a_n x^{n+2} \right) = 1
\]
\[
\sum_{n=0}^\infty a_n x^n - \sum_{n=1}^\infty a_{n-1} x^n + \sum_{n=2}^\infty a_{n-2} x^n = 1
\]
\[
a_0 + (a_1-a_0)x + \sum_{n=2}^\infty \left( a_n - a_{n-1} + a_{n-2} \right) x^n = 1
\]
对该恒等式,各次幂系数相等,则
\[a_0 =1, (a_1-a_0)=0,  a_n - a_{n-1} + a_{n-2} = 0\]
综上有递推式\[a_n - a_{n-1} + a_{n-2} = 0 \quad (n \geq 2)\]
初始值\[a_0 =1, a_1=1\]

\section*{4.3}
已知 \( a_n = c \cdot 3^n + d \cdot (-1)^n \) (\( n \geq 0 \)),其中 \( c, d \) 是常数,求 \(\{a_n\}\) 满足的一个不含 \( c, d \) 的线性常系数递推关系。
\subsection*{解答}
\[a_{n+1} = 3 c \cdot 3^n - d \cdot (-1)^n\]
\[a_{n+2} = 9 c \cdot 3^n + d \cdot (-1)^n\]
待定系数法设
\[Aa_{n+2} + Ba_{n+1}+ a_n=f(n)\]
\[9Ac \cdot 3^n+Ad \cdot (-1)^n+3Bc \cdot 3^n-Bd\cdot (-1)^n+c \cdot 3^n+d \cdot (-1)^n=f(n)\]
要求c,d系数为0,则
\[9A\cdot 3^n+3B\cdot 3^n+3^n=0\]
\[A\cdot (-1)^n-B\cdot (-1)^n+(-1)^n=0\]
解得
\[A=-\frac{1}{3}, B=\frac{2}{3}, f(n)=0\]
故有递推关系
\[a_n-2a_{n-1}-3a_{n-2}=0 \quad (n \geq 2)\]

\section*{4.4}
求解递推关系:
\[
\begin{cases}
    a_n - 2a_{n-1} + a_{n-2} = 5, & n \geq 2, \\
    a_0 = 1, \ a_1 = 2. &
\end{cases}
\]
\subsection*{解答}
\[a_n - 2a_{n-1} + a_{n-2} = 5, \quad n \geq 2\]
\[a_{n+1} - 2a_{n} + a_{n-1} = 5, \quad n \geq 2\]
两式相减,得
\[a_{n+1}- 3a_{n} + 3a_{n-1} - a_{n-2} = 0, \quad n \geq 2\]
设特征方程为
\[r^3 - 3r^2 + 3r - 1= 0\]
求解得到三重根$r=1$,则通解形式为
\[a_n=(A_2n^2+A_1n+A_0)\times1^n=A_2n^2+A_1n+A_0\]
代入\[
\begin{cases}
    a_n - 2a_{n-1} + a_{n-2} = 5, & n \geq 2, \\
    a_0 = 1, \ a_1 = 2. &
\end{cases}
\]
解得
\[A_2=\frac{5}{2}, A_1=-\frac{3}{2}, A_0=1\]
即\[a_n=\frac{5}{2}n^2-\frac{3}{2}n+1\]

\section*{4.5}
由 \( A, B, C, D \) 四个字母组成允许重复的 \( n \) 位字符串,其中子串 \( AB \) 至少出现一次,求满足要求的字符串数目。
\subsection*{解答}
所有长度为 \( n \) 的字符串总数为 \( 4^n \)。
通过计算不含子串 \( AB \) 的长度为 \( n \) 的字符串数目 \( f_n \),可以用总数减去 \( f_n \) 得到满足条件的字符串数目。

设所有不含字串 \( AB \) 的长度为\( n \)字符串中,以 \( A \)结尾的数目为\( a_n \),以其他字母结尾的数目为\( b_n \)
则可得递推式
\[a_n = a_{n-1}+b_{n-1}\]
\[b_n = 2a_{n-1}+3b_{n-1}\]
与初始值
\[a_1=1, b_1=3\]
对\( a_n \)和\( b_n \)分别构造母函数
\[A(x)=\sum_{n=1}^\infty a_nx^n\]
\[B(x)=\sum_{n=1}^\infty b_nx^n\]
由递推式可得
\[A(x)-xA(x)-xB(x)=x\]
\[B(x)-2xA(x)-3xB(x)=3x\]
解得
\[A(x)=\frac{x}{x^2-4x+1}\]
\[B(x)=\frac{x(3-x)}{x^2-4x+1}\]
两种情况加和
\[F(x) = A(x) + B(x) =\frac{4x - x^2}{x^2 - 4x + 1}\]
化简后由Maclaurin展开
\[F(x) = \sum_{n=0}^{\infty} \left[ \frac{(2 + \sqrt{3})^{n+1} }{2\sqrt{3}} - \frac{(2 - \sqrt{3})^{n+1}}{2\sqrt{3}} \right] x^n\]
因此所求通项
\[g(n)=4^n-f(n)=4^n-\left[ \frac{(2 + \sqrt{3})^{n+1} }{2\sqrt{3}} - \frac{(2 - \sqrt{3})^{n+1}}{2\sqrt{3}} \right] \quad (n\geq 1)\]


\section*{4.6}
考虑如下汉诺塔问题的变种:有 \( A, B, C \) 三根柱子,初始时 \( A \) 柱上有 \( n \) 个圆盘,按直径从小到大的顺序编号为 1 到 \( n \);最终目标是将所有偶数编号的盘套在 \( B \) 柱上、所有奇数编号的盘套在 \( C \) 柱上。移动圆盘时的规则不变,求所需的最小移动次数。
\subsection*{解答}
设本问题的移动次数为\(G_n\),已知经典汉诺塔问题的移动次数\[M_n = 2^n -1\]
移动第$n$个盘的步骤如下:(以$n$为奇数的情况为例,$n$为偶数的情况只需调换B、C柱即可得到。)

第一步,将$(n-1)$个盘移动到B上。

第二步,将第$n$个奇数盘移动到C上。

第三步,将B上的$(n-3)$个盘移动到A上,第$(n-2)$个奇数盘移动到C上。
注意,这一步留下了第$(n-1)$个偶数盘在B上没有移动。

第四步,将A上的$(n-3)$个盘,奇数移动到C上,偶数移动到B上。

由此可得递推关系
\[G_n=M_{n-1}+1+M_{n-3}+1+G_{n-3}=2^{n-1}+2^{n-3}+G_{n-3}\]
同理
\[G_{n-1}=2^{n-2}+2^{n-4}+G_{n-4}\]
即
\[2G_{n-1}=2^{n-1}+2^{n-3}+2G_{n-4}\]
联立两式得
\[G_n-2G_{n-1}-G_{n-3}+2G_{n-4}=0\]
解特征方程
\[x^4 - 2x^3 - x + 2 = 0\]
得特征根
\[x_1 = 1, x_2 = 2, x_3 = -\frac{1}{2} + \frac{\sqrt{3}}{2}i, x_4 = -\frac{1}{2} - \frac{\sqrt{3}}{2}i\]
则用待定系数法表示原数列通项为
\[
G_n = A\times1^n + B\times2^n + C\cos\frac{2n}{3}\pi + D\sin\frac{2n}{3}\pi
\]
枚举法计算初始值后代入
\[G_0=0, G_1=1, G_2=2, G_3=5\]
解得
\[
G_n = -\frac{2}{3} + \frac{5}{7}\times2^n - \frac{1}{21}\cos\frac{2n}{3}\pi + \frac{\sqrt{3}}{7}\sin\frac{2n}{3}\pi
\]


\section*{4.7}
使用 \( k \) 种字母组成长度为 \( n \) 的字符串,但不允许相同字母连续出现 3 次,求方案数。
\subsection*{解答}
设 \( a_n \) 表示满足条件的长度为 \( n \) 的字符串的数量。

其中\( b_n \) 表示以长度为 1 的连续相同字母结束的长度为 \( n \) 的字符串数量,
\( c_n \) 表示以长度为 2 的连续相同字母结束的长度为 \( n \) 的字符串数量。

有递推关系
\[
\begin{cases}
b_n=(k-1)b_{n-1}+(k-1)c_{n-1} \\
c_n=b_{n-1}
\end{cases}
\]
消元得
\[b_n=(k-1)b_{n-1}+(k-1)b_{n-2} \quad (n \geq 3)\]
解特征方程
\[x^2 - (k-1)x - (k-1) = 0\]
得特征根
\[
r_1 = \frac{k-1+\sqrt{k^2+2k-3}}{2}, \quad r_2 = \frac{k-1-\sqrt{k^2+2k-3}}{2}.
\]
枚举法计算初始值
\[b_1=k, b_2=k(k-1)\]
由待定系数法得
\[
b_n = \frac{k}{\sqrt{k^2+2k-3}} \left( \left( \frac{k-1+\sqrt{k^2+2k-3}}{2} \right)^n - \left( \frac{k-1-\sqrt{k^2+2k-3}}{2} \right)^n \right) \quad (n \geq 0)
\]
注意,此时要求$k\geq2$。则
\begin{align}
a_n&=b_n+c_n=b_n+b_{n-1}\\
&=\frac{k}{\sqrt{k^2+2k-3}} \left( \left( \frac{k-1+\sqrt{k^2+2k-3}}{2} \right)^n - \left( \frac{k-1-\sqrt{k^2+2k-3}}{2} \right)^n \right)  \\
&+ \frac{k}{\sqrt{k^2+2k-3}} \left( \left( \frac{k-1+\sqrt{k^2+2k-3}}{2} \right)^{n-1} - \left( \frac{k-1-\sqrt{k^2+2k-3}}{2} \right)^{n-1} \right)
\end{align}
当$k=1$时,易得
\[a_1=a_2=1, a_n=0 \quad (n\geq3)\]


\section*{4.8}
计算:
\[
\sum_{k=1}^n k^4。
\]
\subsection*{解答}
令$S_n=\sum_{k=1}^n k^4$
\[S_n-S_{n-1}=n^4\]
\[
(S_n - S_{n-1})-(S_{n-1} - S_{n-2}) = 4n^3 - 6n^2 + 4n - 1
\]
\[
(S_n - 2S_{n-1} + S_{n-2}) - (S_{n-1} - 2S_{n-2} + S_{n-3}) = 12n^2 - 24n + 14
\]
\[
(S_n - 3S_{n-1} + 3S_{n-2} - S_{n-3}) - (S_{n-1} - 3S_{n-2} + 3S_{n-3} - S_{n-4}) = 24n - 36
\]
\[
(S_n - 4S_{n-1} + 6S_{n-2} - 4S_{n-3} + S_{n-4}) - (S_{n-1} - 4S_{n-2} + 6S_{n-3} - 4S_{n-4} + S_{n-5}) = 24
\]
\[
(S_n - 5S_{n-1} + 10S_{n-2} - 10S_{n-3} + 5S_{n-4} - S_{n-5})-(S_{n-1} - 5S_{n-2} + 10S_{n-3} - 10S_{n-4} + 5S_{n-5} - S_{n-6}) = 0
\]
\[
S_n - 6S_{n-1} + 15S_{n-2} - 20S_{n-3} + 15S_{n-4} - 6S_{n-5} + S_{n-6} = 0
\]
解特征方程
\[x^6 - 6x^5 + 15x^4 -20x^3 + 15x^2 - 6x + 1= 0\]
得六重特征根
\[x=1\]
则由待定系数法得递推公式
\[
S_n = An^5 + Bn^4 + Cn^3 + Dn^2 + En + F
\]
枚举法代入初始值可解得
\[
S_n = \frac{1}{5}n^5 + \frac{1}{2}n^4 + \frac{1}{3}n^3 - \frac{1}{30}n
\]

\section*{4.9}
从 1 到 \( n \) 的正整数中选取 \( k \) 个不同且不相邻的数,方案数记为 \( f(n, k) \)。
\begin{itemize}
    \item (1) 求 \( f(n, k) \) 满足的一个线性常系数递推关系;
    \item (2) 用数学归纳法求 \( f(n, k) \) 的通项表示;
    \item (3) 若规定 1 与 \( n \) 是相邻的数,并在此前提下令从 1 到 \( n \) 的正整数中选取 \( k \) 个不同且不相邻的数的方案数为 \( g(n, k) \),利用 \( f(n, k) \) 求 \( g(n, k) \)。
\end{itemize}
\subsection*{解答}
\begin{itemize}
\item (1)
    分类讨论:若不选择 $n$,等效于从前 $(n-1)$ 个数中选 $k$ 个不相邻的数;若选择 $n$,则 $(n-1)$ 不可选,等效于从前 $(n-2)$ 个数中选 $(k-1)$ 个不相邻的数。
    
    因此有递推关系
    \[
    f(n, k) = f(n-1, k) + f(n-2, k-1)
    \]
\item (2)
    $f(n, k) = C^k_{n-k+1} \ (k \leq \frac{n+1}{2})$,证明如下:
    
    当 $n=1$ 时,$k$ 可以取0或1,初始值满足
    \[
    f(1, 0) = C^0_{1-0+1} = 1, \quad f(1, 1) = C^1_{1-1+1} = 1
    \]

    假设对$n \leq i-1$,满足 \[f(n, k) = C^k_{n-k+1} \ (k \leq \frac{i}{2})\]
    则对$n=i, \quad k \leq \frac{i}{2}$
    \begin{align}
    f(i, k) &= f(i-1, k) + f(i-2, k-1) \\
    &= C^k_{i-k} + C^(k-1)_{i-k} \\
    &= C^k_{i-k+1}
    \end{align}
    对$n=i, \quad k = \frac{i+1}{2}$的情况,有且仅有
    \begin{align}
    f(i,k) &= 1 \\
    & = C^{\frac{i+1}{2}}_{\frac{i+1}{2}} \\
    & = C^{k}_{i-k+1}
    \end{align}
    均满足$f(n, k) = C^k_{n-k+1} \ (k \leq \frac{n+1}{2})$
    
    由数学归纳法,得证原式。
\item (3)
    分类讨论:若不选择 $1$,等效于从 $2$ 到 $n$ 的 $(n-1)$ 个数中选 $k$ 个不相邻的数;若选择 $1$,则 $2$ 和 $n$ 不可选,等效于从 $3$ 到 $(n-1)$ 的 $(n-3)$ 个数中选 $(k-1)$ 个不相邻的数。
    因此有递推关系
    \begin{align}
    g(n,k) &= f(n-1,k)+f(n-3,k-1) \\
    &= C^k_{n-k} + C^{k-1}_{n-k-1}, \quad k \leq \frac{n}{2}
    \end{align}
\end{itemize}


\section*{4.10}
使用尺寸为 \( 1 \times 1 \) 的方砖、直角边长为 1 的等腰直角三角形砖,以及斜边长为 2 的等腰直角三角形砖,铺设 \( 1 \times n \) 的路径,求:
\begin{itemize}
    \item (1) 所有可能的铺砖方案数;
    \item (2) 每一种可能的铺砖方案中使用的砖数相加,得到的砖数的总和。
\end{itemize}
\subsection*{解答}
\begin{itemize}
    \item (1)
    设铺设 \( 1 \times n \) 的路径的方案数为$a_n$,对最后一块砖分类讨论:
    
    如果最后一块砖是 \( 1 \times 1 \) 的方砖,对应的方案数是$a_{n-1}$;

    如果最后一块砖是 直角边长为 1 的等腰直角三角形砖,有两种情况:
    
    如果倒数第二块砖也是 直角边长为 1 的等腰直角三角形砖,则对应的方案数是$2 \times a_{n-1}$,因为两个三角砖铺设的对角线有两种方向;

    如果倒数第二块砖是斜边长为 2 的等腰直角三角形砖,且连续铺了 $k$ 块斜边长为 2 的等腰直角三角形砖后变成与 直角边长为 1 的等腰直角三角形砖相接,对应的方案数是$2 \times a_{n-k-1}, \quad (1 \leq k \leq (n-1))$,三角砖同样有两种方向。

    因此有递推关系
    \begin{align}
    a_n &= a_{n-1} + 2a_{n-1} + \sum_{k=1}^{n-1} 2a_{n-k-1}\\
    &= a_{n-1} + \sum_{k=0}^{n-1} 2a_{k}
    \end{align}

    则\[a_{n-1} =  a_{n-2} + \sum_{k=0}^{n-2} 2a_{k}\]
    \[a_n-a_{n-1}=a_{n-1} -a_{n-2}+2a_{n-1}\]
    \[a_n-4a_{n-1}+a_{n-2}=0\]

    解特征方程
    \[x^2 - 4x + 1= 0\]
    得特征根
    \[r_1 = 2+\sqrt{3}, r_2 = 2-\sqrt{3}\]
    易得初始值$a_1=3, a_0=1$

    由待定系数法可得通项
    \[a_n = \frac{(3 + \sqrt{3})}{6}(2 + \sqrt{3})^n +\frac{(3 - \sqrt{3})}{6} (2 - \sqrt{3})^n\]
    
    \item (2) 设铺设 \( 1 \times n \) 的路径所有可能方案中使用的砖数为$b_n$,同理对最后一块砖分类讨论:
    
    如果最后一块砖是 \( 1 \times 1 \) 的方砖,对应的总砖数是$b_{n-1} + a_{n-1}$;

    如果最后一块砖是 直角边长为 1 的等腰直角三角形砖,有两种情况:
    
    如果倒数第二块砖也是 直角边长为 1 的等腰直角三角形砖,则对应的总砖数是$2\times(2\times a_{n-1} + b_{n-1})$,因为两个三角砖铺设的对角线有两种方向;

    如果倒数第二块砖是斜边长为 2 的等腰直角三角形砖,且连续铺了 $k$ 块斜边长为 2 的等腰直角三角形砖后变成与 直角边长为 1 的等腰直角三角形砖相接,对应的总砖数是$2 \times((k+2)\times a_{n-k-1} + b_{n-k-1}), \quad (1 \leq k \leq (n-1))$,三角砖同样有两种方向。

    因此有递推关系
    \begin{align}
    b_n &= b_{n-1} + a_{n-1} + 4a_{n-1} + 2b_{n-1} + 2\sum_{k=1}^{n-1} (k+2)\times a_{n-k-1} + 2\sum_{k=1}^{n-1} b_{n-k-1} \\
    &= a_{n-1} + b_{n-1} + 2\sum_{k=0}^{n-1} [b_{k}+(n+1-k)a_{k}]
    \end{align}

    则\[b_{n-1} =  a_{n-2} + b_{n-2} + 2\sum_{k=0}^{n-2} [b_{k}+(n-k)a_{k}]\]
    \[b_n - b_{n-1} = a_{n-1} + b_{n-1} - a_{n-2} - b_{n-2} + 2b_{n-1} + 4a_{n-1}+2\sum_{k=0}^{n-2}a_{k}\]
    即\[b_n = 4b_{n-1} + 3a_{n-1} - b_{n-2} - a_{n-2} + 2\sum_{k=0}^{n-1}a_{k}\]
    \[b_{n-1} = 4b_{n-2} + 3a_{n-2} - b_{n-3} - a_{n-3} + 2\sum_{k=0}^{n-2}a_{k}\]
    \[b_n-b_{n-1} = 4b_{n-1} + 3a_{n-1} - b_{n-2} - a_{n-2} - 4b_{n-2} - 3a_{n-2} + b_{n-3} + a_{n-3} + 2a_{n-1}\]
    即\[b_n = 5b_{n-1}-5b_{n-2} + b_{n-3} + 5a_{n-1} -4a_{n-2} + a_{n-3}\]
    代入\[a_n-4a_{n-1}+a_{n-2}=0\]
    \[5a_{n-1} -4a_{n-2} + a_{n-3} = 4a_{n-1}\]
    即\[b_n - 5b_{n-1} + 5b_{n-2} - b_{n-3} = 4a_{n-1}\]
    代入\[4a_n-4\times 4a_{n-1}+4a_{n-2}=0\]
    可得齐次递推式
    \[
    b_n - 9b_{n-1} + 26b_{n-2} - 26b_{n-3} + 9b_{n-4} - b_{n-5} = 0.
    \]

    解特征方程
    \[
    x^5-9x^4+26x^3-26x^2+9x-1 = 0.
    \]
    即
    \[
    (x - 1)\left(x^2 - 4x + 1\right)^2 = 0.
    \]
    
    由枚举法得初始值
    \[
    b_0 = 0, \quad b_1 = 5, \quad b_2 = 36, \quad b_3 = 199, \quad b_4 = 984.
    \]
    
    由待定系数法可得通项
    \[
    b_n = \left(\frac{2 + \sqrt{3}}{3} n + \frac{\sqrt{3}}{18}\right)(2 + \sqrt{3})^n 
    + \left(\frac{2 - \sqrt{3}}{3} n - \frac{\sqrt{3}}{18}\right)(2 - \sqrt{3})^n.
    \]
\end{itemize}


\end{document}
