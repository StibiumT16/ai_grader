\documentclass[fontset=windows]{article}
\usepackage[margin=1in]{geometry}
\usepackage{ctex}
\usepackage{setspace}
\usepackage{lipsum}
\usepackage{amsmath}
\usepackage{amsfonts}
\setcounter{section}{1}
\setcounter{subsection}{24}

\title{\heiti\zihao{2} 组合数学第一章作业}
\author{\songti Ch1-1B-8}

\begin{document}
\begin{sloppypar}
\maketitle
\thispagestyle{empty}

\subsection{}
$$\sum_{A \in S}\min A=\sum_{i=1}^{n-r+1}{i\binom{r-1}{n-i}}=\sum_{i=1}^{n-r+1}{\binom{1}{i}\binom{r-1}{n-i}}$$\par
可以看作将这n个数字用一个隔板隔开,左侧选1个数,右侧选r-1个数的方案数,即n+1个位置中选r+1个填充,故得
$$\sum_{A \in S}\min A=\binom{r+1}{n+1}$$
评:4。要考虑问题转换。

\subsection{}
可视为除了pdogq外的21个字母和3个空格作排列,最后再把空格按顺序换成p, dog, q的方案数。
$$\mbox{所求}=\frac{24!}{3!}=2024$$
评:3

\subsection{}
先放好abcde之外的空格数再填字母。除了要求的空格外还有71个空格,故问题可视为71个空格+5个隔板排列,再填入有重字母,得:
$$\frac{76!}{71!5!} \times \frac{95!}{(19!)^5}$$
评:3

\subsection{}
左式可视为$f(x)=x_1+x_2+...+x_m=n, x_m\geq1$的非负整数解数,右式则可视为方程$g(x)=x_1+x_2+...+x_{m-1}+(n-k)=k+(n-k), 0 \leq g(x) \leq n$的非负整数解数,相当于遍历$x_m$可能值并求相应情况下$x_1$到$x_{m-1}$的非负整数解数,故两者数量相同,等式成立。
评:4。给了思路后还是比较容易的,如果没提示用非负整数解视角解题会比较难联想。

\subsection{}
(1)全-3正1负-1正3负=C(9,4)-C(5,1)C(4,3)-C(5,3)C(4,1)=66 \par
(2)可重组合:4正+2正2负+4负=C(7,4)+C(5,2)C(6,2)+C(8,4)=255
评:3

\subsection{}
质因数分解$xyz=10^6=2^6 \times 5^6$,分别分配2和5,视作6个空格2个门:C(8,6)=28,可3正或一正2负,故所求为$28\times28\times4=3136$
评:4。2和5要分开计算;有陷阱,求整数解要考虑负数。

\subsection{}
(1)$n^{k+1}$:把k+1个不同的球放入n个不同的箱子的方案数。\par
(2)P(n,k):k个不同的球放到n个箱子,每个箱子最多放一个的方案数。\par
(3)kP(n,k):在P(n,k)基础上再放一个不同的球,该球要在已放球的箱子里选一个放\par
故$\frac{kP(n,k)}{n^{k+1}}=n^{n-k}\frac{kP(n,k)}{n^{n+1}}$表示放n+1个不同的球,从第k个开始有箱子放了超过一个球,可知其覆盖了所有(1)中的方案,故得证:
$$\sum_{k=1}^{n}{\frac{kP(n,k)}{n^{k+1}}}=1$$
评:5

\subsection{}
即8在$a_6$,7排在$a_4$,5只可能排在$a_2$或$a_3$,5在$a_2$时6不在$a_1$,5在$a_3$时6在$a_1$或$a_2$。
$$C(4,1)4!+C(2,1)4!=144$$
评:3

\subsection{}
选盒子:不相邻组合C(m-n+1,n),放球:$x_1+...+x_n=r-nk$的非负整数解个数C(n+r-nk-1,r-nk),故最终方案数为C(m-n+1,n)C(n+r-nk-1,r-nk)
评:3.5。分两段解题。

\subsection{}
随意排序方案数为$\frac{n!}{\prod\limits_{k=1}^{26}{a_k!}}$,可分3种情况 \par
(1)$a_k$都是偶数:$\frac{n!}{\prod\limits_{k=1}^{26}{a_k!}} - \frac{\frac{n}{2}!}{\prod\limits_{k=1}^{26}{\frac{a_k}{2}!}}$\par
(2)$a_k$中,$a_j$是奇数其余是偶数:$\frac{n!}{\prod\limits_{k=1}^{26}{a_k!}} - \frac{\frac{n-1}{2}!}{\frac{a_j-1}{2}!\prod\limits_{k \neq j}{\frac{a_k}{2}!}}$\par
(3)$a_k$中奇数大于1个:$\frac{n!}{\prod\limits_{k=1}^{26}{a_k!}}$
评:4。提示了从奇偶分析就不难。

\subsection{}
三阶魔方有8个角块和12个棱块,角块有3个方向,棱块有两个方向,故随机拼回有$8!\times3^8\times12!\times2^12$种可能,故合法状态应有$8!\times3^8\times11!\times2^{12}$个。
评:3.5。需要有三阶魔方的先验知识,知道之后就简单了。

\subsection{}
(1)147个人+3个隔板:C(150,3)=551300 \par
(2)相当于$x_A+x_B+x_C+x_D+x_nan=147$的非负整数解个数:C(147+5-1,4)=20811575\par
(3)加上条件$x_C\geq74$得$x_A+x_B+x_C+x_D+x_nan=73$:C(73+5-1,4)=1353275
评:3.5。层层添加条件就行。

\end{sloppypar}
\end{document}