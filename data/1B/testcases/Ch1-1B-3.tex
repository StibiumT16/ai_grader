%
% This is a borrowed LaTeX template file for lecture notes for CS267,
% Applications of Parallel Computing, UCBerkeley EECS Department.
% Now being used for CMU's 10725 Fall 2012 Optimization course
% taught by Geoff Gordon and Ryan Tibshirani.  When preparing
% LaTeX notes for this class, please use this template.
%
% To familiarize yourself with this template, the body contains
% some examples of its use.  Look them over.  Then you can
% run LaTeX on this file.  After you have LaTeXed this file then
% you can look over the result either by printing it out with
% dvips or using xdvi. "pdflatex template.tex" should also work.
%

\documentclass[UTF8,oneside]{article}

\usepackage[UTF8,scheme=plain]{ctex}
% \usepackage[AutoFakeBold,AutoFakeSlant]{xeCJK}  % 载入 xeCJK以支持中文,支持伪粗体,伪斜体
\usepackage[margin=1in]{geometry}
\usepackage{amsmath,amsthm,amssymb}
\usepackage{graphicx}
\usepackage{mathtools}
\usepackage{float, bm}

\setlength{\oddsidemargin}{0.25 in}
\setlength{\evensidemargin}{-0.25 in}
\setlength{\topmargin}{-0.6 in}
\setlength{\textwidth}{6.5 in}
\setlength{\textheight}{8.5 in}
\setlength{\headsep}{0.75 in}
\setlength{\parindent}{0 in}
\setlength{\parskip}{0.1 in}

%
% ADD PACKAGES here:
%

\usepackage{amsmath,amsfonts,graphicx}
\usepackage{etoolbox}
\AtBeginEnvironment{proof}{\normalsize}

%
% The following commands set up the lecnum (lecture number)
% counter and make various numbering schemes work relative
% to the lecture number.
%

%
% The following macro is used to generate the header.
%
\newcommand{\lecture}[4]{
   \pagestyle{myheadings}
   \thispagestyle{plain}
   \newpage
   \setcounter{page}{1}
   \noindent
   \begin{center}
   \framebox{
      \vbox{\vspace{2mm}
    \hbox to 6.28in { {\bf Combination Math
	\hfill 2024 Fall} }
       \vspace{4mm}
       \hbox to 6.28in { {\Large \hfill HW #1  \hfill} }
       \vspace{2mm}
       \hbox to 6.28in { {\it Student: #2 \hfill Time: #3} }
      \vspace{2mm}}
   }
   \end{center}
   \markboth{Lecture #1: #2}{Lecture #1: #2}

}
%
% Convention for citations is authors' initials followed by the year.
% For example, to cite a paper by Leighton and Maggs you would type
% \cite{LM89}, and to cite a paper by Strassen you would type \cite{S69}.
% (To avoid bibliography problems, for now we redefine the \cite command.)
% Also commands that create a suitable format for the reference list.
\renewcommand{\cite}[1]{[#1]}
\def\beginrefs{\begin{list}%
        {[\arabic{equation}]}{\usecounter{equation}
         \setlength{\leftmargin}{2.0truecm}\setlength{\labelsep}{0.4truecm}%
         \setlength{\labelwidth}{1.6truecm}}}
\def\endrefs{\end{list}}
\def\bibentry#1{\item[\hbox{[#1]}]}


\begin{document}
%FILL IN THE RIGHT INFO.
%\lecture{**LECTURE-NUMBER**}{**DATE**}{**LECTURER**}{**SCRIBE**}
\lecture{1}{Ch1-1B-3}{Oct. 8}
%\footnotetext{These notes are partially based on those of Nigel Mansell.}

% **** YOUR NOTES GO HERE:

% Some general latex examples and examples making use of the
% macros follow.
%**** IN GENERAL, BE BRIEF. LONG SCRIBE NOTES, NO MATTER HOW WELL WRITTEN,
%**** ARE NEVER READ BY ANYBODY.

\fontsize{12pt}{24pt}

\section{1.25}
\begin{proof}
    由题意得:
    \begin{align*}
        \sum_{A\in S} \operatorname{min} A &= 1\cdot C^{r-1}_{n-1} + 2\cdot C^{r-1}_{n-2} + \cdots + (n-r+1)\cdot C^{r-1}_{r-1}
    \end{align*}
    注意到$1\cdot C^{r-1}_{n-1}$是多项式$(1+x)^{n-1}$中$x^{r-1}$的系数,$2\cdot C^{r-1}_{n-2}$是多项式$2(1+x)^{n-2}$中$x^{r-1}$的系数,以此类推,考虑多项式:
    \begin{align*}
        & (1+x)^{n-1} + 2(1+x)^{n-2} + \cdots + (n-r+1)(1+x)^{r-1} \\
       = & \{(1+x)^{n-1} + \cdots + (1+x)^{r-1}\} + \{(1+x)^{n-2} + \cdots + (1+x)^{r-1} \} + \cdots + (1+x)^{r-1}\\
       = & (1+x)^{r-1}\cdot \frac{(1+x)^{n-r+1} - 1}{x} + \cdots + (1+x)^{r-1}\cdot \frac{(1+x)^1 - 1}{x} \\
       = & (1+x)^{r-1}\cdot \sum_{k=1}^{n-r+1} \frac{(1+x)^k - 1}{x} \\
       = & -(n-r+1)\frac{(1+x)^{r-1}}{x} + \frac{(1+x)^{r-1}}{x} \cdot \sum_{k=1}^{n-r+1} (1+x)^k \\
       = & -(n-r+1)\frac{(1+x)^{r-1}}{x} + \frac{(1+x)^{r-1}}{x} \cdot (1+x)\cdot \frac{(1+x)^{n-r+1}-1}{x} \\
       = & -(n-r+1)\frac{(1+x)^{r-1}}{x} + \frac{(1+x)^{n+1} - (1+x)^{r}}{x^2}
    \end{align*}
    注意到在求和后的结果中,$x^{r-1}$的系数仅在$\frac{(1+x)^{n+1}}{x^2}$中存在,易得结果为$\binom{n+1}{r+1}$。
\end{proof}

\section{1.26}
考虑``dog''为一个元素,首先不考虑p,dog,q的顺序问题,则原问题等价于24个元素全排列,共有$A^{24}_{24}$种方案,根据题目要求最终排列中必须满足p,dog,q的顺序,故总共有$A^{24}_{24}/3$种方案。

\section{1.27}
假设$a$之前有$x_1$个字母,$a,b$之前有$x_2$个,$b,c$之前有$x_3$个,$c,d$之前有$x_4$个,$d,e$之前有$x_5$个,$e$之后有$x_6$个,于是:
\[
    x_1 + x_2 + x_3 + x_4 + x_5 + x_6 + 5 = 100
\]
其中$x_2\ge 3, x_3\ge 5, x_4\ge 7, x_5\ge 9$, 令$y_1=x_1, y_2 = x_2 - 3, y_3 = x_3 - 5, y_4 = x_4 - 7, y_5 = x_5-9, y_6=x_6$, 得:
\[
    y_1 + y_2 + y_3 + y_4 + y_5 + y_6 = 100 - 5 - 3 - 5 - 7 - 9 = 71
\]
问题转化为求上述非负整数解,共有$C^{71}_{76}$组。最后对剩下95个字母进行多重全排列,得到最终排列方案数目为:
\[
     \binom{76}{71} \cdot \frac{95!}{19!\cdot 19! \cdot 19! \cdot 19! \cdot 19!}
\]


\section{1.28}
考虑以下不定方程的所有非负整数解:
\[
    x_1 + x_2 + \cdots + x_n + x_{n+1} = m
\]
利用隔板法考虑$m$个1和$n$个板的全排列,得到等式左侧$C^{m}_{n+m}$。接着首先考虑$x_{n+1}\ge 1$时解的个数,此时:
\[
    x_1 + x_2 + \cdots + x_n + (x_{n+1} - 1) = m - 1
\]
解的个数为$C^{m-1}_{n+m-1}$, 接着考虑$x_{n+1}=0, x_n\ge 1$的解的个数:
\[
    x_1 + x_2 + \cdots + x_{n-1} + (x_{n} - 1) = m - 1
\]
解的个数为$C^{m-1}_{n+m-2}$, 以此类推,可知当考虑$x_{k+1}=\cdots=x_{n+1}=0, x_k\ge 1$时解的个数为$C^{m-1}_{k+m-2}$。由于每种情况下的解一定不会出现重复,故所有方程的解为:
\[
    \sum_{k=1}^{n+1} C^{m-1}_{k+m-2} = \sum_{k=0}^{n} C^{m-1}_{k+m-1} = C^{m}_{n+m}
\]

\section{1.29}
\subsection{(1)} $C^4_5 + C^2_5\cdot C^2_4 + C^4_4 = 66$.
\subsection{(2)} $5^4 + 5^2\cdot 4^2 + 4^4 = 1281$.


\section{1.30}
由于:
\[
    xyz = 2^6 \cdot 5^6
\]
令:
\[
    x = 2^{a_1}\cdot 5^{b_1}, y = 2^{a_2}\cdot 5^{b_2}, z = 2^{a_3}\cdot 5^{b_3}.
\]
于是问题转化为求下列不定方程的非负整数解:
\[
    a_1+a_2+a_3 = 6, b_1 + b_2 + b_3 = 6
\]
可得$(a_1, a_2, a_3)$共有$C^6_{8}=28$组解, 同理$(b_1, b_2, b_3)$也有28组解。最后考虑$x,y,z$中任意两数为负数或全为正数的情况,最终共有$28\times 28 \times (1 + C^2_3) = 3136$组整数解。

\section{1.31}
考虑如下游戏:开始有一个盒子和$n$个不同的白球,第一次在这$n$个球中取出一个放入盒子中(不放回),每次取到一个白球放入盒子后补充一个的红球。每个红球不同,下一次在再这$n$个球中取一个球放入盒子中,取到红球时就终止。第$k$次操作后这$n$个球中有$n-k$个不同的白球和$k$个不同的红球,第$k+1$次终止的概率为:
\[
    \frac{P_n^{k}\cdot P_{k}^1}{n^{k+1}} = \frac{k\cdot P_n^k}{n^{k+1}}
\]
由于到第$n+1$次时一定会终止(只剩红球),即这个游戏最终一定会终止,而另一方面这个游戏终止的概率等于在每一轮终止的概率之和,于是对$k$从1到$n$进行求和可得:
\[
    \sum_{k=1}^n \frac{k\cdot P_n^k}{n^{k+1}} = 1
\]

\section{1.32}
根据题意可知,$a_4 = 7, a_6 = 8$,$5$仅可在$a_2,a_3$的位置,故共有$C^1_2\cdot A^4_4=48$组解。

\section{1.33}
首先考虑抽取$n$个不相邻的盒子,利用插空法,去掉$n$个盒子后还剩余$m-n$个盒子,有$m-n+1$个空可插入,故选法共有$C_{m-n+1}^n$种。

其次考虑这$n$个盒子每个包含$x_1,\dots,x_n$个小球,故问题转化为:
\[
    x_1 + x_2 +\cdots+x_n = r, \forall i\in [n] \quad x_i \ge k
\]
令$y_i = x_i -r$,得:
\[
    y_1 + y_2 +\cdots+y_n = r - nk, \forall i\in [n] \quad y_i \ge 0
\]
可知$(y_1,\dots,y_n)$共有$C^{r-nk}_{n+r-nk-1}$组解。由于这$n$个盒子互不相同,于是最终结果为:
\[
    A^n_n\cdot C_{m-n+1}^n\cdot C^{r-nk}_{n+r-nk-1}.
\]

\section{1.34}
首先考虑这$n$个字母的多重全排列为:
\[
    \frac{n!}{\Pi_{k=1}^{26}(a_k!)}
\]
当存在超过1个$k\in[26]$, $a_k$为奇数时,此时一定无法出现回文串,此时所有合法字符串为:
\[
    \frac{n!}{\Pi_{k=1}^{26}(a_k!)}
\]

当有且仅有1个$k\in[26]$, $a_{k}$为奇数时,不失一般性,考虑$a_1$为奇数,此时回文串的数量为$(a_1-1)/2$个a以及$a_k/2$个第$k$个字母的多重全排列($k\neq 1$), 为:
\[
    \frac{((n-1)/2)!}{((a_1-1)/2)!\cdot \Pi_{k=2}^{26}(a_k/2)!}
\]
合法的数量即为:
\[
    \frac{n!}{\Pi_{k=1}^{26}a_k!} - \frac{((n-1)/2)!}{((a_1-1)/2)!\cdot \Pi_{k=2}^{26}(a_k/2)!}
\]

最后考虑所有$a_k$均为偶数,此时回文串的数量为:
\[
    \frac{(n/2)!}{\Pi_{k=1}^{26} (a_k/2)!}
\]
剩余合法数量为:
\[
    \frac{n!}{\Pi_{k=1}^{26}a_k!} - \frac{(n/2)!}{\Pi_{k=1}^{26} (a_k/2)!}
\]

\section{1.35}
考虑8个角的全排列,并且每个角可以旋转3次,共有$8!\cdot 3^8$种组合。考虑12个棱的全排列,每个棱可以转2次,共有$12!\cdot 2^{12}$种组合,故合法状态共有:
\[
    \frac{8!\cdot 12!\cdot 2^{12}\cdot 3^8}{12}.
\]

\section{1.36}
\subsection{(1)} 设$A,B,C,D$每个选项各有$x_1,x_2,x_3,x_4$个人选,则:
\[
    x_1 + x_2 + x_3 + x_4 = 147
\]
此时共可能有$C^{147}_{150}=551300$种结果。

\subsection{(2)} 假设有$k$位同学作答,结合题1.28证明的公式,结果为:
\[
    \sum_{k=0}^{147} C^{k}_{k+3} = \sum_{k=0}^{147} C^{3}_{k+3} = C_{151}^4 = 20811575
\]

\subsection{(3)} 由题意得,若此题算分,则$k$为奇数时$x_3 \ge (k+1)/2$, $k$为偶数时$x_3\ge k/2$, 最终结果为:
\[
    C^3_3 + C^3_4 + C^3_4 + \cdots + C^3_{76} + C^3_{76} = 1 + 2\cdot \sum_{k'=1}^{76} C^3_{k'+3} = 1 + 2\cdot C_{80}^4 = 3163161
\]


\end{document}





