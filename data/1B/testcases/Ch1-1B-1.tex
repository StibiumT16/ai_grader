\documentclass[a4paper]{ctexart}
\usepackage{geometry}		%页面设置
\usepackage{amsmath}		%数学公式
\usepackage{graphicx}		%图像
%\usepackage{subfig}		%子图像
\usepackage{listings}		%表单,用于插入代码
\usepackage{xcolor}		%颜色,用于插入代码
\usepackage{enumerate}	%编号
\usepackage{siunitx}		%带SI单位的数字
\usepackage{multirow}		%表格合并行
\usepackage{listings}
\usepackage{amssymb}
\usepackage{subfigure}

\usepackage{algorithm}
\usepackage{algorithmic}

%\usepackage[framed,numbered,autolinebreaks,useliterate]{mcode}
%\usepackage{textcomp} % 必须加上,否则报错

%\usepackage[section]{placeins}	%强制浮动体与章节对齐

\newcommand*{\dif}{\mathop{}\!\mathrm{d}}	%定义微分算子

%设置用于插入代码的表单格式
\definecolor{mygreen}{rgb}{0,0.6,0}
\definecolor{mygray}{rgb}{0.5,0.5,0.5}
\definecolor{mymauve}{rgb}{0.58,0,0.82}
\lstset{
backgroundcolor=\color{white},   				% choose the background color
language=Matlab,
basicstyle=\footnotesize\ttfamily,        		% size of fonts used for the code
columns=fullflexible,
breaklines=true,                 				% automatic line breaking only at whitespace
captionpos=b,                    				% sets the caption-position to bottom
tabsize=4,
commentstyle=\color{mygreen},    				% comment style
escapeinside={\%*}{*)},          				% if you want to add LaTeX within your code
keywordstyle=\color{blue},       				% keyword style
stringstyle=\color{mymauve}\ttfamily,     		% string literal style
frame=shadowbox,
rulesepcolor=\color{red!20!green!20!blue!20},	% identifierstyle=\color{red},
extendedchars=false,  %解决代码跨页时,章节标题,页眉等汉字不显示的问题
escapebegin=\begin{CJK*}{GBK}{hei},escapeend=\end{CJK*}      % 代码中出现中文必须加上,否则报错
numbers=left,
numberstyle=\tiny,
escapeinside=' ',
xleftmargin=2em,
xrightmargin=2em,
aboveskip=1em
%showstringspaces=flase,%不显示代码字符串中间的空格标记
}

%设置页边距
%\geometry{left = 2cm, right = 2cm, top = 3cm, bottom = 3cm}

%代码
%\begin{lstlisting}[language = c++]
%代码
%中文注释
%'\textcolor{mygreen}{注释}'
%\end{lstlisting}

%图像
%\begin{figure}[htbp]
%\centering
%\includegraphics[width = .7\textwidth]{文件名.jpg} 
%\caption{标题}
%\label{fig:myphoto}
%\end{figure}

%含子图像
%\begin{figure}[htbp]
%\centering
%\subfloat[小标题1]{\includegraphics[height = 5.5cm]{文件名1}}\hfill
%\subfloat[小标题2]{\includegraphics[height = 5.5cm]{文件名2}}\hfill
%\caption{标题}
%\end{figure}

%居中的表格
%\begin{center}
%\begin{tabular}{|c|c|c|}
%\hline
% & & \\
%\hline
%\end{tabular}
%\end{center}

%编号条目
%\begin{enumerate}[1)]		%中括号中可以设置编号格式
%\item
%\item
%\end{enumerate}

\title{组合数学第一次作业}
\author{Ch1-1B-1}
\date{\today}

\begin{document}

%封面页
\maketitle
\thispagestyle{empty}		%不编页码

%目录页
%\newpage
\tableofcontents
\thispagestyle{empty}		%不编页码

%正文
\newpage
\setcounter{page}{1}

我们选择\textbf{进阶B}的题目进行作答。
\section{Problem 1.25}
\subsection{Solution}
首先$\min A = k$且满足$A\in S$的数量为$\binom{n-k}{r-1}$,因此我们可以得到:
\begin{equation}
    \sum_{A\in S} \min A = \sum_{k=1}^{n-r+1} \binom{n-k}{r-1}
\end{equation}
\par
从另一个角度来看$\binom{n+1}{r+1}$,首先$r=1$的情况下该表达式显然成立,因此我们假设$r\geq 2$。
\par
我们考察$r$元子集$A$中第二小的元素,假设为$k+1$,那么$k+1$左侧的一个元素有$k$种选择,$k+1$右侧的元素有$\binom{(n+1)-(k+1)}{r-1}=\binom{n-k}{r-1}$种选择,因此我们有:
\begin{equation}
    \binom{n+1}{r+1} = \sum_{k=1}^{n-r+1} \binom{n-k}{r-1}
\end{equation}
\par
从而该问题得证!

\subsection{Judgment}
本题较为容易,实际上只是用一个经典组合恒等式$\binom{n+1}{r+1} = \sum_{k=m}^{n-r+m}\binom{k}{m}\binom{n-k}{r-m}$的$m=1$的特殊形式,我们在解答中提供了该组合恒等式的证明。

\section{Problem 1.26}
\subsection{Solution}
假设$p$左边有$x_1$个字母,$p$和$dog$之间有$x_2$个字母,$dog$和$q$之间有$x_3$个字母,$q$右边有$x_4$个字母,因此我们可以得到以下的不定方程:
\begin{equation}
    x_1 + x_2 + x_3 + x_4 = 21, x_i\geq 0,i = 1,2,3,4
\end{equation}
\par
根据不定方程非负解的个数可得有$\binom{21+4-1}{4-1} = \binom{24}{3}$,然后注意到这个序列是全排列,因此剩下21个位置的排列共有$21!$种,从而总的排列数是:
\begin{equation}
    \binom{24}{3}\times 21! = \frac{24!}{3!}
\end{equation}
\subsection{Judgment}
本题较为容易,直接利用不定方程的解个数公式即可解决

\section{Problem 1.27}
\subsection{Solution}
设$a$左边有$x_1$个字母,$a$和$b$中间有$x_2$个字母,$b$和$c$中间有$x_3$个字母,$c$和$d$之间有$x_4$个字母,$d$和$e$之间有$x_5$个字母,$e$右边有$x_6$个字母,从而根据题目意思列出不定方程:
\begin{equation}
    x_1+x_2+x_3+x_4+x_5+x_6=95, x_1 \geq 0, x_2\geq 3,x_3\geq 5, x_4\geq 7,x_5\geq 9, x_6\geq 0
\end{equation}
\par
稍作处理可以得到:
\begin{equation}
    x_1+(x_2-3)+(x_3-5)+(x_4-7)+(x_5-9)+x_6=71, x_1 \geq 0, x_2\geq 3,x_3\geq 5, x_4\geq 7,x_5\geq 9, x_6\geq 0
\end{equation}
\par
从而化成标准不定方程的形式,解的数目为$\binom{71+6-1}{6-1}=\binom{76}{5}$。
\par
然后注意到这95个位置的排列数为$\frac{95!}{(19!)^5}$,从而结论为:
\begin{equation}
    \binom{76}{5}\times\frac{95!}{(19!)^5}
\end{equation}
\subsection{Judgment}
该问题较为简单,直接使用不定方程的解杂揉多重排列即可解决
\section{Problem 1.28}
\subsection{Solution}
Proof: 考察不定方程
\begin{equation}
    x_1 + x_2 + \dots + x_m = k, k = 0,1,\dots, n, x_i\geq 0
\end{equation}
\par
注意到根据不定方程的非负数解的结论,上面的方程解的个数为
\begin{equation}
    \sum_{k=0}^{n}\binom{k+m-1}{m-1}
\end{equation}
\par
另一方面,非常容易的看出,该不定方程包含了所有了满足$x_1+x_2+\dots+x_m\leq n$的非负解,我们引入$x_{m+1}=n-\sum_{r=1}^{m}x_i\geq 0$,从而有$x_1+x_2+\dots+x_m+x_{m+1}=n$,根据不定方程的解的个数为$\binom{n+m+1-1}{m+1=1}=\binom{n+m}{m}$,得证
\subsection{Judgment}
该问题较为容易,两次应用不定方程的解即可解决问题
\section{Problem 1.29}
\subsection{Solution}
\subsubsection{problem 1}
分类讨论:
\begin{itemize}
    \item 4正0负:$1$
    \item 2正2负:$\binom{5}{2}\times\binom{4}{2}=60$
    \item 0正4负:$\binom{5}{4}=5$
\end{itemize}
\par
总计$1+60+5=66$种
\subsubsection{problem 2}
分类讨论:
\begin{itemize}
    \item 4正0负,$x_1+x_2+x_3+x_4=4$,可得排列数目为$\binom{4+4-1}{4-1}=35$
    \item 2正2负,同理可得排列数为$\binom{2+4-1}{4-1}\times\binom{2+5-1}{5-1}=150$
    \item 0正4负,同理为$\binom{5+4-1}{5-1}=70$
\end{itemize}
\par
因此总计$35+150+70=255$种
\subsection{Judgment}
本题较为容易,应用分类的方法,结合不定方程的非负解容易得到结果。
\section{Problem 1.30}
\subsection{Solution}
首先由于对称性,我们假设$x,y,z\geq 0$, 考虑负数的情况总的结果乘以4即可。首先$1000000=2^65^6$,我们对$x,y,z$进行分解可以得到$x=2^{x_1}5^{x_2},y=2^{y_1}5^{y_2},z=2^{z_1}5^{z_2}$,从而有$x_1+y_1+z_1=6,x_2+y_2+z_2=6$且解非负,从而根据不定方程的解我们可以得到解的个数为$(\binom{6+3-1}{3-1})^2=784$,从而总的结果为$784\times 4=3136$
\subsection{Judgment}
本题较为容易,通过质因数分解就可以化为不定方程解的问题,需要注意正负数的符号问题。
\section{Problem 1.31}
\subsection{Solution}
Proof:考察表达式$n^{k+1}, kP(n,k)$:
\begin{itemize}
    \item $n^{k+1}$表示$n$种不同的球放入$k+1$个盒子,且每个盒子只能放一个球的不同放法数量
    \item $kP(n,k)$表示$n$种不同的球放入前$k$个盒子,且每个盒子只能放一个球,且每个盒子中的球的种类不能相同。第$k+1$个盒子中放的球与前面$k$个盒子中的一个相同的放法数量
\end{itemize}
因此$\frac{kP(n,k)}{n^{k+1}}$表示的是在$n+1$个盒子中放球,第$k+1$个盒子中的球是第一次与前$k$个盒子中某个盒子中的球相同的概率,由鸽巢原理,$n+1$个盒子中放$n$个种类的球,一定有两个盒子的球是同一种类,所以该概率之和为$1$,从而该恒等式得证。
\subsection{Judgment}
该问题难度中等,需要非常巧妙的解释$n^{k+1}, kP(n,k)$的含义,并发现利用鸽巢原理能够保证有相同种类球的情况。
\section{Problem 1.32}
\subsection{Solution}
首先,显然的,5在7前面,7在8前面。那么假设5前面有$a_1$个数,5和7之间有$a_2$个数,7和8之间有$a_3$个数,8后面有$a_4$个数,首先显然的$a_4=2,a_3=1$,且$a_2$那里只能有数字$6$或者没有。也就是说$a_1\leq 2$。我们进行分类讨论:
\begin{itemize}
    \item 数字6出现在$a_1$位置:那么此时$a_1=2,a_2=0,a_3=1,a_4=2$, 此时数字6有2个位置,因此总的可能情况为$2\times 4!=48$
    \item 数字6出现在$a_2$位置:那么此时$a_1=1,a_2=1,a_3=1,a_4=1$, 总的可能情况为$4!=24$
    \item 数字6出现在$a_3,a_4$位置,那么此时$a_1=2,a_2=0,a_3=1,a_4=2$,此时数字6有3个位置,因此总的可能情况数为$3\times 4!=72$
\end{itemize}
因此总的结果是$48+24+72=144$种
\subsection{Judgment}
该题目较为简单,稍加分析再分类讨论即可解决
\section{Problem 1.33}
\subsection{Solution}
假设需要放入的$n$个盒子中放入的小球数量为$y_i,i=1,2,\dots,n$,且每个盒子间隔的盒子数量为$x_i,i=1,2,\dots,n,n+1$,注意此时$x_1,x_{n+1}$分别表示的是第一个盒子左边的盒子数量和第n个盒子右边的盒子数量,从而有如下不定方程:
\begin{equation}
    x_1 + x_2 + \dots + x_{n+1} = m-n, x_1\geq 0, x_i \geq 1,i = 2,3,\dots, n, x_{n+1} \geq 0
\end{equation}
\begin{equation}
    y_1 + y_2 + \dots + y_n = r, y_i\geq k, i = 1,2,\dots, n
\end{equation}
根据不定方程的解的个数可以得到上面两个不定方程解的个数分别是$\binom{m-n+1}{n}$和$\binom{r-nk+n-1}{n-1}$,从而结论是$\binom{m-n+1}{n}\binom{r-nk+n-1}{n-1}$
\subsection{Judgment}
本题较为容易,将题目翻译之后很容易发现是两个不定方程非负解的套壳的问题
\section{Problem 1.34}
\subsection{Solution}
首先非常明显的,所有可能的字符串数量为
\begin{equation}
    U = \frac{n!}{a_1!a_2!\dots a_{26}!}
\end{equation}
\par
我们只需要求出回文串的数量然后减去即可,这里需要分类讨论$a_i(i=1,2,\dots,26)$中奇数的数量。
\begin{itemize}
    \item $a_i$有超过一个奇数,显然无法构成回文串,$L=0$
    \item $a_i$恰好有一个奇数,我们假设序号为$a_w$,那么中间那个字母一定是第$w$个字母,从而前面$\frac{n-1}{2}$个字符串决定了回文串数量:$L=\frac{(\frac{n-1}{2})!}{(\frac{a_w-1}{2})!{\Pi_{j=1,j\neq w}^{26}((\frac{a_j}{2})!)}}$
    \item $a_i$中没有奇数,通上我们容易根据前$\frac{n}{2}$个字母构建回文串,回文串数量为$L=\frac{(\frac{n}{2})!}{{\Pi_{j=1}^{26}((\frac{a_j}{2})!)}}$
\end{itemize}
从而结论为U-T
\subsection{Judgment}
本题较为容易,直接根据分类讨论+多重组合得到结果
\section{Problem 1.35}
\subsection{Solution}
注意到一个三阶魔方共有8个角块和12个棱块,每个棱块有2个方向,每个角块有3个方向,因此总共的状态数为$8!\times 3^8\times 12!\times 2^{12}$,由于这里面的状态只有$\frac{1}{12}$的概率能够复原,因此合法状态数目为$\frac{1}{12}\times 8!\times 3^8\times 12!\times 2^{12}$
\subsection{Judgment}
这个问题较为容易,根据题目求出所有的可能状态数乘以概率即可
\section{Problem 1.36}
\subsection{Solution}
\subsubsection{problem 1}
假设A,B,C,D四个选项的作答人数分别是$x_1,x_2,x_3,x_4$,那么我们有$x_1+x_2+x_3+x_4=147,x_i\geq 0,i=1,2,3,4$,所以解的个数为$\binom{147+4-1}{4-1}=551300$
\subsubsection{problem 2}
假设A,B,C,D四个选项的作答人数分别是$x_1,x_2,x_3,x_4$,那么我们有$x_1+x_2+x_3+x_4\leq 147,x_i\geq 0,i=1,2,3,4$, 我们引入$x_5=147-x_1-x_2-x_3-x_4\geq 0$,从而我们转化成$x_1+x_2+x_3+x_4+x_5=147,x_i\geq 0,i=1,2,3,4,5$,从而结果为$\binom{147+5-1}{5-1}=20811575$
\subsubsection{problem 3}
同理我们转化该问题$x_1+x_2+x_3+x_4\leq 147, x_1\geq 0, x_2\geq 0, x3\geq 74, x_4\geq 0$,引入$x5 = 147-x_1-x_2-x_3-x_4\geq 0$在稍作处理可以得到
\begin{equation}
    x_1+x_2+(x_3-74)+x_4+x_5=73,x_1\geq 0, x_2\geq 0,x_3-74\geq 0, x_4\geq 0,x_5\geq 0
\end{equation}
\par
从而根据不定方程解的个数可以容易得到解的数目为$\binom{73+5-1}{5-1}=1353275$
\subsection{Judgment}
本题较为容易,直接用不定方程非负解公式即可。
\end{document}

