\documentclass{article}

% Language setting
% Replace `english' with e.g. `spanish' to change the document language
\usepackage[english]{babel}
\usepackage{ctex}

% Set page size and margins
% Replace `letterpaper' with `a4paper' for UK/EU standard size
\usepackage[letterpaper,top=2cm,bottom=2cm,left=3cm,right=3cm,marginparwidth=1.75cm]{geometry}

% Useful packages
\usepackage{amsmath}
\usepackage{graphicx}
\usepackage[colorlinks=true, allcolors=blue]{hyperref}

\title{第一章作业-进阶B}
\author{Ch1-1B-16}

\begin{document}
\maketitle

\section*{1.25}
\[\sum_{A\in S} minA\]考虑minA为1时,可以在余下的n-1和数中取r-1个构成r元集合,接着取minA为2等等,由此递推,可以转换为\[\sum_{k=1}^{n-r+1} k*\binom{n-k}{r-1}\]那么该问题转化为如何将该求和式化简
\[\sum_{k=1}^{n-r+1} k\binom{n-k}{r-1}=\sum_{l=r-1}^{n-1} (n-l)\binom{l}{r-1}=\sum_{l=r-1}^{n-1}(\sum_{i=1}^{n-l}1)\binom{l}{r-1}\]
双重求和交换求和顺序将其表示为:
\[\sum_{i=1}^{n-r+1}\sum_{l=r-1}^{n-i}\binom{l}{r-1}=\sum_{i=1}^{n-r+1}\binom{n-i+1}{r}\]
令j=n-i+1,原式改写为:
\[\sum_{j=r}^n \binom{j}{r} = \binom{n+1}{r+1}
\]
\section*{1.26}
将dog视为整体,共24个元素进行全排列,再处于p,dog,q的相对顺序数
\[24!/3!=4*23!=103,408,066,955,539,906,560,000\]
\section*{1.27}
abcde五个字母将整列分割成6部分,每部分多少个数字记为$x_1,x_2,...x_6$,其中$x_2\geq3,x_3\geq5,x_4\geq7,x_5\geq9$记$x_2-3=y_2,x_3-5=y_3,x_4-7=y_4,x_5-9=y_4$,有$x_1+y_2+y_3+y_4+y_5+x_6=71$,有$\binom{71+6-1}{6-1}$种,此时再考虑fghij的排列数共$\frac{95!}{(19!)^5}$种,总排列数为$\binom{76}{5}\frac{95!}{(19!)^5}$
\section*{1.28}
$\binom{n+m}{n}$即$x_1+x_2+...+x_{m+1}=n$的非负整数解数量,$\binom{k+m-1}{m-1}$即$x_1+...+x_m=k$的非负整数解数量,我们将k看作$n-x_{m+1}$,k可以取0到n的值,所以求和后两式相等
\section*{1.29}
\subsection*{(1)}
两正两负、全正或全负,$\binom{4}{4}+\binom{4}{2}\binom{5}{2}+\binom{5}{4}=66$
\subsection*{(2)}
\[\binom{4+4-1}{4}+\binom{5+2-1}{2}\binom{4+2-1}{2}+\binom{5+4-1}{4}=255\]
\section*{1.30}
质因数分解$xyz=2^6*5^6$,可以转化为xyz中2和5的指数求和的非负整数解的问题,组合数为
\[\binom{6+3-1}{3-1}\binom{6+3-1}{3-1}=784\]
\section*{1.31}
考虑$\frac{P(n,k)}{n^{k}}\frac{k}{n}$的含义,可以表示n个不同元素中有放回地取一些元素,取到第k+1个时与之前的元素出现重复的概率,对这个进行求和代表着所有可能的第一次发生重复的位置的概率之和,到n就穷尽了,所以有
\[\sum_{k=1}^n \frac{k·P(n,k)}{n^{k+1}}=1\]
\section*{1.32}
8最大,中介数为2,所以8在第六个;7中介数为3,7不能在8的右边,在左边的话,7在第4个位置;5的中介数是3,5只能放在第2个或第3个,第3个时6在左边两个空位中,2*4!,第2个时6在右边4个空位中4*4!,总共144种
\section*{1.33}
m中选n个不相邻盒子,共$\binom{m-n+1}{n}$种;r个相同球放在n个不同盒子里,共$\binom{r-nk+n-1}{n-1}$种;所以总共有$\binom{m-m+1}{n}\binom{r-nk+n-1}{n-1}$种
\section*{1.34}
总的排列数为$\frac{n!}{a_1!a_2!...a_{26}!}$,需要减去回文串的排列数
当n为偶数时,考虑某半侧的情况,结果为\[\frac{n!}{a_1!a_2!...a_{26}!}-\frac{(\frac{n}{2})!}{(\frac{a_1}{2})!...(\frac{a_{26}}{2})!}\]
当n为奇数时,出现次数为奇数的字母需要有一个在字符串中间,所以只能有且仅有一个k使$a_k$为奇数,结果为\[\frac{n!}{a_1!a_2!...a_{26}!}-\frac{(\frac{n-1}{2})!}{(\frac{a_1}{2})!...(\frac{a_{k}-1}{2})!...(\frac{a_{26}}{2})!}\]
\section*{1.35}
可以拼回的块有8个角和12个边,角3个方向,边2个方向,合法状态为$8!*12!*3^8*2^{12}/12$种
\section*{1.36}
\subsection*{(1)}
\[\binom{147+4-1}{3}=551300\]
\subsection*{(2)}
\[\binom{147+5-1}{4}=20811575\]
\subsection*{(3)}
考虑非负整数解问题,选c的不少于74个
\[\binom{73+5-1}{4}=1353275\]
\end{document}