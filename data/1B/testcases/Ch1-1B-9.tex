\documentclass{article}
\usepackage{amsmath,amsfonts,amsthm,amssymb}
\usepackage{setspace}
\usepackage{fancyhdr}
\usepackage{lastpage}
\usepackage{extramarks}
\usepackage{chngpage}
\usepackage{soul,color}
\usepackage{graphicx,float,wrapfig}
%\usepackage{CJK}
\usepackage{algorithmic}
\usepackage{graphicx}
\usepackage{cases}
\usepackage{tikz}
\usepackage[UTF8]{ctex}
\usetikzlibrary{trees}

\newcommand{\Class}{组合数学}

% Homework Specific Information. Change it to your own
\newcommand{\Title}{第一章习题(进阶B)}
\newcommand{\StudentName}{Ch1-1B-9}
\newcommand{\StudentNumber}{Ch1-1B-9}

% In case you need to adjust margins:
\topmargin=-0.45in      %
\evensidemargin=0in     %
\oddsidemargin=0in      %
\textwidth=6.5in        %
\textheight=9.0in       %
\headsep=0.25in         %

% Setup the header and footer
\pagestyle{fancy}                                                       %
\lhead{\StudentName}                                                 %
\chead{\Title}  %
\rhead{\firstxmark}                                                     %
\lfoot{\lastxmark}                                                      %
\cfoot{}                                                                %
\rfoot{Page\ \thepage\ of\ \protect\pageref{LastPage}}                          %
\renewcommand\headrulewidth{0.4pt}                                      %
\renewcommand\footrulewidth{0.4pt}                                      %

%%%%%%%%%%%%%%%%%%%%%%%%%%%%%%%%%%%%%%%%%%%%%%%%%%%%%%%%%%%%%
% Some tools
\newcommand{\enterProblemHeader}[1]{\nobreak\extramarks{#1}{#1 continued on next page\ldots}\nobreak%
	\nobreak\extramarks{#1 (continued)}{#1 continued on next page\ldots}\nobreak}%
\newcommand{\exitProblemHeader}[1]{\nobreak\extramarks{#1 (continued)}{#1 continued on next page\ldots}\nobreak%
	\nobreak\extramarks{#1}{}\nobreak}%

\newcommand{\homeworkProblemName}{}%
\newcounter{homeworkProblemCounter}%
\newenvironment{homeworkProblem}[1][Problem \arabic{homeworkProblemCounter}]%
{\stepcounter{homeworkProblemCounter}%
	\renewcommand{\homeworkProblemName}{#1}%
	\section*{\homeworkProblemName}%
	\enterProblemHeader{\homeworkProblemName}}%
{\exitProblemHeader{\homeworkProblemName}}%

\newcommand{\homeworkSectionName}{}%
\newlength{\homeworkSectionLabelLength}{}%
\newenvironment{homeworkSection}[1]%
{% We put this space here to make sure we're not connected to the above.
	
	\renewcommand{\homeworkSectionName}{#1}%
	\settowidth{\homeworkSectionLabelLength}{\homeworkSectionName}%
	\addtolength{\homeworkSectionLabelLength}{0.25in}%
	\changetext{}{-\homeworkSectionLabelLength}{}{}{}%
	\subsection*{\homeworkSectionName}%
	\enterProblemHeader{\homeworkProblemName\ [\homeworkSectionName]}}%
{\enterProblemHeader{\homeworkProblemName}%
	
	% We put the blank space above in order to make sure this margin
	% change doesn't happen too soon.
	\changetext{}{+\homeworkSectionLabelLength}{}{}{}}%

\newcommand{\Answer}{\ \\\textbf{Answer:} }
\newcommand{\Acknowledgement}[1]{\ \\{\bf Acknowledgement:} #1}

%%%%%%%%%%%%%%%%%%%%%%%%%%%%%%%%%%%%%%%%%%%%%%%%%%%%%%%%%%%%%


%%%%%%%%%%%%%%%%%%%%%%%%%%%%%%%%%%%%%%%%%%%%%%%%%%%%%%%%%%%%%
% Make title
\title{\textmd{\bf \Class: \Title}\\\normalsize\vspace{0.1in}}
\date{}
\author{\textbf{\StudentName}\ \ \ \ \StudentNumber}
%%%%%%%%%%%%%%%%%%%%%%%%%%%%%%%%%%%%%%%%%%%%%%%%%%%%%%%%%%%%%

\begin{document}
	\begin{spacing}{1.1}
		\maketitle \thispagestyle{empty}
		%\cite{}
		%%%%%%%%%%%%%%%%%%%%%%%%%%%%%%%%%%%%%%%%%%%%%%%%%%%%%%%%%%%%%
		% Begin edit from here
		
		\begin{homeworkProblem}[1.25]
			
			考虑r元子集中最小元为$i$($i=1,\dots,n-r+1$)的子集,相当于从$i+1$到$n$之间选$r-1$个,共有$C_{n-i}^{r-1}$种。因此左式=$\sum_{i=1}^{n-r+1}{i\cdot C_{n-i}^{r-1}}$.

			已知组合恒等式$C_{a}^{b}=C_{a-1}^{b-1}+\dots +C_{b-1}^{b-1}$,($a\geq b$),相当于有第一个人参加的组合数,加上第一个人不参加而第二个人参加的组合数,一直加到前$a-b$个人都不参加而第$a-b+1$个人参加的组合数。由此可得右式=$C_{n+1}^{r+1}=C_n^r+C_{n-1}^r+\dots+C_r^r=(C_{n-1}^{r-1}+\dots+C_{r-1}^{r-1})+(C_{n-2}^{r-1}+\dots+C_{r-1}^{r-1})+\dots+C_{r-1}^{r-1}=\sum_{i=1}^{n-r+1}{i\cdot C_{n-i}^{r-1}}=$左式.

		\end{homeworkProblem}
		
		\begin{homeworkProblem}[1.26]
			
			考虑p,dog,q,这三个部分的前后关系固定,剩余21个字母任意插空,因此全排列数量为$4\times 5 \times \dots \times 24 = \frac{24!}{3!}$.
			
		\end{homeworkProblem}
		
		\begin{homeworkProblem}[1.27]
			
			首先假设本题并非是f,g,h,i,j各19个,而是95个完全不同的符号。
			
			将a,b,c,d,e依次放好,进而根据其间距最低要求,确认ab之间最靠左的紧邻的3个字母,bc之间最靠左的紧邻的5个字母,cd之间最靠左的紧邻的7个字母,de之间最靠左的紧邻的9个字母,这$3,5,7,9$共计$24$个符号分别与a,b,c,d绑定。

			然后将剩余的95-24=71个符号插入到a之前、b之前、c之前、d之前、e之前和e之后这6个空中,相当于插5个板子,因此共有$C_{71+5}^5=C_{76}^5$种位置分布。

			而这95个符号在确定位置后的全排列共有$95!$种。
			回归本题背景,因为19个f没有区别,整体还需去重,除以$19!$,一共要除5次。
			
			综上,结果为$C_{76}^5\cdot\frac{95!}{(19!)^5}$.

		\end{homeworkProblem}
		
		\begin{homeworkProblem}[1.28]
			
			考虑不定方程$a_1+\dots+a_{m+1}=n$的非负整数解个数,$\forall i\in\{1,\dots,m+1\}$,$a_i\geq0$.

			可被视作n个1和m个板,第$i-1$个和第$i$个板之间1的数量即为$a_i$的解(第0个板和第$m+1$个板为左右边界),因此总解数为$C_{n+m}^{m}$.

			考虑另一种计数方式,分类讨论$a_1=j$,$j\in\{0,1,\dots,n\}$,则相应的其余m个未知数的解有$C_{n-j+m-1}^{m-1}$种,因此总解数为$\sum_{j=0}^{n}{C_{n-j+m-1}^{m-1}}$.记$k=n-j$,则上式等价于$\sum_{k=0}^{n}{C_{k+m-1}^{m-1}}$.由此得证。
			
		\end{homeworkProblem}
		
		\begin{homeworkProblem}[1.29]

			(1)

			乘积为正,则只能有0、2、4个数为负。

			0个数为负,$C_5^0=1$种。

			2个数为负,$C_5^2\times C_4^2=60$种。

			4个数为负,$C_5^4=5$种。

			因此总共66种。\\

			

			(2)

			乘积为正,则只能有0、2、4个数为负。

			0个数为负,则从4种正数中可重复的选出4个数,按照重复类型分共有$(4,0,0,0)$, $(3,1,0,0)$, $(2,2,0,0)$, $(2,1,1,0)$, $(1,1,1,1)$这五类,分别包含$A_4^1=4$,$A_4^2=12$,$C_4^2=6$,$A_4^1\times C_3^2=12$,$C_4^4=1$种,共计35种。

			2个数为负,则从4种正数中可重复的选出2个数,按照重复类型分共有$(2,0,0,0)$, $(1,1,0,0)$这两类,分别包含$A_4^1=4$,$C_4^2=6$种,共计10种。
			再从5种负数中可重复的选出2个数,按照重复类型分共有$(2,0,0,0,0)$, $(1,1,0,0,0)$这两类,分别包含$A_5^1=5$,$C_5^2=10$种,共计15种。
			因此总共150种。

			4个数为负,则从5种负数中可重复的选出4个数,按照重复类型分共有$(4,0,0,0,0)$, $(3,1,0,0,0)$, $(2,2,0,0,0)$, $(2,1,1,0,0)$, $(1,1,1,1,0)$这五类,分别包含$A_5^1=5$,$A_5^1\times C_4^1=20$,$C_5^2=10$,$A_5^1\times C_4^2=30$,$C_5^4=5$种,共计70种。

			综上,总共255种。

		\end{homeworkProblem}

		\begin{homeworkProblem}[1.30]

			$xyz=2^6\times5^6$,相当于6个2、6个5分给3个不同的人,每个人可以不分到。

			考虑2的分配与5的分配独立,因此先考虑2的,分配方式共$(6,0,0)$, $(5,1,0)$, $(4,2,0)$, $(4,1,1)$, $(3,3,0)$, $(3,2,1)$, $(2,2,2)$这7类,分别包含$A_3^1=3$, $A_3^1\times C_2^1=6$, $C_3^2=3$, $A_3^1\times C_2^1=6$, $C_3^2=3$, $A_3^1\times C_2^1=6$, $C_3^3=1$种,共计28种。

			5的分配同理,因此综合起来共有$28\times28=784$种。

			此外,还需考虑负数的情况,上述每一种结果都可以对应于xyz三者符号分别为($+++$),($--+$),($-+-$),($+--$)这4种情况,因此总共有$784\times4=3136$种。

		\end{homeworkProblem}

		\begin{homeworkProblem}[1.31]

			由分析可得$\forall k \in \{1,\dots,n-1\}$,$\frac{A_n^{k+1}}{n^{k+1}}+\frac{k\cdot A_n^k}{n^{k+1}}=\frac{A_n^k\cdot(n-k)+k\cdot A_n^k}{n^{k+1}}=\frac{A_n^k}{n^k}$

			又因$\frac{nA_n^n}{n^{n+1}}+\frac{(n-1)A_n^{n-1}}{n^n}=\frac{A_n^{n-1}}{n^{n-1}}$可得

			
			\begin{align}
				\text{左式}& = \frac{nA_n^n}{n^{n+1}}+\frac{(n-1)A_n^{n-1}}{n^n}+\frac{(n-2)A_n^{n-2}}{n^{n-1}}+\dots+\frac{1\cdot A_n^1}{n^2} \\
				& = \frac{A_n^{n-1}}{n^{n-1}} + \frac{(n-2)A_n^{n-2}}{n^{n-1}} + \dots + \frac{1\cdot A_n^1}{n^2} \\
				& = \dots \\
				& = \frac{A_n^2}{n^2} + \frac{1\cdot A_n^1}{n^2} \\
				& = 1
			\end{align}
			

		\end{homeworkProblem}

		\begin{homeworkProblem}[1.32]
			
			因为5与7的中介数相等,因此7一定在5的右侧,否则7的中介数至少比5的大一。由于8为最大的数,且其中介数为2,因此8一定在7的右侧,且8右侧仅有两个空位。6的位置相对特殊,因此根据它进行分类。

			若6在5、7之间,则这八个数字分布只能为x567x8xx,其中x表示其余4个数字任意放,共有$A_4^4=24$种。

			若6在5之前,则6未使5、7的中介数加一,因此除去6外,数字分布为x57x8xx,四个x分别为1~4,6可插入作为第一个数或第二个数,因此共有$2\cdot A_4^4=48$种。

			若6在7之后,则数字分布为x5x7x8xx,其中6只可作为后三个x中的一个,其他1~4任意,因此共有$3\cdot A_4^4=72$种。

			综上,共有$24+48+72=144$种。

		\end{homeworkProblem}

		\begin{homeworkProblem}[1.33]

			首先选定盒子,因为每个盒子互不相邻,因此相当于前n-1个选定的盒子之后都还带一个盒子,其余$m-(2n-1)$个盒子被这n个"板"进行分隔,因此共有$C_{m-2n+1+n}^{n}$种。

			下面n个盒子每个都先放k个球,还剩$r-nk$个球,相当于在其间任意插入$n-1$个隔板,共有$C_{r-nk+n-1}^{n-1}$种。

			综上,共有$C_{m-n+1}^{n}\cdot C_{r-nk+n-1}^{n-1}$种。

		\end{homeworkProblem}

		\begin{homeworkProblem}[1.34]

			可组成的字符串种类共$\frac{n!}{a_1!\dots a_{26}!}$.下面只需删去回文串数量

			若各$a_i$中有大于等于两个为奇数,则没有回文串的可能,因为回文串只有在n为奇数时会因为最中心的字符而产生字符出现次数为奇数。因此此时非回文串数量就是$\frac{n!}{a_1!\dots a_{26}!}$.

			若各$a_i$均为偶数,则回文串的数量相当于用$\frac{a_1}{2}, \dots, \frac{a_{26}}{2}$构成的串数,因为回文串的后一半由前一半确定,因此非回文串数量为$\frac{n!}{a_1!\dots a_{26}!}-\frac{\frac{n}{2}!}{\frac{a_1}{2}!\dots \frac{a_{26}}{2}!}$.

			若各$a_i$中有一个为奇数,不妨设其为$a_j$,则回文串的数量相当于用$\frac{a_i}{2},\forall i \neq j$, 及$\frac{a_j-1}{2}$,因为回文串的后$\frac{n+1}{2}$位根据前$\frac{n-1}{2}$位及字符出现次数确定,因此非回文串数量为$\frac{n!}{a_1!\dots a_{26}!}-\frac{\frac{n-1}{2}!}{\frac{a_j-1}{2}!\cdot \prod_{i\neq j}{\frac{a_i}{2}!}}$.

		\end{homeworkProblem}

		\begin{homeworkProblem}[1.35]

			三阶魔方在拆下角块棱块后,根据六个面块即可确定唯一的摆放方向,因此固定魔方,无需考虑旋转,此时每个角块每个棱块地位均不相同。

			因此对于八个不同的角块,确定摆放位置共$A_8^8=8!$种,同时每个角块均可以扭转为三种方式放到同一位置上,因此共$8!\times3^8$种。

			同理,对于12个不同的棱块,确定摆放位置共$A_{12}^{12}=12!$种,同时每个棱块均可以扭转为两种方式放到同一位置上,因此共$12!\times2^{12}$种。

			又因为可复原概率为$\frac{1}{12}$,因此合法状态共有$\frac{8!\times3^8\times12!\times2^{12}}{12}=8!\times3^8\times11!\times2^{12}$种。

		\end{homeworkProblem}

		\begin{homeworkProblem}[1.36]

			(1)

			相当于147个相同的球放到4个不同的篮子中,因此是$C_{147+3}^3=C_{150}^3$.\\


			(2)

			相当于多一个选项E:未填,因此是$C_{147+4}^4=C_{151}^4$.\\


			(3)

			至少有74个人在C中,相当于73个相同球放到五个不同篮子中,因此是$C_{73+4}^4=C_{77}^4$.\\


		\end{homeworkProblem}
		% End edit to here
		%%%%%%%%%%%%%%%%%%%%%%%%%%%%%%%%%%%%%%%%%%%%%%%%%%%%%%%%%%%%%
		
	\end{spacing}
\end{document}

%%%%%%%%%%%%%%%%%%%%%%%%%%%%%%%%%%%%%%%%%%%%%%%%%%%%%%%%%%%%%
