% --------------------------------------------------------------
% This is all preamble stuff that you don't have to worry about.
% Head down to where it says "Start here"
% --------------------------------------------------------------
 
\documentclass[12pt]{article}
 
\usepackage[margin=1in]{geometry} 
\usepackage{amsmath,amsthm,amssymb}
\usepackage{ctex}
 
\newcommand{\N}{\mathbb{N}}
\newcommand{\Z}{\mathbb{Z}}
 
\newenvironment{theorem}[2][Theorem]{\begin{trivlist}
\item[\hskip \labelsep {\bfseries #1}\hskip \labelsep {\bfseries #2.}]}{\end{trivlist}}
\newenvironment{lemma}[2][Lemma]{\begin{trivlist}
\item[\hskip \labelsep {\bfseries #1}\hskip \labelsep {\bfseries #2.}]}{\end{trivlist}}
\newenvironment{exercise}[2][Exercise]{\begin{trivlist}
\item[\hskip \labelsep {\bfseries #1}\hskip \labelsep {\bfseries #2.}]}{\end{trivlist}}
\newenvironment{problem}[2][Problem]{\begin{trivlist}
\item[\hskip \labelsep {\bfseries #1}\hskip \labelsep {\bfseries #2.}]}{\end{trivlist}}
\newenvironment{question}[2][Question]{\begin{trivlist}
\item[\hskip \labelsep {\bfseries #1}\hskip \labelsep {\bfseries #2.}]}{\end{trivlist}}
\newenvironment{corollary}[2][Corollary]{\begin{trivlist}
\item[\hskip \labelsep {\bfseries #1}\hskip \labelsep {\bfseries #2.}]}{\end{trivlist}}

\newenvironment{solution}{\begin{proof}[Solution]}{\end{proof}}
 
\begin{document}
 
% --------------------------------------------------------------
%                         Start here
% --------------------------------------------------------------
 
\title{组合数学第一次作业}
\author{Ch1-1B-10}

\maketitle

注:以下对题目的评分范围为1-10分。
\begin{proof}[\textbf{1.25}]
    $S$是由 $\{1 \cdots n\}$ 全体 $r$ 元子集构成的集合,则我们有:
    \begin{align*}
        |\{\min A = i, A \in S\}| = C(n-i, r-1) \quad \forall 1 \leq i \leq n-r+1 
    \end{align*}
    从而:
    \begin{align*}
        \sum_{A \in S} \min A &= \sum_{i = 1}^{n-r+1} C(n-i, r-1) \times i \\
        &= \sum_{i = 1}^{n-r+1} C(n-i, r-1) \times (n + 1) - \sum_{i = 1}^{n-r+1} C(n-i, r-1) \times (n - i + 1)
    \end{align*}
    我们有:
    \begin{align*}
        \sum_{i = 1}^{n-r+1} C(n-i, r-1) \times (n + 1) &= (n+1)\sum_{i = 1}^{n-r+1} C(n-i, r-1) = (n+1)C(n,r)
    \end{align*}
     \begin{align*}
        \sum_{i = 1}^{n-r+1} C(n-i, r-1) \times (n -i + 1) &= \sum_{i = 1}^{n-r+1} r \times C(n-i+1, r)\\
        &= r\sum_{i = 1}^{n-r+1} C(n-i+1, r) = rC(n+1,r+1)
    \end{align*}
    从而可得:
    \begin{align*}
        \sum_{A \in S} \min A &= (n+1)C(n,r) - rC(n+1,r+1) \\
        &= (r+1)C(n+1,r+1) - rC(n+1,r+1) = C(n+1,r+1)
    \end{align*}
\end{proof}
评价:这道题关于min A的变化较为简单,但后面的公式证明在第一次处理时比较困难,需要一些技巧,在第一次处理类似问题时难度较大,很考验的组合公式的掌握,评分8分。

\begin{proof}[\textbf{1.26}]
    要求单词 dog 出现在排列中,则不妨将dog看成一个单独的字符,即对24个字符进行排列。而要求字母p在字母 q
前面、dog 在 p 和 q 之间,对总排列数除$P(3,3)$即可。\\
    从而满足上述条件的的排列方案数有 $P(24,24) \div P(3,3) = \frac{24!}{3!}$
\end{proof}
评价:这道题需要对排列组合的定义有一定的了解,在基础的变换后即可计算出,评分3分。

\begin{proof}[\textbf{1.27}]
    我们先选定a、b、c、d、e的位置,要求字母 a、b、c、d、e 按照字母表顺序排列,并且 a、b 之间相
隔至少 3 个字母,b、c 之间相隔至少 5 个字母,c、d 之间相隔至少 7 个字母,d、e 之间相
隔至少 9 个字母,该问题等价于,在(95-3-5-7-9 = 71)个相同小球中插入5个相同的挡板,共有$C(76,5)$种选法。

我们再确定f、g、h、i、j在去掉a、b、c、d、e后剩下95个位置的排法:共有$\frac{P(95,95)}{[P(19,19)]^5}$种。

综上,共有$\frac{C(76,5)\cdot P(95,95)}{[P(19,19)]^5}$种方案数。
\end{proof}
评价:本题考查对排列组合基本问题综合后的解决,一步步进行讨论分解即可,比较注重对基础知识的掌握,评分6分。

\begin{proof}[\textbf{1.28}]
    从不定方程非负整数解数目的视角来看:
    
    $C(n+m,m)=C(n+m,n)$对应$x_1 + x_2 + \cdots + x_{m+1} = n$的非负整数解个数;
    
    而由于$x_{m+1}$的值可以被$x_1\cdots x_m$唯一确定,故我们遍历$x_1 + x_2 + \cdots + x_m = k, k\in \{0,1,\cdots,n\}$的所有非负整数解同样一一对应到原方程的所有非负整数解。固定k的情况下,$x_1 + x_2 + \cdots + x_m = k$的非负整数解个数为$C(m+k-1.m-1)$。
    
    故对所有$k$求和我们可以得到 $C(n+m,m) = \sum_{k=0}^{n}C(m+k-1.m-1)$
\end{proof}
评价:此题考查组合数和不定方程非负整数解的对应关系,掌握对应关系后进行基本的变化即可得到证明,评分4分。

\begin{proof}[\textbf{1.29(1)}]
    最终的乘积为正数,则需选择0,2,4个正数,由于若选择的数字互不相同,下分别讨论:
    
    选择0个正数:$C(5,4) = 5$; 
    
    选择2个正数:$C(5,2) \times C(4,2) = 60$; 
    
    选择4个正数:$C(4,4) = 1$; 
    
    故共有66种方案。
\end{proof}
\begin{proof}[\textbf{1.29(2)}]
    最终的乘积为正数,则需选择0,2,4个正数,若选择的数字允许相同,即对应选择可重组合,下分别讨论:
    
    选择0个正数:$C(8,4) = 70$; 
    
    选择2个正数:$C(6,2) \times C(5,2) = 150$; 
    
    选择4个正数:$C(7,4) = 35$; 
    
    故共有255种方案。
\end{proof}
评价:本题考查组合和可重组合的基本定义,掌握基本定义后分类讨论可很快完成,评分3分。

\begin{proof}[\textbf{1.30}]
    不定方程$xyz = 1000000 = 2^6 5^6$,我们先考虑$x,y,z$的绝对值,即$|x| \times |y| \times |z| = 1000000 = 2^6 5^6$:

    该不定方程对应6个一样的红色小球和6个一样的蓝色小球放到三个不同的盒子中的放法,共计:$C(8,2) \times C(8,2) = 784$种;

    再考虑$x,y,z$的正负性,均为正数有1种可能,两负一正有$C(3,1) = 3$种可能,共计4种可能;

    故原方程的解个数共有$4 \times 784 = 3136$个。
\end{proof}
评价:本题需对不定方程进行一定的划归,划归后较为简单,重点是不定方程和组合问题的转换,评分5分。

\begin{proof}[\textbf{1.31}]
    \begin{align*}
        \sum_{k = 1}^n \frac{k\cdot P(n,k)}{n^{k+1}} = \sum_{k = 1}^n \frac{n\cdot P(n,k)}{n^{k+1}} - \sum_{k = 1}^n \frac{(n-k)\cdot P(n,k)}{n^{k+1}}
    \end{align*}
    而:
    \begin{align*}
        \sum_{k = 1}^n \frac{n\cdot P(n,k)}{n^{k+1}} = \sum_{k = 1}^n \frac{P(n,k)}{n^{k}}
    \end{align*}
    \begin{align*}
        \sum_{k = 1}^n \frac{(n-k)\cdot P(n,k)}{n^{k+1}} 
        &= \sum_{k = 1}^{n-1} \frac{(n-k)\cdot \frac{n!}{(n-k)!}}{n^{k+1}} \\
        &= \sum_{k = 1}^{n-1} \frac{P(n,k+1)}{n^{k+1}} \\
        &= \sum_{k = 2}^{n} \frac{P(n,k)}{n^{k}} 
    \end{align*}
    从而:
    \begin{align*}
        \sum_{k = 1}^n \frac{k\cdot P(n,k)}{n^{k+1}} = \sum_{k = 1}^n \frac{P(n,k)}{n^{k}} -  \sum_{k = 2}^{n} \frac{P(n,k)}{n^{k}} = \frac{P(n,1)}{n} = 1
    \end{align*}
\end{proof}
评价:该题的变换手法与1.25类同,第一次解决难度较大,掌握变换技巧后较为简单,评分8分。

\begin{proof}[\textbf{1.32}]
    为了表达方便,我们用粗体的$\textbf{1},\textbf{2},\cdots,\textbf{8}$表示数字在字符串中的位置,由于8比其他任何数都大,8 的中介数为 2说明8一定在\textbf{6}号位;同理,7比除了8以外的人后数都大,7 的中介数为 3说明7一定在\textbf{4}号位;而此时5还能选的位置为\textbf{1}、\textbf{2}、\textbf{3}、\textbf{5}、\textbf{7}、\textbf{8},由于5的中介数为3,那么\textbf{5}、\textbf{7}、\textbf{8}这三个位置必定不可能,中介数一定小于3,且\textbf{1}号位也不行,如果5在\textbf{1}号位那么它的中介数一定是4;下面分别对5在\textbf{2}号位和\textbf{3}号位进行分类讨论:

    若5在\textbf{2}号位:那么\textbf{1}号位一定要放一个比5小的数,剩下4个空位对剩下的数进行全排列即可——共有$C(4,1) \times P(4,4) = 96$种方案。

    若5在\textbf{3}号位:那么\textbf{1}号位和\textbf{2}号位一定是要放一个比5小的数和一个比5大的数排列,剩下3个空位对剩下的数进行全排列即可——共有$C(4,1) \times P(2,2) \times P(3,3) = 48$种方案。

    综上,总的方案数有144种。
\end{proof}
评价:该题需要对中介数的基本定义和性质有一定的掌握,然后熟练地运用分类和组合解决问题,评分6分。

\begin{proof}[\textbf{1.33}]
    先在m个不同的盒子中选n个不相邻的盒子,即选取不相邻组合,共有$C(m-n+1,n)$种选法;

    再将r个相同的小球放入这m个不同的盒子当中,每盒至少包含k个球,即考虑$x_1+x_2+\cdots+x_m = r, \forall x_i \geq k$的非负整数解个数,共有$C(r-nk+n-1,n-1)$种放法;

    综上,共有$C(m-n+1,n) \times C(r-nk+n-1,n-1)$种方案数。
\end{proof}
评价:本题考查组合问题中两个基础问题的结合,需分别熟练掌握两个问题的解法,评分5分。

\begin{proof}[\textbf{1.34}]
    要求不能是回文串,由于总的排列数为多重全排列$P(n,a_1,a_2,\cdots,a_{26})$,我们只需要考虑回文串的数量:

    若$a_1,a_2,\cdots,a_26$中有两个及以上的奇数,那么不存在回文串,因为数量为奇数的字母在回文串中必须要占据中心位置,而中心位置的数量为0或1(取决于n的奇偶),故此时合法的排列数就为$P(n,a_1,a_2,\cdots,a_{26})$。

    若$a_1,a_2,\cdots,a_{26}$中有且仅有一个奇数,记为$a_k$,那么此时n也为奇数。回文串中$a_k$对应的字母一定占据唯一的中心位置,之后两边对称排开,左右两边各有$\frac{n-1}{2}$个字母,其中$a_k$对应的字母的数量是$\frac{a_k-1}{2}$,其余字母的数量是$\frac{a_j}{2}$,此时回文串的数量为全排列$P(\frac{n-1}{2},\frac{a_1}{2},\cdots,\frac{a_{k-1}}{2},\frac{a_{k}-1}{2},\frac{a_{k+1}}{2},\cdots,\frac{a_{26}}{2})$,故此时合法的排列数就为$P(n,a_1,a_2,\cdots,a_{26}) - P(\frac{n-1}{2},\frac{a_1}{2},\cdots,\frac{a_{k-1}}{2},\frac{a_{k}-1}{2},\frac{a_{k+1}}{2},\cdots,\frac{a_{26}}{2})$

    若$a_1,a_2,\cdots,a_{26}$中没有奇数,那么此时n也为偶数。只需要考虑所有字母数量的一半的全排列再进行对称操作即可得到全部的回文串,即此时回文串的数量为全排列\\$P(\frac{n}{2},\frac{a_1}{2},\cdots,\frac{a_{k}}{2},\cdots,\frac{a_{26}}{2})$,故此时合法的排列数就为$P(n,a_1,a_2,\cdots,a_{26}) - P(\frac{n}{2},\frac{a_1}{2},\cdots,\frac{a_{k}}{2},\cdots,\frac{a_{26}}{2})$
\end{proof}
评价:该题需要对回文串的生成有一定掌握,巧妙运用奇偶性将问题简化,评分6分。

\begin{proof}[\textbf{1.35}]
    三阶魔方的6个中心块是固定不动的,将角块和棱块随机拼回:
    
    角块有8个,每个有三面,每面是不同的颜色,因此拼法有$P(8,8) \times 3^8$种;

    棱块有12个,每个有两面,每面是不同的颜色,因此拼法有$P(12,12) \times 2^{12}$种;

    因此总的拼法为$P(8,8) \times P(12,12) \times 2^{12} \times 3^8$,而其中合法的拼法占据$\frac{1}{12}$,故合法的拼法有 
    \begin{align*}
        P(8,8) \times P(12,12) \times 2^{12} \times 3^8 \times \frac{1}{12} &= P(8,8) \times P(12,12) \times 2^{10} \times 3^7 \\
        &= 43252003274489856000 
    \end{align*}
    约等于4325亿亿种合法状态。
\end{proof}
评价:该题给出了合法状态的比例,故只需要对魔方的结构有一定的了解即可,评分3分。

\begin{proof}[\textbf{1.36(1)}]
    所有同学都回答了此题,即考虑方程$A+B+C+D = 147$的非负整数解的个数,易得为$C(150,3) = 551300$
\end{proof}

\begin{proof}[\textbf{1.36(2)}]
    每位同学均可能没有回答此题,那么回答了问题的总人数可能为0至147,即考虑方程$A+B+C+D = k, k \in \{0, 1, \cdots, 147\}$的所有非负整数解的个数和,易得为$\sum_{k = 0}^{147} C(k+3, 3) = \sum_{k = 0}^{147} C(k+3, k) = C(151,147) = C(151,4) = 20811575$
\end{proof}

\begin{proof}[\textbf{1.36(3)}]
    每位同学均可能没有回答此题且选C的同学要占交卷人数的一半(即至少有74个同学选C),即考虑方程$A+B+C+D = k, k \in \{0, 1, \cdots, 147\}, C \geq 74$的所有非负整数解的个数和, 等价于考虑方程$A+B+C+D = k, k \in \{0, 1, \cdots, 73\}$的所有非负整数解的个数和,易得为$\sum_{k = 0}^{73} C(k+3, 3) = \sum_{k = 0}^{73} C(k+3, k) = C(77,73) = C(77,4) = 1353275$
\end{proof}
评价:本题考虑基本的不定方程组和组合问题的对应关系,掌握基本知识即可解答,评分4分。

 
% --------------------------------------------------------------
%     You don't have to mess with anything below this line.
% --------------------------------------------------------------
 
\end{document}
